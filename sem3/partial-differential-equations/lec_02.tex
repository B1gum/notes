\lecture{2}{5. September 2025}{Even and odd functions}

\exercise{1.} Find a period and the corresponding Fourier coefficients of

\paragraph{a)} $f(x) = \sin \left( \frac{\pi}{4} \right) \sin \left( \frac{2x}{4} \right) + \cos \left( \frac{\pi}{5}\right) \sin \left( \frac{3x}{7} \right)$
\bigbreak
We rewrite the given function as:
\begin{align*}
  f(x) &= \sin \left( \frac{\pi}{4} \right) \sin \left( \frac{2x}{4} \right) + \cos \left( \frac{\pi}{5} \right) \sin \left( \frac{3x}{7} \right) \\
  &= \sin \left( \frac{\pi}{4} \right) \sin \left( \frac{7 \pi x}{14 \pi} \right) + \cos \left( \frac{\pi}{5} \right) \sin \left( \frac{6 \pi x}{14 \pi} \right)
.\end{align*}
We will now compare this to the general formulation of the Fourier series as:
\[ 
f(x) = a_0 + \sum_{n = 1}^{\infty} \left( a_n \cos \left( \frac{n \pi x}{L} \right) + b_n \sin \left( \frac{n \pi x}{L} \right) \right)
.\]
We can quickly see that $L = 14 \pi \implies p = 28 \pi$, $b_6 = \cos \left( \frac{\pi}{5} \right)$, $b_7 = \sin \left( \frac{\pi}{4} \right)$. All other unmentioned Fourier coefficients are zero.


\paragraph{b)} $f(x) = \sin \left( \sqrt{2x} \right) + \cos \left( \frac{x}{\sqrt{2}} \right)$
\bigbreak
We once again start by rewriting as:
\begin{align*}
  f(x) &= \sin \left( \sqrt{2x} \right) + \cos \left( \frac{x}{\sqrt{2}} \right) \\
       &= \sin \left( \frac{2\pi x}{\sqrt{2}\pi} \right) + \cos \left( \frac{\pi x}{\sqrt{2}\pi} \right)
.\end{align*}
Comparing this with the general formulation of the Fourier series:
\[ 
f(x) = a_0 + \sum_{n = 1}^{\infty} \left( a_n \cos \left( \frac{n \pi x}{L} \right) + b_n \sin \left( \frac{n \pi x}{L} \right) \right)
.\]
We see that $L = \sqrt{2} \pi \implies p = 2 \sqrt{2} \pi$, $a_1 = 1$, $b_2 = 1$. All other Fourier coefficients are zero.

\paragraph{c)} $f(x) = \sin \left( \frac{\pi}{3} + \frac{3 \pi x}{4} \right)$
\bigbreak
We use that
\[ 
\sin (\alpha + \beta) = \sin \alpha \cos \beta + \sin \beta \cos \alpha
\]
to rewrite the equation as:
\begin{align*}
  f(x) &= \sin \left( \frac{\pi}{3} + \frac{3\pi x}{4}\right) \\
  &= \sin \frac{\pi}{3} \cdot \cos \frac{3\pi x}{4} + \sin \frac{3\pi x}{4} \cdot \cos \frac{\pi}{3}
.\end{align*}
Comparing this to the general formulation of the Fourier series
\[ 
f(x) = a_0 + \sum_{n = 1}^{\infty} \left( a_n \cos \left( \frac{n \pi x}{L} \right) + b_n \sin \left( \frac{n \pi x}{L} \right) \right)
\]
we see that $L = 4 \implies p = 8$, $a_3 = \sin \frac{\pi}{3}$, $b_3 = \cos \frac{\pi}{3}$. All other Fourier coefficients are zero

\exercise{2.} Are the following functions even or odd? Give arguments.

\paragraph{a)} $f(x) = \left( \cos x^2 + \cos x \right) \sin x$
\bigbreak
We check if the function is even or odd by checking what happens for an input of $-x$ as:
\begin{align*}
  f(-x) &= \left( \cos (-x)^2 + \cos (-x) \right) \sin (-x) \\
  &= \left( \cos x^2 + \cos (-x) \right) (- \sin x) \\
  &= - f(x)
.\end{align*}
Hence the function is odd.

\paragraph{b)} $f(x) = \frac{1 + \cos x}{2 + \cos x^2}$
\bigbreak
We follow the same procedure:
\begin{align*}
  f(-x) &= \frac{1 + \cos (-x)}{2 + \cos (-x)^2} \\
  &= \frac{1 + \cos x}{2 + \cos x} \\
  &= f(x)
.\end{align*}
Hence the function is even.

\paragraph{c)} $f(x) = \frac{1 + \sin x}{2 + \sin x^2}$
\bigbreak
The same procedure is applied:
\begin{align*}
  f(-x) &= \frac{1 + \sin (-x)}{2 + \sin (-x)^2} \\
  &= \frac{1 -\sin x}{2 + \sin x} \\
  &\neq \begin{cases}
  f(x)\\
  -f(x)
  \end{cases}
\end{align*}
Hence the function is neither even nor odd.

\paragraph{d)} $f(x) = x |x|$
\bigbreak
The same procedure is applied:
\begin{align*}
  f(-x) &= (-x) \left| -x \right| \\
        &= -x \left| x \right| \\
        &= -f(x)
.\end{align*}
Hence the function is odd.


\exercise{3.} Consider the periodic function
\begin{align*}
  f(x) &= \begin{cases}
    x, & -2 < x < -1 \\
    x + k, & -1 < x < 1 \\
    x, & 1 < x < 2\\
  \end{cases} \\
    f(x) &= f(x+p), p = 4
\end{align*}
where $k$ is a constant. Decompose $f(x)$ into its even and odd part.
\bigbreak
We apply the definition of decomposition into odd and even parts for the first interval, $-2<x< -1$ as:
\begin{align*}
  f_1(x) &= \frac{1}{2} (f(x) + f(-x)) = \frac{1}{2}(x + (-x)) = 0 \\
  f_2 (x) &= \frac{1}{2} (f(x) - f(-x) = \frac{1}{2} (x - (-x)) = x
.\end{align*}
For the second interval, $-1 < x < 1$ we get
\begin{align*}
  f_1(x) &= \frac{1}{2} (f(x) + f(-x)) = \frac{1}{2}((x+k) + (-x + k)) = k \\
  f_2 (x) &= \frac{1}{2} (f(x) - f(-x) = \frac{1}{2} ((x+k) - (-x + k)) = x
.\end{align*}
And for the last interval, $1 < x < 2$ we get:
\begin{align*}
  f_1(x) &= \frac{1}{2} (f(x) + f(-x)) = \frac{1}{2}(x + (-x)) = 0 \\
  f_2 (x) &= \frac{1}{2} (f(x) - f(-x) = \frac{1}{2} (x - (-x)) = x
.\end{align*}
We thus have:
\begin{align*}
  f(x) &= f_1(x) + f_2(x) \\
  f_1(x) &= \begin{cases}
  0 & -2 < x < -1\\
  k & -1 < x < 1 \\
  0 & 1<x<2
  .\end{cases} \\
  f_1(x) &= f_1(x+p), \quad p = 4 \\
  f_2(x) &= x, \qquad -2 < x < 2 \\
  f_2(x) &= f_2(x+p), \quad p = 4
.\end{align*}

\exercise{4.} Consider the periodic function
\begin{align*}
  f(x) &= \begin{cases}
  0, & -\pi < x < 0\\
  2 x, & 0 < x < \pi
  \end{cases} \\
  f(x) &= f(x+p), p = 2\pi
.\end{align*}
Decompose $f(x)$ into its even and odd part and find its Fourier series.

Hint: You may want to use examples from the lecture.
\bigbreak
For the interval $-\pi < x < 0$ we have
\begin{align*}
  f_1(x) &= \frac{1}{2} (0 + (-2x)) = -x \\
  f_2(x) = \frac{1}{2} (0 - (-2x)) = x
.\end{align*}
And for $ 0 < x < \pi$ we have
\begin{align*}
  f_1 (x) &= \frac{1}{2} (2x + 0) = x \\
  f_2(x) &= \frac{1}{2} (2x - 0) = x
.\end{align*}
We thus have 
\begin{align*}
  f(x) &= f_1(x) + f_2(x) \\
  f_1(x) &= |x|, -\pi < x < \pi \\
  f_1(x) &= f_1(x+p), \quad p = 2\pi \\
  f_2(x) &= x, \quad -\pi < x < \pi \\
  f_2(x) &= f_2(x+p) \quad p = 2\pi
.\end{align*}
From the lecture we know that $f_1$ has the Fourier cosine series
\[ 
f_1(x) = \frac{\pi}{2} + \sum_{n = 1}^{\infty} \frac{2}{\pi n^2} \left( \cos n\pi - 1 \right) \cos nx
\]
and that $f_2$ has the Fourier sine series:
\[ 
f_2(x) = \sum_{n = 1}^{\infty} \left( -1 \right)^{n+1} \frac{2}{n} \sin nx
.\]
Combining these we get the Fourier series for $f(x)$ as:
\[ 
f(x) = \frac{\pi}{2} + \sum_{n = 1}^{\infty} \left( \frac{2}{\pi n^2} (\cos n \pi -1) \cos nx + (-1)^{n+1} \frac{2}{n} \sin nx \right)
.\]


\exercise{5.} Decompose
\[ 
f(x) = e^{ix}
\]
into its even and odd part.

Hint; you may want to use the representation of $f(x)$ in terms of $\cos x$ and $\sin x$.
\bigbreak
We apply the same procedure as in the last couple of exercises whilst remembering that:
\[ 
e^{ix} = \cos x + i \sin x
.\]
We therefore get:
\begin{align*}
  f_1(x) &= \frac{1}{2} (f(x) + f(-x)) \\
  &= \frac{1}{2} \left( \cos x + i \sin x + \cos x - i \sin x \right) \\
  &= \cos x \\
  f_2(x) &= \frac{1}{2} \left( f(x) - f(-x) \right) \\
  &= \frac{1}{2} \left( \cos x + i \sin x - (\cos x - i \sin x) \right) \\
  &= i \sin x
.\end{align*}
This also is what would be expected when looking at the representation of $e^{ix}$ introduced at the start of this answer.
