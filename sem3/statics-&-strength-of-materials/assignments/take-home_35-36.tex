\documentclass[a4paper]{article}

\usepackage[danish]{babel}
\usepackage{amsfonts, amssymb, mathtools, amsthm, amsmath}
\usepackage{graphicx, pgfplots}
\usepackage{url}
\usepackage[dvipsnames]{xcolor}
\usepackage{lastpage}

%loaded last
\usepackage[hidelinks]{hyperref}

\usepackage{siunitx}
  \sisetup{exponent-product = \cdot,
    output-decimal-marker = {,}}

%Giles Castelles incfig
\usepackage{import}
\usepackage{xifthen}
\usepackage{pdfpages}
\usepackage{transparent}

\newcommand{\incfig}[2][1]{%
  \def\svgwidth{#1\columnwidth}
  \import{./figures/}{#2.pdf_tex}
}

\setlength{\parindent}{0in}
\setlength{\parskip}{12pt}
\setlength{\oddsidemargin}{0in}
\setlength{\textwidth}{6.5in}
\setlength{\textheight}{8.8in}
\setlength{\topmargin}{0in}
\setlength{\headheight}{18pt}

\usepackage{fancyhdr}
\pagestyle{fancy}

\fancyhead{}
\fancyfoot{}
\fancyfoot[R]{Side \thepage{} af \pageref{LastPage}}
\fancyhead[L]{\footnotesize{Noah Rahbek Bigum Hansen}}

% Redefine the plain page style to be consistent
\fancypagestyle{plain}{
  \fancyhead{} % Clears all header content
  \fancyfoot{} % Clears all footer content
  \renewcommand{\headrulewidth}{0pt} % Removes the horizontal line
  \fancyfoot[R]{Side \thepage{} af \pageref{LastPage}} % Page number in the footer
}
\pgfplotsset{compat=newest}

\pgfplotsset{every axis/.append style={
  axis x line=middle,    % put the x axis in the middle
  axis y line=middle,    % put the y axis in the middle
  axis line style={<->,color=black}, % arrows on the axis
}}

\makeatletter

\newcommand{\opgave}[1]{%
 \def\@exercise{#1}%
 \paragraph{Opgave #1}
}

\makeatother

%Format lim the same way in intext and in display
\let\svlim\lim\def\lim{\svlim\limits}

% horizontal rule
\newcommand\hr{
\noindent\rule[0.5ex]{\linewidth}{0.5pt}
}

\author{Noah Rahbek Bigum Hansen -- 202405538}



\title{Take-Home Assignment weeks 35--36}
\date{}

\begin{document}

\maketitle

\exercise{P1-1}
Determine the magnitude and coordinate direction angles of the resultant force acting at $A$.

\begin{figure} [ht]
  \centering
  \includegraphics[width=0.35\linewidth]{../figures/P1_1}
\end{figure}

We start by adding a coordinate system such that the point $A$ is located at the origin $O$. The resultant force here $\textbf{F}_A$ is simply the sum of the two acting forces $F_B = \qty{900}{N}$ and $F_C = \qty{600}{N}$. We must therefore express these two in terms of Cartesian coordinates. First of all we start by determining the side length of the isoceles right triangle comprised by $\textbf{r}_B$ and its components. These components can be found as:
\[ 
  \cos \ang{45} \cdot \qty{4.5}{m} = \sin \ang{45} \cdot \qty{4.5}{m} = \qty{3,18}{m}
.\]
This means that the position vector $\textbf{r}_B$ can be written as:
\[ 
\textbf{r}_B = \left( \qty{3,18}{m} \textbf{i} - \qty{3,18}{m} \textbf{j} - \qty{6}{m} \textbf{k} \right)
.\]
A unit vector along this direction can be found as:
\[ 
\textbf{u}_B = \frac{\textbf{r}_B}{\left| \textbf{r}_B \right|} = \frac{ \left( \qty{3,18}{m} \textbf{i} - \qty{3,18}{m} \textbf{j} - \qty{6}{m} \textbf{k} \right)}{\sqrt{ \left( \qty{3,18}{m}  \right)^2 + \left( \qty{3,18}{m}  \right)^2 + \left( \qty{6}{m}  \right)^2 }} = \num{0,4241}  \textbf{i} - \num{0,4241}  \textbf{j} - \num{0,8002} \textbf{k}
.\]
Now if we just multiply this by the magnitude of the force $F_B$ we get the force in the direction of $B$ in cartestian coordinates:
\[ 
\textbf{F}_B = F_B \cdot \textbf{u}_B = \qty{900}{N} \cdot \left( \num{0,4241} \textbf{i} - \num{0,4241} \textbf{j} - \num{0,8002} \textbf{k}  \right) = \qty{381,69}{N} \, \textbf{i} - \qty{381,69}{N} \, \textbf{j} - \qty{720,18}{N} \, \textbf{k}
.\]

We can also find the unit vector along $\textbf{r}_C$ as:
\[ 
\textbf{u}_C = \frac{\textbf{r}_C}{\left| \textbf{r}_C \right|} = \frac{\left( - \qty{3}{m} \, \textbf{i} - \qty{6}{m} \, \textbf{j} - \qty{6}{m}\, \textbf{k} \right)}{\sqrt{\left( \qty{3}{m}  \right)^2 + \left( \qty{6}{m}  \right)^2 + \left( \qty{6}{m}  \right)^2}} = - \num{0,333} \textbf{i} - \num{0,666} \textbf{j} - \num{0,666} \textbf{k}
.\]
Therefore we can now find $\textbf{F}_C$ as:
\[ 
\textbf{F}_C = F_C \textbf{u}_C = \qty{600}{N} \cdot \left( -\num{0,333} \textbf{i} - \num{0,666} \textbf{j} - \num{0,666}  \textbf{k} \right) = - \qty{200}{N} \, \textbf{i} - \qty{400}{N} \, \textbf{j} - \qty{400}{N} \, \textbf{k}
.\]
The resultant force $\textbf{F}_A$ can now be found as:
\[ 
  \textbf{F}_A = \textbf{F}_B + \textbf{F}_C = \qty{0,182}{kN} \, \textbf{i} - \qty{0,782}{kN} \, \textbf{j} - \qty{1,12}{kN} \, \textbf{k}
.\]
The magnitude of this can be found by the Pythagorean theorem as:
\[ 
F_A = \sqrt{\left( \qty{0,182}{kN}  \right)^2 + \left( \qty{0,782}{kN}  \right)^2 +( \qty{1,12}{kN} )^2} = \qty{1,38}{kN} 
.\]
Now we can find the angles with each of the three axes as:
\begin{align*}
  \alpha &= \cos^{-1} \left( \frac{F_{Ax}}{F_A} \right) = \cos^{-1} \num{0,132} = \ang{82,4} \\
  \beta &= \cos^{-1} \left( \frac{F_{Ay}}{F_A} \right) = \cos^{-1} \num{-0,567} = \ang{124,5}  \\
  \gamma &= \cos^{-1} \left( \frac{F_{Az}}{F_A} \right) = \cos^{-1} \num{-0,813} = \ang{144,4} 
.\end{align*}
Therefore the resultant force $\textbf{F}_A$ is of magnitude \qty{1,38}{kN} and makes angles of \ang{82,4}, \ang{124,5}, and \ang{144,4} with the positive $x$-, $y$-, and $z$-axes respectively. 


\exercise{P2-1}
Determine the moment of the force of $F = \qty{600}{N}$ about point $A$.

\begin{figure} [ht]
  \centering
  \includegraphics[width=0.35\linewidth]{../figures/P2_1.png}
\end{figure}
\bigbreak
The vector formulation of the moment is
\[ 
  \textbf{M}_A = \textbf{r} \times \textbf{F}
.\]
Where $\textbf{r}$ is the position vector between the point $A$ and any point on the line of action of $\textbf{F}$, e.g. $B$. Therefore
\[ 
  \textbf{r} = \textbf{r}_B - \textbf{R}_A = \left( 4 \cdot \sin \ang{45}, 0, 4 \cdot \cos \ang{45} \right) - \left( 0, 0, 4 \right) = \left( 2 \sqrt{2}, 0, 2 \sqrt{2} -4 \right) \unit{m}
.\]
The unit vector in the direction of $\textbf{F}$ is:
\[ 
  \textbf{u}_{F} = \frac{\textbf{C} - \textbf{B}}{C - B} = \frac{\left( 6 - 2 \sqrt{2}, 6, -2 \sqrt{2} \right)}{\sqrt{\left( 6 - 2 \sqrt{2} \right)^2 + 6^2 + \left( 2 \sqrt{2} \right)^2}} = \left( \num{0,4314}, \num{0,8161}, \num{-0,3847} \right)
.\]
And the force vector can thus be found as:
\[ 
  \textbf{F} = F \cdot \textbf{u}_F = \qty{600}{N} \cdot \left( \num{0,4314}, \num{0,8161}, \num{-0,3847}  \right) = \left( \num{258,82}, \num{489,63}, - \num{230,81} \right) \unit{N}
.\]
And the moment about $A$ is thus
\[ 
  \textbf{M}_A = \textbf{r} \times \textbf{F} = \left( 2 \sqrt{2}, 0, 2 \sqrt{2} - 4 \right) \unit{m} \cdot \left( \num{258,82}, \num{489,63}, - \num{230,81}   \right) \unit{N} = \left( \num{573,64}, \num{349,62}, \num{1384,89}  \right) \unit{N.m} 
.\]


\exercise{P2-2}
The friction at sleeve $A$ can provide a maximum resisting moment of \qty{125}{N.m} about the $x$-axis. Determine the largest magnitude of force $\textbf{F}$ that can be applied to the bracked so that the bracket will not turn.

\begin{figure} [ht]
  \centering
  \includegraphics[width=0.35\linewidth]{../figures/P2_2.png}
\end{figure}
\bigbreak
Here, the same kind of reasoning is applied as above. We first find a position vector from the point of rotation $A$ to a point on the line of attack of the force $\textbf{F}$ at $B$. This is done as:
\[ 
\textbf{r} = \textbf{r}_B - \textbf{r}_{A} = \left( - 150, 300, 100 \right) \unit{mm} - \textbf{0} =  \left( - 150, 300, 100 \right) \unit{mm} 
.\]
Now we must find a unit vector in the direction of $\textbf{F}$, $\textbf{u}_F$. The drawing is a bit hard to decipher but it is understood as the force $\textbf{F}$ lying \ang{60} from the $-x$-axis, \ang{60} from the $+y$-axis, and \ang{45} from the $+z$-axis. Using the direction cosines we therefore get:
\begin{align*}
  \cos \alpha &= - \cos \ang{60} = - \num{0,5} \\
  \cos \beta &= \cos \ang{60} = \num{0,5}  \\
  \cos \gamma &= \cos \ang{45} = \num{0,7071} 
.\end{align*}
The unit vector of $\textbf{F}$ is therefore:
\[ 
\textbf{u}_F = \left( -\num{0,5} ; \num{0,5} ; \num{0,7071}  \right)
.\]

We have the following condition for the largest force the resisting friction can withstand:
\[ 
\textbf{M}_{A} = \textbf{r} \times \textbf{F} = \textbf{r} \times \left( \textbf{u}_F \cdot F \right)
.\]
We can solve this for the magnitude $F$ of the force $\textbf{F}$ as:
\begin{align*}
  \textbf{M}_A &= \textbf{r} \times \left( \textbf{u}_F \cdot F \right) \\
  \textbf{M}_A &= F \left( \textbf{r} \times \textbf{u}_F \right) \\
  F &= \frac{\textbf{M}_A}{\textbf{r} \times \textbf{u}_F}
.\end{align*}
And plugging in known values we get:
\[ 
F = \frac{\left( 125, 0, 0 \right) \unit{N.m}}{\left( - \num{0,15} ; \num{0,3} ; \num{0,1}  \right)\unit{m} \times \left( -\num{0,5} ; \num{0,5} ; \num{0,7071}  \right)} = \qty{771}{N}
.\]
Therefore the force $\textbf{F}$ should have a magnitude of at least $F = \qty{771}{N}$ before the friction force would not be able to hold the sleeve still. 

\end{document}
