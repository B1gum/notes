\section{Kendte fordelinger}


\subsection{Diskrete stokastiske variable}
\begin{table}[ht]
\begin{tabular}{|l|l|}
\hline
\textbf{Fordeling} &
  \textbf{Eksperiment, der beskriver fordelingen} \\ \hline
\textit{Bernoullifordelingen} &
  \begin{tabular}[c]{@{}l@{}}Et tilfældigt eksperiment med to outcomes.\\ Success med sandsynlighed $p$ og fiasko\\ med sandsynlighed $1-p$.\end{tabular} \\ \hline
\textit{Binomialfordelingen} &
  \begin{tabular}[c]{@{}l@{}}Beskriver antallet af successer ved $n$\\ uafhængige Bernoulli-forsøg, hvor hvert\\ har samme sandsynlighed $p$ for success.\end{tabular} \\ \hline
\textit{Poissonfordelingen} &
  \begin{tabular}[c]{@{}l@{}}En fordeling, der modellerer antallet af \\ gange en hændelse sker på et givent tids-\\ interval. Specielt bruges det for diskrete\\ uafhængige begivenheder, der sker med\\ konstant rate. Eksempelvis radioaktivt\\ henfald.\end{tabular} \\ \hline
\textit{Den geometriske fordeling} &
  \begin{tabular}[c]{@{}l@{}}En fordeling, der modellerer antallet af\\ forsøg, der behøves for at få den første\\ success i en række uafhængige Bernoulli-\\ forsøg, med success-sandsynlighed $p$.\end{tabular} \\ \hline
\textit{Den negative binomialfordeling} &
  \begin{tabular}[c]{@{}l@{}}En fordeling, der modellerer antallet af\\ forsøg, der behøves for at få $r$ successer\\ i en række uafhængige Bernoulli-forsøg\\ med success-sandsynlighed $p$.\end{tabular} \\ \hline
\end{tabular}
\end{table}


\subsubsection{Bernoullifordelingen}
\underline{\hyperref[afs:forber]{Bernoullifordelingen}}. En stokastisk variabel $X$ der kun antager værdierne 0 og 1 siges at være \textit{Bernoullifordelt} med parameter $p$, hvor $p = P(X = 1)$. Denne har sandsynlighedsfunktion givet ved:
\[ 
P(X = x) = \begin{cases}
p & \text{for } x = 1, \\
1 - p & \text{for } x = 0
.\end{cases}
\]
og fordelignsfunktion givet ved
\[ 
F_X(x) = \begin{cases}
0 & \text{for } x < 0, \\
1 - p & \text{for } 0 \leq x \leq 1, \\
1 & \text{for } x \geq 1
.\end{cases}
\]
For bernoullifordelingen gælder
\begin{align*}
  E[X] &= p \\
  \mathrm{Var}(X) &= p - p^2
.\end{align*}


\subsubsection{Binomialfordelingen}
\underline{\hyperref[afs:forbin]{Binomialfordelingen}}. Lad $0 \leq p \leq 1$ og $n = 1, 2, 3, \ldots$. En stokastisk variabel $X$ siges da at være \textit{binomialfordelt} med parametre $(n,p)$, hvis $X$ har en sandsynlighedsfunktion givet som
\[ 
p(i) = \binom{n}{i}p^{i}(1-p)^{n-i}, \qquad \text{for} \qquad i = 0, 1, 2, \ldots , n.
.\]
$p$ er sandsynligheden for en success så vi har altså fået $i$ successer $n-i$ fiaskoer og $\binom{n}{i}$ er antallet af permutationer af de $i$ successer. 
Fordelingsfunktionen for binomialfordelingen er givet ved
\[ 
F_X(k) = P(X \leq k) = \sum_{i = 0}^{\left\lfloor k \right\rfloor} \binom{n}{i} p^{i}(1-p)^{n-i}
\]
For $k \in \mathbb{R}$.
For binomialfordelingen gælder
\begin{align*}
  E[X] &= np \\
  \mathrm{Var}(X) &= np(1-p)
.\end{align*}



\subsubsection{Poissonfordelingen}
\underline{\hyperref[afs:forpoi]{Poissonfordelingen}}. Lad $\lambda >0$. Så siges en stokastisk variabel $X$ at være \textit{Poissonfordelt} med parameter $\lambda$ hvis sandsynlighedsfunktion $p$ er givet ved
\[ 
p(i) = \frac{e^{-\lambda i} \lambda^{i}}{i!} \qquad \text{for} \qquad i = 0, 1, 2,\ldots 
\]
og fordelingsfunktion givet ved
\[ 
F_X(k) = \sum_{i = 0}^{\left\lfloor k \right\rfloor } \frac{\lambda^{i}e^{-\lambda}}{i!}
.\]
For Poissonfordelingen gælder
\begin{align*}
  E[X] &= \lambda \\
  \mathrm{Var}(X) &= \lambda
.\end{align*}


\subsubsection{Den geometriske fordeling}
\underline{\hyperref[afs:forgeo]{Den geometriske fordeling}}. Lad $0 < p < 1$ så siges en stokastisk variabel $X$ at være geometrisk fordelt med sandsynlighedsparameter $p$ hvis sandsynlighedsfunktion er givet ved
\[ 
p(n) = p(1-p)^{n-1}, \qquad n = 1,2,3,\ldots 
\]
og fordelignsfunktion givet ved
\[ 
F_X(k) = P(X \leq k) 1 - (1-p)^{k}, \qquad k = 1, 2, 3, \ldots 
\]
for $k < 1, F_X(K) = 0$.
For den geometriske fordeling gælder
\begin{align*}
  E[X] &= \frac{1}{p} \\
  \mathrm{Var}(X) &= \frac{1-p}{p^2}
.\end{align*}



\subsubsection{Den negative binomialfordeling}
\underline{\hyperref[afs:fornegbin]{Den negative binomialfordeling}} er beskrevet med sandsynlighedsparameter $p$ og antalsparameter $r$, $(p,r)$. Sandsynlighedsfunktionen er givet som
\[ 
P(X) = \binom{r-1}{n-1}p^{r}(1-p)^{n-r}, \qquad n = r, r+1, \ldots 
\]
og fordelingsfunktion givet ved
\[ 
F_X(k) = P(X \leq k) = \sum_{i = 0}^{\left\lfloor k \right\rfloor} \binom{i + r -1}{r - 1}p^{r}(1-p)^{i}, \qquad k = 0, 1, 2, \ldots 
.\]
For den negative binomialfordeling gælder
\begin{align*}
  E[X] &= \frac{r}{p} \\
  \mathrm{Var}(X) = r \cdot \frac{1-p}{p^2}
.\end{align*}


\subsection{Kontinuerte stokastiske variable}
\begin{table}[ht]
\begin{tabular}{|l|l|}
\hline
\textbf{Fordeling}               & \textbf{Eksperiment, der beskriver fordelingen}                                                                                \\ \hline
\textit{Den uniforme fordeling} &
  \begin{tabular}[c]{@{}l@{}}Den uniforme fordeling beskriver en stokastisk\\ variabel der har lige stor sandsynlighed for at\\ være enhver given værdi i et interval.\end{tabular} \\ \hline
\textit{Normalfordelingen}       & \begin{tabular}[c]{@{}l@{}}Modellerer mange naturlige fænomener såsom\\ højder, vægte og målefejl. Klokke-kurven.\end{tabular} \\ \hline
\textit{Eksponentialfordelingen} & \begin{tabular}[c]{@{}l@{}}Eksponentialfordelingen beskriver ventetiden\\ mellem hændelser i en Poisson-process.\end{tabular}  \\ \hline
\textit{Gammafordelingen}        & \begin{tabular}[c]{@{}l@{}}Gammafordelingen modellerer tiden indtil $k$\\ hændelser sker i en Poisson-process.\end{tabular}    \\ \hline
\end{tabular}
\end{table}


\subsubsection{Den uniforme fordeling}
\underline{\hyperref[afs:foruni]{Den uniforme fordeling}}. Lad $\alpha < \beta$. En stokastisk variabel $X$ siges at være uniformfordelt på intervallet $(\alpha, \beta)$, hvis $X$ er kontinuert med tæthedsfunktion
\[ 
f(x) = \frac{1}{\beta - \alpha}, \quad \text{for } \alpha < x < \beta \quad \text{og} \quad f(x) = 0 \text{ ellers}
.\]
Denne har fordelignsfunktion givet som
\[ 
F(x) = \frac{x - \alpha}{\beta - \alpha}, \quad \text{for} \quad \alpha \leq x \leq \beta
.\]
For denne gælder, at
\begin{align*}
  E[X] &= \frac{\alpha + \beta}{2} \\
  \mathrm{Var}(X) &= \frac{\beta - \alpha}{12}
.\end{align*}


\subsubsection{Normalfordelingen}
\underline{\hyperref[afs:fornor]{Normalfordelingen}}. Lad $\mu \in \mathbb{R}$ og $\sigma^2 \geq 0$. En stokastisk variabel $X$ siges at være normalfordelt med parametre $\mu$ og $\sigma^2$, hvis $X$ har tæthedsfunktion givet ved
\[ 
f(x) = \frac{1}{\sqrt{2\pi}\sigma}e^{\frac{(x-\mu)^2}{2\sigma^2}}, \text{ for alle } x
.\]
Hvis $\mu = 0$ og $\sigma^2 = 1$ siges $X$ endvidere at være standard-normalfordelt. Normalfordelingen har fordelingnsfunktion (se evt. \autoref{fig:F13_1}) givet ved
\[ 
\Phi(x) = \frac{1}{\sqrt{2\pi}} \int_{-\infty}^{x} e^{\frac{s^2}{2}} \, \mathrm{d}s
.\]
For normalfordelingen gælder
\begin{align*}
  E[X] &= \mu \\
  \mathrm{Var}(X) &= \sigma^2
.\end{align*}


\subsubsection{Eksponentialfordelingen}
\underline{\hyperref[afs:foreks]{Eksponentialfordelingen}}. Lad $\lambda > 0$. En stokastisk variabel $X$ siges at være eksponentialfordelt med parameter $\lambda$, hvis $X$ har tæthedsfunktion givet ved
\[ 
  f(x) = \lambda e^{-\lambda x}, \qquad \text{for } x > 0 \text{ og } f(x) = 0, \text{ ellers}
.\]
Denne har fordelingsfunktion givet ved:
\[ 
F(x) = 1 - e^{-\lambda x}, \text{ for } x > 0
.\]
For eksponentialfordelingen gælder at
\begin{align*}
  E[X] &= \frac{1}{\lambda} \\
  \mathrm{Var}(X) &= \frac{1}{\lambda^2}
.\end{align*}


\subsubsection{Gammafordelingen}
\underline{\hyperref[afs:forgam]{Gammafordelingen}}. Lad $\alpha, \lambda > 0$. En stokastisk variabel $X$ siges at være gammafordelt med parametre $(\alpha, \lambda)$, hvis $X$ har tæthedsfunktion $f$ givet ved
\[ 
f(x) = \frac{\lambda e^{- \lambda x} (\lambda x)^{\alpha - 1}}{\Gamma(\alpha)}, \text{ for } x > 0
.\]
Denne har fordelingsfunktion givet ved
\[
  F_X(x) = P(X \leq x) = \frac{1}{\Gamma(k)} \int_{0}^{x} \lambda^{k} t^{k-1} e^{-\lambda t} \, \mathrm{d}t
.\]
For gammafordelingen gælder
\begin{align*}
  E[X] &= \frac{\alpha}{\lambda} \\
  \mathrm{Var}(X) &= \frac{\alpha}{\lambda^2}
.\end{align*}
