\documentclass[12pt]{article}
\usepackage[danish]{babel}
\usepackage{amsfonts, amssymb, mathtools, amsthm, amsmath}
\usepackage{graphicx, pgfplots}
\usepackage{url}
\usepackage[dvipsnames]{xcolor}
\usepackage{sagetex}
\usepackage{lastpage}

%loaded last
\usepackage[hidelinks]{hyperref}

\usepackage{siunitx}
  \sisetup{exponent-product = \cdot,
    output-decimal-marker = {,}}

%Giles Castelles incfig
\usepackage{import}
\usepackage{xifthen}
\usepackage{pdfpages}
\usepackage{transparent}

\newcommand{\incfig}[2][1]{%
  \def\svgwidth{#1\columnwidth}
  \import{../figures/}{#2.pdf_tex}
}

\setlength{\parindent}{0in}
\setlength{\oddsidemargin}{0in}
\setlength{\textwidth}{6.5in}
\setlength{\textheight}{8.8in}
\setlength{\topmargin}{0in}
\setlength{\headheight}{18pt}

\usepackage{fancyhdr}
\pagestyle{fancy}

\fancyhead{}
\fancyfoot{}
\fancyfoot[R]{\thepage}
\fancyhead[C]{\leftmark}

\pgfplotsset{compat=newest}

\pgfplotsset{every axis/.append style={
  axis x line=middle,    % put the x axis in the middle
  axis y line=middle,    % put the y axis in the middle
  axis line style={<->,color=black}, % arrows on the axis
}}

\usepackage{thmtools}
\usepackage{tcolorbox}
  \tcbuselibrary{skins, breakable}
  \tcbset{
    space to upper=1em,
    space to lower=1em,
  }

\theoremstyle{definition}

\newtcolorbox[auto counter]{definition}[1][]{%
  breakable,
  colframe=ForestGreen,  %frame color
  colback=ForestGreen!5, %background color
  colbacktitle=ForestGreen!25, %background color for title
  coltitle=ForestGreen!70!black,  %title color
  fonttitle=\bfseries\sffamily, %title font
  left=1em,              %space on left side in box,
  enhanced,              %more options
  frame hidden,          %hide frame
  borderline west={2pt}{0pt}{ForestGreen},  %display left line
  title=Definition \thetcbcounter: #1,
}

\newtcolorbox{greenline}{%
  breakable,
  colframe=ForestGreen,  %frame color
  colback=white,          %remove background color
  left=1em,              %space on left side in box
  enhanced,              %more options
  frame hidden,          %hide frame
  borderline west={2pt}{0pt}{ForestGreen},  %display left line
}

\newtcolorbox[auto counter, number within=section]{eks}[1][]{%
  brekable,
  colframe=NavyBlue,  %frame color
  colback=NavyBlue!5, %background color
  colbacktitle=NavyBlue!25,    %background color for title
  coltitle=NavyBlue!70!black,  %title color
  fonttitle=\bfseries\sffamily, %title font
  left=1em,            %space on left side in box,
  enhanced,            %more options
  frame hidden,        %hide frame
  borderline west={2pt}{0pt}{NavyBlue},  %display left line
  title=Eksempel \thetcbcounter: #1
}

\newtcolorbox{blueline}{%
  breakable,
  colframe=NavyBlue,     %frame color
  colback=white,         %remove background
  left=1em,              %space on left side in box,
  enhanced,              %more options
  frame hidden,          %hide frame
  borderline west={2pt}{0pt}{NavyBlue},  %display left line
}

\newtcolorbox{teo}[1][]{%
  breakable,
  colframe=RawSienna,  %frame color
  colback=RawSienna!5, %background color
  colbacktitle=RawSienna!25,    %background color for title
  coltitle=RawSienna!70!black,  %title color
  fonttitle=\bfseries\sffamily, %title font
  left=1em,              %space on left side in box,
  enhanced,              %more options
  frame hidden,          %hide frame
  borderline west={2pt}{0pt}{RawSienna},  %display left line
  title=Teori: #1,
}

\newtcolorbox[auto counter, number within=section]{sæt}[1][]{%
  breakable,
  colframe=RawSienna,  %frame color
  colback=RawSienna!5, %background color
  colbacktitle=RawSienna!25,    %background color for title
  coltitle=RawSienna!70!black,  %title color
  fonttitle=\bfseries\sffamily, %title font
  left=1em,              %space on left side in box,
  enhanced,              %more options
  frame hidden,          %hide frame
  borderline west={2pt}{0pt}{RawSienna},  %display left line
  title=Sætning \thetcbcounter: #1,
  before lower={\textbf{Bevis:}\par\vspace{0.5em}},
  colbacklower=RawSienna!25,
}

\newtcolorbox{redline}{%
  breakable,
  colframe=RawSienna,  %frame color
  colback=white,       %Remove background color
  left=1em,            %space on left side in box,
  enhanced,            %more options
  frame hidden,        %hide frame
  borderline west={2pt}{0pt}{RawSienna},  %display left line
}

\newtcolorbox{for}[1][]{%
  breakable,
  colframe=NavyBlue,  %frame color
  colback=NavyBlue!5, %background color
  colbacktitle=NavyBlue!25,    %background color for title
  coltitle=NavyBlue!70!black,  %title color
  fonttitle=\bfseries\sffamily, %title font
  left=1em,              %space on left side in box,
  enhanced,              %more options
  frame hidden,          %hide frame
  borderline west={2pt}{0pt}{NavyBlue},  %display left line
  title=Forklaring #1,
}

\newtcolorbox{bem}{%
  breakable,
  colframe=NavyBlue,  %frame color
  colback=NavyBlue!5, %background color
  colbacktitle=NavyBlue!25,    %background color for title
  coltitle=NavyBlue!70!black,  %title color
  fonttitle=\bfseries\sffamily, %title font
  left=1em,              %space on left side in box,
  enhanced,              %more options
  frame hidden,          %hide frame
  borderline west={2pt}{0pt}{NavyBlue},  %display left line
  title=Bemærkning:,
}

\makeatother
\def\@lecture{}%
\newcommand{\lecture}[3]{
  \ifthenelse{\isempty{#3}}{%
    \def\@lecture{Lecture #1}%
  }{%
    \def\@lecture{Lecture #1: #3}%
  }%
  \subsection*{\makebox[\textwidth][l]{\@lecture \hfill \normalfont\small\textsf{#2}}}
}

\makeatletter

\newcommand{\opgave}[1]{%
 \def\@opgave{#1}%
 \subsection*{Opgave #1}
}

\makeatother

%Format lim the same way in intext and in display
\let\svlim\lim\def\lim{\svlim\limits}

% horizontal rule
\newcommand\hr{
\noindent\rule[0.5ex]{\linewidth}{0.5pt}
}

\title{TØ-opgaver uge 13}
\author{Noah Rahbek Bigum Hansen}
\date{25. November 2024}

\begin{document}

\maketitle

\section*{Prøveeksamensopgave 23}
Lad $X$ betegne en stokastisk variabel der er Poissonfordelt med parameter $\lambda = 7$. Angiv
\[ 
  \mathrm{Var}(4+2X) = k
.\]
Hvor $k$ er et helt tal mellem 0 og 99.
\bigbreak
Vi kan benytte at $\mathrm{Var}(aX + b) = a^2\mathrm{Var}(X)$ så vi får
\[ 
  \mathrm{Var}(4 + 2X) = 4 \mathrm{Var}(X)
.\]
For en Poissonfordeling er variansen givet som
\[ 
  \mathrm{Var}(X) = \lambda
.\]
Vi får dermed
\[ 
  \mathrm{Var}(4 + 2X) = 4 \lambda = 4 \cdot 7 = 28
.\]
Altså er $k = 28$.

\section*{Opg. A}
Et olieselskab gennemfører en geologisk undersøgelse, der viser, at olieboringer har 20\% chance for at finde olie.

\subsection*{1.}
Hvad er sandsynligheden for at selskabet har fundet olie ved præcis 3 af deres 7 boringer?
\bigbreak
I dette tilfælde må den stokastiske variabel være bindomialfordelt, da denne angiver antallet af successer ud af $n$ forsøg. Her gælder generelt følgende sandsynlighedsfunktion
\[ 
P(X = i) = \binom{n}{i} p^{i}(1-p)^{n-i}
.\]
Hvis vi sætter tallene fra opgaven ind fås
\[ 
P(X = 3) = \binom{7}{3} (\num{0,2})^{3}(1-\num{0,2})^{7-3} = \num{0,1147}  
.\]


\subsection*{2.}
Hvor mange boringer skal man i middel foretage for at have fundet olie 3 gange?
\bigbreak
Her ønsker vi at finde ud af hvor mange forsøg det tager at få et antal successer. Derfor må den ovenstående stokastiske variabel være negativt binomialfordelt, med sandsynlighedsparameter $p = \num{0,2}$ og antalsparameter $r = 3$. Vi har altså
\[ 
  E[X] = \frac{r}{p} = \frac{3}{\num{0,2}} = 15
.\]


\section*{Opg. B}
Lad $X$ være en Poissonfordelt stokastisk variabel med parameter $\lambda$. Udregn middelværdien af $X(7X-2)$.
\bigbreak
Vi ønsker at finde
\[ 
  E[X(7X - 2)] = E[7X^2 - 2X] = 7E[X^2] - 2E[X]
.\]
2.-momentet kan findes fra formlen for varians som
\begin{align*}
  \mathrm{Var}(X) &= E[X^2] - (E[X])^2 \\
  \implies E[X^2] = \mathrm{Var}(X)+ (E[X])^2
.\end{align*}
For en Poisson-fordeling har vi $E[X] = \lambda$ og $\mathrm{Var}(X) = \lambda$. Vi får da
\[ 
  E[X^2] = \lambda + \lambda^2
.\]
Dette sættes ind i udtrykket fra starten så vi får
\begin{align*}
  E[X(7x-2)] &= 7 \left( \lambda + \lambda^2 \right) - 2\lambda \\
             &= 7\lambda^2 + 5\lambda
.\end{align*}


\section*{Opg. C}
Lad $X$ være binomialfordelt ($5, \frac{3}{4}$) og $Y$ være binomialfordelt ($7, \frac{1}{2}$). Udregn middelværdien for $X-Y$ og bestem om den er positiv eller negativ.
\bigbreak
Idet middelværdien af en sum er lig summen af middelværdierne kan udtrykket fra opgaven skrives som
\[ 
  E[X-Y] = E[X] - E[Y]
.\]
Vi finder først $E[X]$ som
\[ 
  E[X] = np = 5\cdot \frac{3}{4} = \frac{15}{4}
.\]
Og dernæst kan vi finde $E[Y]$ som
\[ 
  E[Y] = np = 7\cdot \frac{1}{2} = \frac{7}{2}
.\]
Vi har da
\[ 
  E[X-Y] = \frac{15}{4} - \frac{7}{2} = \frac{1}{4}
.\]
Denne er positiv.


\section*{Opg. D}
Lad $X$ være binomialfordelt ($3, \frac{1}{2}$) og sæt $Z = (X - 1)^2$.

\subsection*{1.}
Udregn sandsynlighedsfunktionen for $Z$.

\subsection*{2.}
Udregn fordelingsfunktionen for $Z$.

\section*{Opg. 2.1}

\section*{Prøveeksamensopgave 24}
Lad $X$ være antallet af kast, der kræves for at få krone, når man kaster en fair mønt indtil den viser krone. Find middelværdien af $X$:
\[ 
  E[X] = k
.\]
Hvor $k$ er et helt tal mellem 0 og 99.

\section*{Opg. G}
Lad $X$ være en kontinuert stokastisk variabel med tæthedsfunktion $f$ givet ved
\begin{equation*}
  f(x) = \left\{
    \begin{aligned}
      & a + bx^2 &\qquad &\text{hvis } 0 < x < 1 \\
      & 0 &\qquad &\text{ellers}
    \end{aligned}
  \right.
\end{equation*}
hvor $a$ og $b$ er konstanter. Det oplyses at $E[X] = \frac{3}{5}$. Bestem konstanterne $a$ og $b$


\section*{Opg. E}
Lad $X$ betegne en kontinuert stokastisk variabel med tæthedsfunktion
\begin{equation*}
  f(x) = \left\{
    \begin{aligned}
      & 3x^2 &\qquad &\text{hvis } 0 < x < 1 \\
      & 0 &\qquad &\text{ellers.}
    \end{aligned}
  \right.
\end{equation*}
Udregn $E \left[ X^3 \right]$

\section*{Opg. F}
Lad $X$ betegne en kontinuert stokastisk variabel. Find det til $\alpha$ som minimerer udtrykket
\[ 
E \left[ (X - \alpha)^2 \right]
.\]
Fortolk på dit resultat.


\end{document}
