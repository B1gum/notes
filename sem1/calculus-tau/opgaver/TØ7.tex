\documentclass[12pt]{article}

\usepackage[utf8]{inputenc}
\usepackage[danish]{babel}
\usepackage{latexsym, amsfonts, amssymb, amsthm, amsmath}
\usepackage{siunitx}
\usepackage{graphicx, pgfplots}
\usepackage{fancyhdr, lastpage}

%loaded last
\usepackage[hidelinks]{hyperref}

\sisetup{exponent-product = \cdot,
  output-decimal-marker = {,}}

%Giles Castelles incfig
\usepackage{import}
\usepackage{xifthen}
\usepackage{pdfpages}
\usepackage{transparent}

\newcommand{\incfig}[2][1]{%
  \def\svgwidth{#1\columnwidth}
  \import{../figures/}{#2.pdf_tex}
}

\pdfsuppresswarningpagegroup=1

\setlength{\parindent}{0in}
\setlength{\oddsidemargin}{0in}
\setlength{\textwidth}{6.5in}
\setlength{\textheight}{8.8in}
\setlength{\topmargin}{0in}
\setlength{\headheight}{18pt}

\pgfplotsset{compat=newest}

\pgfplotsset{every axis/.append style={
  axis x line=middle,    % put the x axis in the middle
  axis y line=middle,    % put the y axis in the middle
  axis line style={<->,color=black}, % arrows on the axis
}}

\title{TØ-opgaver uge 7}
\author{Noah Rahbek Bigum Hansen}
\date{13. Oktober 2024}

\begin{document}

\maketitle

\section*{Opg. 10.1.2}
\textbf{Find the general solution for each differential equation. Verify that each solution satisfies the original differential equation.}
\[
\frac{\mathrm{d}y}{\mathrm{d}x} = -4x + 12x^2
.\]
\bigbreak
Differentialligningen separeres så vi får at
\[
\, \mathrm{d}y = -4x + 12x^2 \, \mathrm{d}x \implies \int \, \mathrm{d}y = \int -4x + 12x^2 \, \mathrm{d}x
.\] 
Altså får vi at
\[
y(x) = -2x^2 + 4x^3 + c
.\]
Det indses at ovenstående er korrekt ved indsættelse i det oprindelige udtryk og at differentialligningen derfor må være korrekt løst.

\section*{Opg. 10.1.9}
\textbf{Find the general solution for each differential equation. Verify that each solution satisfies the original differential equation.}
\[ 
\frac{\mathrm{d}y}{\mathrm{d}x} = 8x^2y-9xy
.\]
\bigbreak
Først omskrives som
\[
\frac{\mathrm{d}y}{\mathrm{d}x} = y\left( 8x^2-9x \right) \implies \int \frac{1}{y} \, \mathrm{d}y = \int 8x^2-9x \, \mathrm{d}x
.\]
Dermed får vi at
\[
\ln(y) = \frac{8}{3}x^3 - \frac{9}{2}x^2 + c \implies y = ke^{\frac{8}{3}x^3 - \frac{9}{2}x^2}
.\] 

\section*{Opg. 10.1.21}
\textbf{Find the particular solution for each initial value problem.}
\[
\frac{\mathrm{d}y}{\mathrm{d}x} = \frac{5x^3}{4y}; \qquad y(0) = 2
.\]
\bigbreak
Først separeres differentialligningen så vi får at
\[
\int y \, \mathrm{d}y = \int \frac{5}{4}x^3 \, \mathrm{d}x \implies \frac{1}{2}y^2 = \frac{5}{16}x^4 + c
.\] 
Og dermed har vi at
\[
y^2 = \frac{5}{8}x^4 + 2c
.\]
Hvilket giver os at
\[
y = \sqrt{\frac{5}{8}x^4+2c} 
.\] 


\section*{Opg. 10.2.3}
\textbf{Find the general solution for each differential equation.}
\[
\frac{\mathrm{d}y}{\mathrm{d}x} + 2xy = 16x
.\] 
\bigbreak
Vi omskriver så vi får at
\[
\frac{\mathrm{d}y}{\mathrm{d}x} = -2x(y-8)
.\] 
Dermed har vi at
\[
\int \frac{1}{y-8} \, \mathrm{d}y = \int -2x \, \mathrm{d}x
.\]
Hvilket giver
\[
\ln(y-8) = -x^2 + c \implies y-8 = e^{-x^2+c} \implies y = ke^{-x^2}+8
.\] 



\section*{Opg. 10.2.6}
\textbf{Find the general solution for each differential equation.}
\[
x \frac{\mathrm{d}y}{\mathrm{d}x} + 2xy - x^2 = 0
.\] 
\bigbreak
Ved at addere med $x^2$ og dividere med $x$ kan omskrives så
\[
\frac{\mathrm{d}y}{\mathrm{d}x} + 2y  = x
.\] 
Det indses at dette er en 1.-ordens lineær inhomogen differentialligning. Standardløsningen for denne er
\[
  y = \frac{1}{v(x)}\left( \int v(x)Q(x) \, \mathrm{d}x + C  \right) 
,\]
for
\begin{gather*}
  \frac{\mathrm{d}y}{\mathrm{d}x} + P(x)y = Q(x) \\
v(x) = e^{\int P(x) \, \mathrm{d}x}
\end{gather*}
Vi indsætter
\begin{align*}
  y &= e^{-\int 2 \, \mathrm{d}x}\left( \int e^{\int 2 \, \mathrm{d}x}x \, \mathrm{d}x + C \right) \\
  &= e^{-2x}\left( \int e^{2x}x \, \mathrm{d}x + C \right)  \\
  &= e^{-2x} \left( \frac{1}{2}e^{2x}x - \frac{1}{2} \int e^{2x} \, \mathrm{d}x + C \right)   \\
  &= e^{-2x}\left( \frac{1}{2}e^{2x}x - \frac{1}{4}e^{2x} + C \right)  \\
  &= \frac{x}{2} - \frac{1}{4} + e^{-2x}C  \\
\end{align*}


\section*{Opg. 10.2.20}
\textbf{Solve each differential equation, subject to the given initial conditions.}
\[
\frac{\mathrm{d}y}{\mathrm{d}x} + 3x^2y - 2xe^{-x^3} = 0; \qquad y(0) = \num{1000}
.\]
\bigbreak
Der omskrives til standardform
\[
\frac{\mathrm{d}y}{\mathrm{d}x} + 3x^2y = 2xe^{-x^3}
.\] 
Indsætter i standardløsning
\begin{align*}
  y &= e^{-\int 3x^2 \, \mathrm{d}x}\left( \int e^{\int 3x^2 \, \mathrm{d}x} 2xe^{-x^3} \, \mathrm{d}x + C \right)  \\
  &= e^{-x^3}\left( \int e^{x^3}2xe^{-x^3} + C \right)  \\
  &= e^{-x^3}\left( \int 2x \, \mathrm{d}x + C \right)  \\
  &= e^{-x^3}\left( x^2 + C \right)  \\
.\end{align*}
Vi indsætter begyndelsesbetingelsen
\[
1000 = e^{-0^3}\left( 0^2 + C \right) = 1\left( C \right) = C
.\] 
Altså er den partikulære løsning
\[
y = e^{-x^3}\left( x^2 + 1000 \right) 
.\] 


\section*{Prøveeksamensopgave 11}
Find den fuldstændige løsning til differentialligningen
\[
\frac{\mathrm{d}y}{\mathrm{d}x} + \frac{y}{x} = x; \qquad x>0
.\] 
Det er funktioner på følgende form (hvor $C$ er en konstant)
\[
y(x) = \frac{C}{x} + \frac{x^2}{?}
.\] 
Dit svar skal være et helt tal mellem $\num{0}$ og $\num{99}$.
\bigbreak
Differentialligningen er allerede skrevet på standardform og kan derfor sættes direkte ind i standardløsningen så vi får at
\begin{align*}
  y &= e^{- \int \frac{1}{x} \, \mathrm{d}x} \left( \int e ^{\int \frac{1}{x} \, \mathrm{d}x} x \, \mathrm{d}x + C \right)  \\
  &= \frac{1}{|x|} \left( \int |x| x \, \mathrm{d}x + C \right)  \\
  &= \frac{1}{x} \left( \int x^2 \, \mathrm{d}x + C \right)  \\
  &= \frac{1}{x}\left( \frac{1}{3}x^3 + C \right)  \\
  &= \frac{1}{3}x^2 + \frac{C}{x}\\
.\end{align*} 
Altså er svaret \num{3}.



\section*{Prøveeksamensopgave 12}
Find den fuldstændige løsning til differentialligningen
\[
\frac{\mathrm{d}y}{\mathrm{d}x} = \frac{xy^2}{8}
.\]
Det er funktioner på følgende form (hvor $K$ er en konstant):
 \[
y(x) = \frac{-?}{K+x^2}
,\] 
defineret på et interval så nævneren er forskellig fra nul.
Dit svar skal være et helt tal mellem $0$ og $99$.
\bigbreak
Differentialligningen er seperabel idet følgende omskrivning kan udføres
\[
\frac{1}{y^2} \, \mathrm{d}y = \frac{1}{8}x \, \mathrm{d}x
.\]
Dermed har vi at
\[
\int \frac{1}{y^2} \, \mathrm{d}y = \int \frac{1}{8}x \, \mathrm{d}x \implies - \frac{1}{y} = \frac{1}{16}x^2 + C
.\] 
Dermed fås at
\[
y = -\frac{16}{x^2+K}
.\]
Altså er svaret 16.


\end{document}
