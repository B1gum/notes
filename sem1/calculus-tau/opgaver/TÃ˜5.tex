\documentclass[12pt]{article}

\usepackage[utf8]{inputenc}
\usepackage[danish]{babel}
\usepackage{latexsym, amsfonts, amssymb, amsthm, amsmath, siunitx, graphicx, pgfplots}

\pdfsuppresswarningpagegroup=1

\setlength{\parindent}{0in}
\setlength{\oddsidemargin}{0in}
\setlength{\textwidth}{6.5in}
\setlength{\textheight}{8.8in}
\setlength{\topmargin}{0in}
\setlength{\headheight}{18pt}

\pgfplotsset{compat=newest}

\pgfplotsset{every axis/.append style={
  axis x line=middle,    % put the x axis in the middle
  axis y line=middle,    % put the y axis in the middle
  axis line style={<->,color=black}, % arrows on the axis
}}

\title{TØ-opgaver uge 5}
\author{Noah Rahbek Bigum Hansen}
\date{26. september - 2025}

\begin{document}

\maketitle

\section*{Opg. 1}
Afgør om talfølgen $a_n = \left( \frac{n+1}{2n} \right) \left( 1 - \frac{1}{n} \right) $ konvergerer eller divergerer.  I tilfælde af konvergens, find grænseværdien.
\bigbreak
Idet det gælder at:
\[
\lim_{n \to \infty} a_n b_n = ab
.\] 
For $a_n \to a$ og $b_n \to b$.
Kan vi starte med undersøge den første parentes. Her har vi:
\[
\lim_{n \to \infty} \frac{n+1}{2n} = \frac{1}{2}
.\] 
Og for den anden parentes har vi:
\[
\lim_{n \to \infty} 1 - \frac{1}{n} = 1 - 0 = 1
.\] 
Altså har vi:
\[
\lim_{n \to \infty} a_n b_n = ab = \frac{1}{2}\cdot 1 = \frac{1}{2}
.\] 

\section*{Opg. 2}
Afgør om talfølgen $a_n = n - \frac{1}{n}$ er monoton (voksende eller aftagende), om den er begrænset og om den konvergerer.
\bigbreak
Det kan hurtigt ses at vi ved at indsætte en større $n$-værdi altid får et større tal da første led stiger med mindst 1 pr. $n$-værdi og 2.-led maksimalt trækker 1 fra (dog i praksis noget mindre end 1 for selv relativt lave $n$-værdier). Dette betyder altså at $a_n$ er strengt voksende den er dog ikke begrænset med samme argument som ovenfor og den konvergerer derfor ikke.

\section*{Opg. 3}
Afgør om talfølgen $a_n = \frac{2^n-1}{2^n}$ er monoton (voksende eller aftagende), om den er begrænset og om den konvergerer.
\bigbreak
Denne splittes op:
\[
a_n = \frac{2^n}{2^n} - \frac{1}{2^n}
.\] 
Og her har vi:
\begin{align*}
  \lim_{n \to \infty} \frac{2^n}{2^n} = 1 \\
\end{align*}
  Og
\begin{align*}
  \lim_{n \to \infty} \frac{1}{2^n} = 0
.\end{align*}
Og altså:
\[
a = 1-0 = 1
.\] 
Dette er både den laveste øvre grænse og den værdi som den strengt voksende talfølge konvergerer mod.


\section*{Opg. 4}
Bestem værdien af rækken $\sum_{n=0}^{\infty} \frac{\cos\left( \pi n \right)}{5^n}$. \textit{Hint: overvej først hvilke værdier} $\cos(\pi n)$ \textit{tager.}
\bigbreak
De første værdier af rækken opskrives:
\begin{align*}
  s_0 &= \frac{\cos\left( 0 \right) }{5^0} = \frac{1}{1} \\
  s_1 &= \frac{\cos \left( \pi \right)}{5} = -\frac{1}{5}\\
  s_2 &= \frac{\cos\left( 2\pi \right) }{5^2} = \frac{1}{25} \\
  s_3 &= \frac{\cos\left( 3\pi \right) }{5^3} = -\frac{1}{125} \\
  s_4 &= \frac{\cos\left( 4\pi \right) }{5^4} = \frac{1}{625}
.\end{align*} 
Det indses hurtigt at rækken er absolut konvergent da den i så fald ville aftage hurtigere end eksempelvis $\frac{1}{n^2}$ som er kendt som konvergent. Rækken kan omskrives til:
\[
  \left( \frac{1}{5} \right)^n 
.\] 
Hvilket hurtigt kan ses værende en geometrisk række med $z = \frac{1}{5}$. Altså har vi løsningssummen som:
\[
s = \frac{1}{1+\frac{1}{5}} = \frac{5}{6}
.\] 
Altså konvergerer summen til værdien $\frac{5}{6}$.


\section*{Opg. 5}
Brug sammenligningskriteriet til at afgøre om rækken $\sum_{n=2}^{\infty} \frac{1}{\sqrt{n}-1}$ konvergerer eller divergerer. 
\bigbreak
For hver $n$-værdi vil rækken i opgaven antage en større værdi end rækken $\frac{1}{\sqrt{n} }$. Da $\frac{1}{\sqrt{n}} = \frac{1}{n^{\frac{1}{2}}}$ og eksponenten til $n$ derfor er mindre end 1 vil  $\frac{1}{\sqrt{n}}$ divergere og da rækken i opgaven er større til alle $n$-værdier vil denne også divergere.

\section*{Opg. 6}
Brug kvotientkriteriet til at afgøre om rækken $\sum_{n=1}^{\infty} \frac{2^n}{n!}$ konvergerer eller divergerer.
\bigbreak
Kvotientkriteriet siger at hvis der gælder:
\[
\lim_{n \to \infty} |\frac{a_{n+1}}{a_n}| = R
.\] 
Udtrykket indsættes:
\[
R = \lim_{n \to \infty} \frac{2^{n+1}\cdot n!}{\left( n+1 \right)! \cdot 2^n} = \lim_{n \to \infty} \frac{2}{n+1} = 0
\] 
Altså konvergerer udtrykket

\section*{Opg. 7}
Brug kvotientkriteriet til at afgøre om rækken $\sum_{n=1}^{\infty} \frac{2^{n+1}}{3^{n-1}n}$ konvergerer eller divergerer.
\bigbreak
Samme fremgangsmåde som ovenfor benyttes:
\[
R = \lim_{n \to \infty} \frac{2^{n+2}\cdot 3^{n-1}n}{3^n n \cdot 2^{n+1}} = \frac{2}{3} 
.\] 
Da $\frac{2}{3}<1$ har vi altså at serien konvergerer.

\section*{Prøveeksamenssættet opg. 7}
Bestem summen af følgende uendelige række:
\[
\sum_{n=0}^{\infty} \frac{5^n}{6^n}
.\] 
Dit svar skal være et helt tal mellem 0 og 99.
\bigbreak
Det indses hurtigt at summen kan skrives som:
\[
\sum_{n=0}^{\infty} \frac{5^n}{6^n} = \sum_{n=0}^{\infty} \left( \frac{5}{6} \right)^n
.\] 
Hvilket er en geometrisk række med $a=0$ og  $z=\frac{5}{6}$. Altså konvergerer denne til:
\[
\frac{1}{1-\frac{5}{6}} = 6
.\] 

\section*{Prøveeksamenssættet opg. 8}
Bestem grænseværdien:
\[
  \lim_{n \to \infty} \left[ -76e^{\frac{5}{n}} \sin\left( \frac{2n\pi}{n+2} \right) + 47\left( e^{\frac{2n}{n^2+1}}-\cos\left( \pi-\frac{4}{n} \right)  \right)  \right]  
.\] 
Dit svar skal være et helt tal mellem 0 og 99.
\bigbreak
Først iagttages 1.-leddet:
\[
\lim_{n \to \infty} -76e^{\frac{5}{n}}\sin\left( \frac{2n\pi}{n+2} \right) 
.\]
Her betragtes først den første del af produktet:
\[
  \lim_{n \to \infty} -76e^{\frac{5}{n}} = -76e^{\frac{5}{\infty}} = -76
.\] 
Den anden del af produktet kan skrives som:
\[
\sin\left( 2\pi \lim_{n \to \infty} \frac{n}{n+2} \right) = \sin(2\pi) = 0
.\] 
Altså bliver første led:
\[
\lim_{n \to \infty} -76e^{\frac{5}{n}}\sin\left( \frac{2n\pi}{n+2} \right) = -76\cdot 0 = 0 
.\]
Vi kan dernæst kigge på 1.-leddet i parentesen, herefter kaldt 2.-leddet. For $n \to \infty$ har vi:
\[
2\cdot \lim_{n \to \infty} \frac{n}{n^2+1} = 2\cdot 0 = 0
.\] 
Altså har vi $e^0 = 1$. For 2.-leddet inde i parentesen, herefter kaldt 3.-leddet har vi for  $n \to \infty$:
\[
\lim_{n \to \infty} \cos \left( \pi-\frac{4}{n} \right) = \cos(\pi) = -1
.\] 
Altså har vi for parentesen:
\[
47\left( e^{\frac{2n}{n^2+1}}-\cos\left( \pi-\frac{4}{n} \right)  \right) = 47\left( 1 - (-1= \right)  = 94
.\] 
Altså konvergerer summen til 94.


\end{document}
