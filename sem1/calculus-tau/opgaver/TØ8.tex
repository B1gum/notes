\documentclass[12pt]{article}
\usepackage[danish]{babel}
\usepackage{amsfonts, amssymb, mathtools, amsthm, amsmath}
\usepackage{graphicx, pgfplots}
\usepackage{url}
\usepackage[dvipsnames]{xcolor}
\usepackage{sagetex}

%loaded last
\usepackage[hidelinks]{hyperref}

\usepackage{siunitx}
  \sisetup{exponent-product = \cdot,
    output-decimal-marker = {,}}

%Giles Castelles incfig
\usepackage{import}
\usepackage{xifthen}
\usepackage{pdfpages}
\usepackage{transparent}

\newcommand{\incfig}[2][1]{%
  \def\svgwidth{#1\columnwidth}
  \import{../figures/}{#2.pdf_tex}
}

\setlength{\parindent}{0in}
\setlength{\oddsidemargin}{0in}
\setlength{\textwidth}{6.5in}
\setlength{\textheight}{8.8in}
\setlength{\topmargin}{0in}
\setlength{\headheight}{18pt}

\usepackage{fancyhdr}
\pagestyle{fancy}

\fancyhead{}
\fancyfoot{}
\fancyfoot[R]{\thepage}
\fancyhead[C]{\leftmark}

\pgfplotsset{compat=newest}

\pgfplotsset{every axis/.append style={
  axis x line=middle,    % put the x axis in the middle
  axis y line=middle,    % put the y axis in the middle
  axis line style={<->,color=black}, % arrows on the axis
}}

\usepackage{thmtools}
\usepackage{tcolorbox}
  \tcbuselibrary{skins, breakable}
  \tcbset{
    space to upper=1em,
    space to lower=1em,
  }

\theoremstyle{definition}

\newtcolorbox[auto counter]{definition}[1][]{%
  breakable,
  colframe=ForestGreen,  %frame color
  colback=ForestGreen!5, %background color
  colbacktitle=ForestGreen!25, %background color for title
  coltitle=ForestGreen!70!black,  %title color
  fonttitle=\bfseries\sffamily, %title font
  left=1em,              %space on left side in box,
  enhanced,              %more options
  frame hidden,          %hide frame
  borderline west={2pt}{0pt}{ForestGreen},  %display left line
  title=Definition \thetcbcounter: #1,
}

\newtcolorbox{greenline}{%
  breakable,
  colframe=ForestGreen,  %frame color
  colback=white,          %remove background color
  left=1em,              %space on left side in box
  enhanced,              %more options
  frame hidden,          %hide frame
  borderline west={2pt}{0pt}{ForestGreen},  %display left line
}

\newtcolorbox[auto counter, number within=section]{eks}[1][]{%
  brekable,
  colframe=NavyBlue,  %frame color
  colback=NavyBlue!5, %background color
  colbacktitle=NavyBlue!25,    %background color for title
  coltitle=NavyBlue!70!black,  %title color
  fonttitle=\bfseries\sffamily, %title font
  left=1em,            %space on left side in box,
  enhanced,            %more options
  frame hidden,        %hide frame
  borderline west={2pt}{0pt}{NavyBlue},  %display left line
  title=Eksempel \thetcbcounter: #1
}

\newtcolorbox{blueline}{%
  breakable,
  colframe=NavyBlue,     %frame color
  colback=white,         %remove background
  left=1em,              %space on left side in box,
  enhanced,              %more options
  frame hidden,          %hide frame
  borderline west={2pt}{0pt}{NavyBlue},  %display left line
}

\newtcolorbox{teo}[1][]{%
  breakable,
  colframe=RawSienna,  %frame color
  colback=RawSienna!5, %background color
  colbacktitle=RawSienna!25,    %background color for title
  coltitle=RawSienna!70!black,  %title color
  fonttitle=\bfseries\sffamily, %title font
  left=1em,              %space on left side in box,
  enhanced,              %more options
  frame hidden,          %hide frame
  borderline west={2pt}{0pt}{RawSienna},  %display left line
  title=Teori: #1,
}

\newtcolorbox[auto counter, number within=section]{sæt}[1][]{%
  breakable,
  colframe=RawSienna,  %frame color
  colback=RawSienna!5, %background color
  colbacktitle=RawSienna!25,    %background color for title
  coltitle=RawSienna!70!black,  %title color
  fonttitle=\bfseries\sffamily, %title font
  left=1em,              %space on left side in box,
  enhanced,              %more options
  frame hidden,          %hide frame
  borderline west={2pt}{0pt}{RawSienna},  %display left line
  title=Sætning \thetcbcounter: #1,
  before lower={\textbf{Bevis:}\par\vspace{0.5em}},
  colbacklower=RawSienna!25,
}

\newtcolorbox{redline}{%
  breakable,
  colframe=RawSienna,  %frame color
  colback=white,       %Remove background color
  left=1em,            %space on left side in box,
  enhanced,            %more options
  frame hidden,        %hide frame
  borderline west={2pt}{0pt}{RawSienna},  %display left line
}

\newtcolorbox{for}[1][]{%
  breakable,
  colframe=NavyBlue,  %frame color
  colback=NavyBlue!5, %background color
  colbacktitle=NavyBlue!25,    %background color for title
  coltitle=NavyBlue!70!black,  %title color
  fonttitle=\bfseries\sffamily, %title font
  left=1em,              %space on left side in box,
  enhanced,              %more options
  frame hidden,          %hide frame
  borderline west={2pt}{0pt}{NavyBlue},  %display left line
  title=Forklaring #1,
}

\newtcolorbox{bem}{%
  breakable,
  colframe=NavyBlue,  %frame color
  colback=NavyBlue!5, %background color
  colbacktitle=NavyBlue!25,    %background color for title
  coltitle=NavyBlue!70!black,  %title color
  fonttitle=\bfseries\sffamily, %title font
  left=1em,              %space on left side in box,
  enhanced,              %more options
  frame hidden,          %hide frame
  borderline west={2pt}{0pt}{NavyBlue},  %display left line
  title=Bemærkning:,
}

\makeatother
\def\@lecture{}%
\newcommand{\lecture}[3]{
  \ifthenelse{\isempty{#3}}{%
    \def\@lecture{Lecture #1}%
  }{%
    \def\@lecture{Lecture #1: #3}%
  }%
  \subsection*{\makebox[\textwidth][l]{\@lecture \hfill \normalfont\small\textsf{#2}}}
}

\makeatletter

\newcommand{\opgave}[1]{%
 \def\@opgave{#1}%
 \subsection*{Opgave #1}
}

\makeatother

%Format lim the same way in intext and in display
\let\svlim\lim\def\lim{\svlim\limits}

% horizontal rule
\newcommand\hr{
\noindent\rule[0.5ex]{\linewidth}{0.5pt}
}

\title{TØ-opgaver uge 8}
\author{Noah Rahbek Bigum Hansen}
\date{29. Oktober 2024}

\begin{document}

\maketitle

\section*{Opg. 1}
Find den fuldstændige løsning til den homogene differentialligning
\[
  y'' + 6y' + 5y = 0
.\]
\bigbreak
Først opskrives den karakteristiske ligning for differentialligningen som så
\[
r^2 + 6r + 5 = 0 \implies D = 6^2 - 4\cdot 1\cdot 5 = 16
.\]
Da diskriminanten er over 0 må løsningen være på formen
\[
y(x) = C_1e^{r_1x} + C_2e^{r_2x} 
.\]
De to rødder til den karakteristiske ligning
\[
r = \frac{-6 \pm 4}{2} = -3 \pm 2 
.\]
Altså er den fuldstændige løsning til den homogene differentialligning
\[
y(x) = C_1e^{-5x} + C_2e^{-x}  
.\]

\section*{Opg. 2}
Find løsningen til den homogene differentialligning
\[
y'' + 6y' + 5y = 0
\]
der opfylder begyndelsesbetingelserne $y(0) = 0$ og $y'(0) = 3$.
\bigbreak
Vi har den fuldstændige løsning fra før
\[
y(x) = C_1e^{-5x} + C_2e^{-x} 
.\]
Denne differentieres
\[
y'(x) = -5C_1e^{-5x} - C_2e^{-x} 
.\]
Vi har dermed
\begin{align}
y(0) &= 0 \\
C_1e^{0} + C_2e^{0} &= 0 \\
C_1 + C_2 &= 0 \label{eq:3}
\end{align}
Og
\begin{align}
y'(0) &= 3 \\
-5C_1e^{0} - C_2e^{0} &= 3 \\
-5C_1 - C_2 &= 3 \\
C_2 &= - 3 - 5C_1 \label{eq:6} 
\end{align}
Ved at indsætte \textbf{\autoref{eq:6}} i \textbf{\autoref{eq:3}} fås
\begin{align*}
C_1 - 3 - 5C_1 &= 0 \\
4C_1 &= -3 \\
C_1 &= -\frac{3}{4} \implies C_2 = \frac{3}{4}
\end{align*}
Dermed er den partikulære løsning
\[
y(x) = -\frac{3}{4}e^{-5x} + \frac{3}{4}e^{-x} 
.\]

\section*{Opg. 3}
Find løsningerne til den homogene differentialligning
\[
y'' + 16y = 0
\]
der opfylder begyndelsesbetingelserne $y(0) = 2$ og $y'(0)=-2$ 
\bigbreak
Først opskrives den karakteristiske ligning som
\[
r^2 + 16 = 0 \implies r^2 = - 15 \implies r = \pm 4i 
.\]
Eftersom diskriminanten er negativ må det gælde at følgende løsning er den fuldstændige løsning
\begin{align*}
  y(x) &= e^{0x}\left( C_1 \cos \left( 4x \right) + C_2 \sin \left( 4x \right)   \right)  \\
  &= C_1 \cos \left( 4x \right) + C_2 \sin \left( 4x \right)
\end{align*}
Vi differentierer denne
\[
y'(x) = -4C_1 \sin \left( 4x \right) + 4C_2 \cos \left( 4x \right) 
.\]
Og indsætter begyndelsesbetingelserne
\begin{align*}
  2 &= C_1 \cos \left( 0 \right) + C_2 \sin \left( 0 \right)  \\
  C_1 &= 2
\end{align*}
Og
\begin{align*}
  -2 &= -2\cdot4 \sin \left( 0 \right) + C_2\cdot 4 \cos \left( 0 \right)  \\
  &= 4C_2 \\
  C_2 &= -\frac{2}{4} = -\frac{1}{2}
\end{align*}
Altså bliver den partikulære løsning
\[
y(x) = 2 \cos \left( 4x \right) - \frac{1}{2}\sin \left( 4x \right)  
.\]

\section*{Opg. 4}
Find fuldstændig løsning til den inhomogene differentialligning
\[
7y'' - y' - 2y = 20 \cos \left( x \right) 
.\]
\bigbreak
For at finde løsningen til den inhomogene differentialligning løses først den tilsvarende homogene differentialligning så vi starter med at opskrive den karakteristiske ligning
\[
7r^2 - r - 2 = 0
.\]
Diskriminanten findes
\[
D = 1 - 4\cdot 7\cdot (-2) = 57
.\]
Altså har vi en positiv diskriminant og derfor må løsningen være på formen
\[
y(x) = C_1e^{r_1x} + C_2e^{r_2x} 
.\]
Vi finder den karakteristiske lignings rødder
\[
r = \frac{1 \pm \sqrt{57}}{14} 
.\]
Altså har vi
\[
y(x) = C_1 e^{ \frac{1+\sqrt{57}}{14}x} + C_2 e^{ \frac{1-\sqrt{57}}{14}x}  
.\]
Vi benytter \textit{ubestemte koefficienters metode} så dermed har vi at vores 'gæt' på en partikulær løsning skal være på formen
\begin{align*}
  y_p &= A \cos \left( x \right) + B \sin \left( x \right) \\
  y_p' &= -A \sin \left( x \right) + B \cos \left( x \right)  \\
  y_p'' &= -A \cos \left( x \right) - B \sin \left( x \right)   \\
\end{align*}
Vi indsætter dette i vores differentialligning
\begin{align*}
  20 \cos \left( x \right) &= 7\left( -A \cos \left( x \right) - B \sin \left( x \right)  \right)  - \left( -A \sin \left( x \right) + B \cos \left( x \right)  \right) - 2\left( A \cos \left( x \right) + B \sin \left( x \right)  \right) \\
  &= -7A \cos \left( x \right) - 7B \sin \left( x \right) + A \sin \left( x \right) - B \cos \left( x \right) - 2A \cos \left( x \right) - 2B \sin \left( x \right) \\
  &=  -9A \cos \left( x \right) - 9B \sin \left( x \right) + A \sin \left( x \right) - B \cos \left( x \right) \\
  &= \cos \left( x \right)\left( -9A - B \right) + \sin \left( x \right) \left( -9B + A \right) 
\end{align*}
Vi har dermed to ligninger med to ubekendte som løses ved
\begin{align*}
  0 &= -9B + A & \qquad 20 &= -9A - B \\
  A &= 9B & \qquad 20 &= -81B - B  \\
  A &= 9\cdot \left( -\frac{10}{41} \right) &\qquad 20 &= -82B \\
  A &= -\frac{90}{41} &\qquad B &= -\frac{20}{82} = -\frac{10}{41} \\
\end{align*}
Altså har vi at løsningen til differentialligningen er
\begin{align*}
  y(x) &= y_{hom} + y_p \\
  &= C_1e^{ \frac{1+\sqrt{57} }{14} } + C_2e^{ \frac{1-\sqrt{57} }{14} } - \frac{90}{41}\cos \left( x \right) - \frac{10}{41}\sin \left( x \right) 
\end{align*}


\section*{Opg. 5}
Find fuldstændig løsning til den inhomogene differentialligning
\[
y'' - y = e^x + x^2
.\]
\bigbreak
Den karakteristiske ligning opskrives for den tilsvarende homogene differentialligning
\[
r^2-1 = 0 \implies r = \pm 1 
.\]
Den komplementære løsning er altså
\[
y_c(x) = C_1e^{x} + C_2e^{-x}
.\]
Vi benytter \textit{ubestemte koefficienters metode} og gætter på en løsning på formen
\begin{align*}
y_p &= Axe^{x} + Bx^2 + Cx + D  \\
y_p' &= Axe^{x} + Ae^{x} + 2Bx + C  \\
y_p'' &= Axe^{x} + 2Ae^{x} + 2B
\end{align*}
Vi indsætter dette i differentialligningen
\begin{align*}
e^x + x^2 &= Axe^{x} + 2Ae^{x} + 2B - Axe^{x} - Bx^2 - Cx - D \\
&= 2Ae^{x} + 2B - Bx^2 - Cx - D
\end{align*}
Vi løser for de ubekendte
\begin{align*}
  1 &= 2A & &\iff & A &= \frac{1}{2} \\
  1 &= -B & &\iff & B &= -1 \\
  0 &= C & &\iff & C &= 0 \\
  0 = 2B &- D = -2 - D & &\iff & D &= -2 \\
\end{align*}
Altså bliver den løsningen
\[
y(x) = C_1e^{x} + C_2e^{-x} + \frac{1}{2}xe^{x} - x^2 + 2
.\]


\section*{Opg. 6}
Find løsningen til den inhomogene differentialligning
\[
y'' + y = e^{2x} 
\]
der opfylder begyndelsesbetingelserne $y(0) = 0$ og $y'(0) = \frac{2}{5}$
\bigbreak
Vi løser den karakteristiske ligning
\[
r^2 + 1 = 0 \implies r = \pm i
.\]
Dette giver den komplementære løsning
\[
y_c(x) = C_1 \cos \left( x \right) + C_2 \sin \left( x \right) 
.\]
Vha. \textit{ubestemte koefficienters metode} gætter vi på at den partikulære løsning er på formen 
\begin{align*}
  y_p &= Ae^{2x} \\
  y_p' &= 2Ae^{2x}  \\
  y_p'' &= 4Ae^{2x}  \\
\end{align*}
Vi indsætter dette i differentialligningen
\begin{align*}
  e^{2x} &= 4Ae^{2x} + Ae^{2x}  \\
  \implies A &= \frac{1}{5} \\
\end{align*}
Den generelle løsning er altså
\[
y(x) = C_1 \cos \left( x \right) + C_2 \sin \left( x \right) + \frac{1}{5}e^{2x} 
.\]
Denne differentieres til
\[
y'(x) = -C_1 \sin \left( x \right) + C_2 \cos \left( x \right) + \frac{2}{5}e^{2x} 
.\]
Vi indsætter begyndelsesbetingelserne
\begin{align*}
  0 &= C_1 + \frac{1}{5} & &\implies & C_1 &= -\frac{1}{5} \\
  \frac{2}{5} &= C_2 + \frac{2}{5} & &\implies & C_2 &= 0
\end{align*}
Altså er løsningen
\[
y(x) = -\frac{1}{5}\cos \left( x \right) + \frac{1}{5}e^{2x} 
.\]


\section*{Prøveeksamensspørgsmål 13}
Bestem løsning til følgende homogene differentialligning med begyndelsesbetingelser
\[
2y'' + 2y' + y = 0, \qquad y(0) = 4, \qquad y'(0) = 2
.\]
For et tal $a$ er løsningen:
\[
y(x) = 4e^{-\frac{1}{2}x} \left( a \cdot \sin \left( \frac{x}{a} \right) + \cos \left( \frac{x}{a} \right)  \right) 
.\]
Svaret er et helt tal mellem \num{0} og \num{99}.
\bigbreak
Den karakteristiske ligning opskrives
\[
2r^2 + 2r + 1 = 0
.\]
Løsningerne findes
\[
r = \frac{-2 \pm \sqrt{2^2 - 4\cdot 2\cdot 1}}{2\cdot 2} = \frac{-2 \pm 2i}{4} = -\frac{1}{2}\pm \frac{1}{2}i  
.\]
Altså bliver løsningen
\[
y(x) = C_1e^{-\frac{1}{2}x} \cos \left( \frac{1}{2}x \right) + C_2e^{-\frac{1}{2}x} \sin \left( -\frac{1}{2}x \right) 
.\]
Altså må $a$ være $2$.

\section*{Prøveeksamensspørgsmål 14}
Bestem løsning til følgende inhomogene differentialligning med begyndelsesbetingelser
\[
y'' - 2y' + y = e^{x}, \qquad y(0) = 1, \qquad y'(0) = 2
.\]
Løsningen er
\[
y(x) = \left( \frac{1}{2}x^2 + Px + 1 \right)e^x
.\]
For et helt tal $0<P<99$
\bigbreak
Den karakteristiske ligning opskrives
\[
r^2 - 2r + 1 = 0
.\]
Dennes løsninger findes
\[
r = \frac{2 \pm \sqrt{4-4\cdot1 \cdot 1} }{2} = 1 
.\]
Altså er den fuldstændige løsning
\[
y(x) = C_1xe^{x} + C_2e^{x} 
.\]
Vores løsningsgæt er
\begin{align*}
  y_p &= Ax^2e^{x}  \\
  y_p' &= Axe^{x} + Ax^2e^{x} \\
  y_p'' &= 2Ax^2e^{x} + 2Ae^{x} + Ax^2e^{x} + 2Axe^{x} = 4Axe^{x} + 2Ae^{x} + Ax^2e^{x}
\end{align*}
Dette indsættes i differentialligningen
\begin{align*}
  e^{x} &= 4Axe^{x} + 2Ae^{x} + Ax^2e^{x} - 4Axe^{x} - 2Ax^2e^{x} + Ax^2e^{x} \\
  &= 2Ax^2 e^{x}  \\
  \implies A &= \frac{1}{2} 
\end{align*}
Dette indsættes og vi får at
\[
y(x) = \left( C_1x + C_2 + \frac{1}{2}x^2 \right)e^{x} 
.\]
Altså har vi
\[
y'(x) = \left( C_1x + C_2 + \frac{1}{2}x^2 \right) e^{x} + \left( C_1 + x \right)e^{x}
.\]
Vi indsætter begyndelsesbetingelserne
\begin{align*}
1 &= C_2  \\
2 &= 1+C_1 \implies C_1 = 1
\end{align*}
Altså er vores partikulære løsning
\[
y(x) = \left( 1 + x + \frac{1}{2}x^2 \right)e^x
.\]
Svaret må altså være $P=1$


\end{document}
