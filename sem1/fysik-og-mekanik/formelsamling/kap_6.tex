\section{Arbejde og kinetisk energi}

\begin{table}[ht]
\begin{tabular}{|l|l|l|}
\hline
\textbf{Givet}                    & \textbf{Ønsker at finde} & \textbf{Relevante formler} \\ \hline
Kraft, strækning, (vinkel) & Arbejde & \begin{tabular}[c]{@{}l@{}}\ref{afs:w1d} – 1 dimension\\ \ref{afs:w2d} – flere dimensioner\end{tabular}          \\ \hline
Masse, hastighed                  & Kinetisk energi          & \ref{afs:enkin}            \\ \hline
Forskel mellem kinetiske energier & Arbejde                  & \ref{afs:wetheo}           \\ \hline
Kraft, strækning                  & Arbejde                  & \ref{afs:wvf}              \\ \hline
Kraft, strækning, vinkel          & Arbejde                  & \ref{afs:wetheocur}        \\ \hline
Ændring i arbejde, tid     & Effekt  & \begin{tabular}[c]{@{}l@{}}\ref{afs:aveff} – gennemsnitseffekt\\ \ref{afs:inseff} – øjeblikseffekt\end{tabular} \\ \hline
Kraft, hastighed                  & Effekt                   & \ref{afs:inseff}           \\ \hline
\end{tabular}
\end{table}


\subsection{Arbejde af konstante kræter (6.1-6.2)}

\subsubsection{Arbejde i 1 dimension} \label{afs:w1d}
Arbejdet $W$ udført af en konstant kraft $F$ over en strækning $s$ kan findes som
\[ 
W = Fs
.\]

\subsubsection{Arbejde i flere dimensioner} \label{afs:w2d}
Arbejdet $W$ udført af en konstant kraft $F$ over en strækning $s$ med en vinkel $\phi$ mellem $F$ og $s$ er
\[ 
W = Fs \cos\phi
.\]

Dette er i øvrigt det samme som prikproduktet mellem kraft-vektoren $\Vec{F}$ og strækningsvektoren $\Vec{s}$. Altså
\[ 
W = \Vec{F} \cdot \Vec{s}
.\]

\subsubsection{Kinetisk energi} \label{afs:enkin} 
Den kinetiske energi $K$ af et objekt med masse $m$ og hastighed $v$ kan findes som
\[ 
K = \frac{1}{2}mv^2
.\]


\subsubsection{Arbejde-energi-teoremet} \label{afs:wetheo}
Arbejde-energi-teoremet lyder, at arbejdet $W_{tot}$ udført af den resulterende kraft på en partikel tilsvarer ændringen i partiklens energi $\Delta K$. Altså
\[ 
W_{tot} = \Delta K = K_2 - K_1
.\]
Dette betyder bl.a. at den kinetiske energi af en partikel netopsvarer til alt det arbejde der er udført på partiklen siden stilstand.


\subsection{Arbejde og energi for variable kræfter (6.3)}

\subsubsection{Arbejde af variabel kraft} \label{afs:wvf}
For en 1-dimensional variabel kraft $F_x$, der virker fra $x_1$ til $x_2$ er det totale arbejde udført af kraften givet som
\[ 
W = \int_{x_1}^{x_2} F_x \, \mathrm{d}x
.\]
Det ses også at såfremt kraften er konstant simplificeres udtrykket ovenfor til arbejdet for en konstant kraft
\[ 
W = \int_{x_1}^{x_2} F_x \, \mathrm{d}x = F_x \int_{x_1}^{x_2} \, \mathrm{d}x = F_x(x_2-x_1)
.\]

\subsubsection{Arbejde-energi-teoremet for bevægelse langs en kurve} \label{afs:wetheocur}
Givet en partikel der bevæger sig langs en kurve fra $P_1$ til $P_2$ kan kurven deles op i en række små forskydninger $\mathrm{d}\Vec{I}$. Kaldes kraften ved hver lille forskydning $\mathrm{d}\Vec{I}$ for $\Vec{F}$ og vinklen mellem $\Vec{F}$ og $\mathrm{d}\Vec{I}$ for $\phi$ kan det totale arbejde findes som
\[ 
W = \int_{P_1}^{P_2} \Vec{F} \cdot \, \mathrm{d}\Vec{I} = \int_{P_1}^{P_2} F \cdot \cos\phi \, \mathrm{d}I
.\]

\subsection{Effekt (6.4)}

\subsubsection{Gennemsnitseffekt} \label{afs:aveff}
Givet en ændring i arbejde $\Delta W$ og en ændring i tid $\Delta t$ kan den gennemsnitlige effekt af arbejdet $P_{avg.}$ findes som
\[ 
  P_{avg.} = \frac{\Delta W}{\Delta t}
.\]

\subsubsection{Øjeblikseffekt} \label{afs:inseff}
For en ændring i arbejde $\Delta W$ og en ændring i tid $\Delta t$ kan øjeblikseffekten $P$ findes som
\[ 
P = \lim_{\Delta t \to 0} \frac{\Delta W}{\Delta t} = \frac{\mathrm{d}W}{\mathrm{d}t}
.\]

For en kraft $\Vec{F}$ der udfører et arbejde på en partikel med en hastighed $\Vec{v}$ kan øjeblikseffekten i øvrigt findes som
\[ 
P = \Vec{F} \cdot \Vec{v}
.\]

