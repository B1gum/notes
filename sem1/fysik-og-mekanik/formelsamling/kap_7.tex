\section{Potentiel energi og energikonservation}

\begin{table}[ht]
\begin{tabular}{|l|l|l|}
\hline
\textbf{Givet}                      & \textbf{Ønsker at finde}         & \textbf{Relevante formler} \\ \hline
Masse, højde                        & Potentiel energi for tyngdekraft & \ref{afs:gravpot}          \\ \hline
masse, højdeforskel                 & Tyngdekraftens arbejde           & \ref{sec:gravwork}         \\ \hline
Kinetisk- og potentiel energiforskel &
  Energikonservation &
  \begin{tabular}[c]{@{}l@{}}\ref{afs:ekonstyn} – i tyngdefelt\\ \ref{afs:ebevkr} – inkl. eksterne kræfter\\ \ref{afs:ekons} – ændringer lig 0\end{tabular} \\ \hline
Fjederkonstant, udstrækning         & Potentiel energi i fjeder        & \ref{afs:epotela}          \\ \hline
Forskel i potentiel energi i fjeder & Arbejdet af en fjeder            & \ref{afs:wela}             \\ \hline
Potentiel energi &
  Kraft &
  \begin{tabular}[c]{@{}l@{}}\ref{sec:potfor1} – 1 dimension\\ \ref{afs:potfor3} – 3 dimensioner\end{tabular} \\ \hline
\end{tabular}
\end{table}

\subsection{Gravitationel potentiel energi (7.1)} \label{afs:gravpot}
Betragtes en partikel kun udsat for en tyngdekraft kan dens samlede energi findes som
\[ 
U_{grav} = mgy
.\]
Hvor $m$ er partiklens masse, $g$ er tyngdeaccelerationen og $y$ er den vertikale afstand af partiklen. 

\subsubsection{Tyngdekraftens arbejde} \label{sec:gravwork}
Givet en partikels masse $m$, tyngdeaccelerationen $g$ starthøjden $y_1$ og sluthøjden $y_2$ kan det samlede arbejde udført af tyngdefeltet findes som
\[
  W_{grav} = mgy_1 - mgy_2 = U_{grav,1} - U_{grav,2} = -\Delta U_{grav}
\]

\subsubsection{Energikonservation i et tyngdefelt} \label{afs:ekonstyn}
Såfremt kun tyngdekraften yder et arbejde på en partikel er mekanisk energi konserveret, altså
\[ 
K_1 + U_{grav,1} = K_2 - U_{grav,2} \implies \frac{1}{2}mv_1^2 + mgy_1 = \frac{1}{2}mv_2^2 + mgy_2
.\]
Hvor $m$ er partiklens masse, $v_1$ og $v_2$ er henholdsvis partiklens start- og sluthastighed, $g$ er tyngdeaccelerationen og $y_1$ og $y_2$ er henholdsvis partiklens start- og sluthøjde.

\subsection{Elastisk potentiel energi (7.2)}

\subsubsection{Elastisk potentiel energi} \label{afs:epotela}
Givet fjederkonstanten $k$ og forskydningen af en fjeder $x$ ($x>0$ for en udstrakt fjeder og $x<0$ for en sammenpresset fjeder) kan den elastiske potentielle energi lagret i fjederen findes som
\[ 
U_{el} = \frac{1}{2}kx^2
.\]

\subsubsection{Arbejdet udført af den elastiske kraft} \label{afs:wela}
Arbejdet udført af den elastiske kraft er på mange måder parallelt med \ref{sec:gravwork}: \nameref{sec:gravwork} idet arbejdet udført af den elastiske kraft kan findes som
\[ 
W_{el} = \frac{1}{2}kx_1^2 - \frac{1}{2}kx_2^2 = U_{el,1} - U_{el,2} = -\Delta U_{el}
.\]
Hvor $k$ er fjederkonstanten og $x$ er fjederens forskydning ($x>0$ for en udstrakt fjeder og $x<0$ for en sammenpresset fjeder).


\subsubsection{Mekanisk energibevarelse med arbejde fra andre kræfter} \label{afs:ebevkr}
Mekanisk energibevarelse foreskriver at
\[ 
K_1 + U_1 + W_{other} = K_2 + U_2
.\]
Hvor $K_1$ og $K_2$ er hhv. start- slut-kinetisk energi, $U_1$ og $U_2$ er start- og slut-potentiel energi og $W_{other}$ er arbejdet udført af alle kræfter der ikke er associeret med den potentielle energi. Altså er arbejdet udført af alle andre kræfter end tyngdekraften og den elastiske kraft lig ændringen i mekanisk energi.


\subsection{Konservative og ikke-konservative kræfter (7.3)}

\subsubsection{Konservation af energi} \label{afs:ekons}
Energikonservationsloven foreskriver at
\[ 
\Delta K + \Delta U + \Delta U_{int} = 0
.\]
Hvor $\Delta K$ er ændringen i kinetisk energi, $\Delta U$ er ændringen i potentiel energi og $\Delta U_{int}$ er ændringen i indre energi. Det gælder generelt at energi er konserveret for lukkede systemer, hvor der kun virker konservative kræfter. Konservative kræfter (tyngdekraften, fjederkraften, elektromagnetiske kræfter) er alle kræfter der er stiuafhængige. 


\subsection{Kraft og potentiel energi (7.4)}

\subsubsection{Kraft fra potentiel energi i en dimension} \label{sec:potfor1}
Givet en potentiel-energi funktion i en dimension $U(x)$ kan den associerede kraft findes som
\[ 
F_x(x) = -\frac{\mathrm{d}U(x)}{\mathrm{d}x} 
.\]
Resultatet ovenfor gælder kun for konservative kræfter (tyngde-, fjeder- eller elektromagnetisk-kraft). For eksempelvis fjederkraften ses dog også at resultatet holder idet
\[ 
F_x(x) = -\frac{\mathrm{d}}{\mathrm{d}x} \frac{1}{2}kx^2 = -kx
.\]

\subsubsection{Kraft fra potentiel energi i 3 dimensioner} \label{afs:potfor3}
Analogt med resultatet fra \ref{sec:potfor1}: \nameref{sec:potfor1} kan det vises at komposanterne for en konservativ kraft ($F_x, F_y$ og $F_z$) kan findes som den negative partielt afledede til det punkt af den associerede potentiel-energi-funktion ($U(x, y, z)$) som
\[ 
F_x = -\frac{\partial U}{\partial x}, \qquad F_y = - \frac{\partial U}{\partial y} \qquad F_z = - \frac{\partial U}{\partial z}
.\]
Dette kan også skrives som et samlet udtryk for $\Vec{F}$ som
\[ 
\Vec{F} = -\left( \frac{\partial U}{\partial x}\hat{\imath} + \frac{\partial U}{\partial y}\hat{\jmath} + \frac{\partial U}{\partial z}\hat{k} \right) = - \Vec{\nabla} U
.\]
