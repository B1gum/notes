\section{Newtons bevægelseslove}

\subsection{Krafter og interaktioner (4.1)}

\subsubsection{Den resulterende kraft}
For et legeme påvirket af kræfter kan den samlede kraftpåvirkning $\Vec{F_{res}}$ findes som summen af af de andre påvirkende kræfter $\Vec{F_1}, \Vec{F_2}, \Vec{F_3},\ldots $ som
\[ 
\Vec{F}_{\text{res}} = \sum \Vec{F} = \Vec{F_1} + \Vec{F_2} + \Vec{F_3} + \ldots 
.\]

\subsection{Newtons love (4.2-4.5)}

\subsubsection{Newtons 1. lov}
Newtons 1. lov siger, at et objekt der bliver påvirket af $F_{res} = 0$ har en acceleration $a = 0$. Dette kan også formuleres som, at et objekt i ligevægt har en samlet kraftpåvirkning på 0. Altså
\[ 
\sum \Vec{F} = 0
.\]

\subsubsection{Newtons 2. lov}
Newtons 2. lov foreskriver at der er ligefrem proportionalitet mellem accelerationen og den resulterende kraft som forårsager accelerationen. Proportionalitetsfaktoren vil da være objektets masse $m$. Desuden bemærker Newtons 2. lov, at retningen på accelerationen vil være den samme som retningen for den resulterende kraft. Altså
\[ 
\sum \Vec{F} = m \Vec{a}
.\]

\subsubsection{Newtons 2. lov med komposanter}
Newtons 2. lov gælder uafhængigt i alle bevægelsesretninger -- den del af kraften der skubber i en given retning er proportionel med accelerationen i samme retning. Vi har altså
\[ 
\sum F_x = ma_x, \qquad \sum F_y = ma_y, \qquad \sum F_z = ma_z
.\]

\subsubsection{Newtons 3. lov}
Newtons 3. lov foreskriver, at hvis et objekt $A$ yder en kraft på et objekt $B$ (en \textit{aktion}) så vil objekt $B$ udøve en tilsvarende men modsatrettet kraft på objekt $A$ (en \textit{reaktion}). Altså
\[ 
\Vec{F}_{A / B} = - \Vec{F}_{B / A}
.\]

