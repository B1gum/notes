\section{Periodisk bevægelse}

\begin{table}[ht]
\begin{tabular}{|l|l|l|}
\hline
\textbf{Givet} &
  \textbf{Ønsker at finde} &
  \textbf{Relevante formler} \\ \hline
Periode &
  Frekvens &
  \ref{afs:frekper} \\ \hline
Periode eller frekvens &
  Vinkelfrekvens &
  \ref{afs:vinfrek} \\ \hline
Fjederkonstant, forskydning &
  Kraft i SHM &
  \ref{afs:SHMkra} \\ \hline
Fjederkonstant, masse, forskydning &
  Acceleration i SHM &
  \ref{afs:SHMacc} \\ \hline
Fjederkonstant, masse &
  Vinkelfrekvens i SHM &
  \ref{afs:SHMvinfrek} \\ \hline
Vinkelfrekvens &
  Frekvens i SHM &
  \ref{afs:SHMfrek} \\ \hline
Fjederkonstant, masse &
  Frekvens i SHM &
  \ref{afs:SHMfrek} \\ \hline
Frekvens &
  Periode i SHM &
  \ref{afs:SHMper} \\ \hline
\begin{tabular}[c]{@{}l@{}}Amplitude, vinkelfrekvens, tid, \\ faseforskydning\end{tabular} &
  Position i SHM &
  \ref{afs:SHMpos} \\ \hline
\begin{tabular}[c]{@{}l@{}}Masse, hastighed, fjederkonstant,\\  forskydning, amplitude\end{tabular} &
  Mekanisk energi i SHM &
  \ref{afs:emekSHM} \\ \hline
Kraftkonstant og mase &
  \begin{tabular}[c]{@{}l@{}}Vinkelfrekvens for simpelt\\ pendul\end{tabular} &
  \ref{afs:vinfrekspen} \\ \hline
Længde &
  \begin{tabular}[c]{@{}l@{}}Vinkelfrekvens for simpelt\\ pendul\end{tabular} &
  \ref{afs:vinfrekspen} \\ \hline
Vinkelfrekvens eller længde &
  Periode for simpelt pendul &
  \ref{afs:perspen} \\ \hline
Masse, længde, inertimoment &
  \begin{tabular}[c]{@{}l@{}}Vinkelfrekvens for fysisk\\ pendul\end{tabular} &
  \ref{afs:vinfrekfpen} \\ \hline
Inertimoment, masse, længde &
  Periode for fysisk pendul &
  \ref{afs:perfpen} \\ \hline
\begin{tabular}[c]{@{}l@{}}Amplitude, dæmpningskonstant, \\ masse, vinkelfrekvens, tid, \\ faseforskydning\end{tabular} &
  \begin{tabular}[c]{@{}l@{}}Forskydning i dæmpet\\ oscillation\end{tabular} &
  \ref{afs:foroscdæmp} \\ \hline
Kraftkonstant, masse, dæmpningskonstant &
  \begin{tabular}[c]{@{}l@{}}Vinkelfrekvens i dæmpet\\ oscillation\end{tabular} &
  \ref{afs:vinfrekoscdæmp} \\ \hline
\begin{tabular}[c]{@{}l@{}}Størrelsen af drivkraften, kraftkonstant,\\ masse, drivende vinkelfrekens,\\ dæmpningskonstant\end{tabular} &
  \begin{tabular}[c]{@{}l@{}}Amplitude af tvungen\\ oscillation\end{tabular} &
  \ref{afs:amptvu} \\ \hline
\begin{tabular}[c]{@{}l@{}}Initialamplitude, dæmpningskonstand,\\ masse, tid\end{tabular} &
  \begin{tabular}[c]{@{}l@{}}Amplitude til tid \\af dæmpet oscillation\end{tabular} &
  \ref{afs:ampdæmtid} \\ \hline
\begin{tabular}[c]{@{}l@{}}Inertimoment, kraft,\\ kraftarm\end{tabular} &
  \begin{tabular}[c]{@{}l@{}}Bevægelsesligning for pendul\end{tabular} &
  \ref{afs:bevpen} \\ \hline
\end{tabular}
\end{table}


\subsection{Beskrivelse af oscillationer (14.1)}

\subsubsection{Forholdet mellem frekvens og periode} \label{afs:frekper}
Det gælder at frekvensen af en oscillation $f$ og perioden af selvsamme $T$ er inverst proportionelle. Altså at
\[ 
f = \frac{1}{T} \iff T = \frac{1}{f}
.\]

\subsubsection{Vinkelfrekvens og periode eller frekvens} \label{afs:vinfrek}
Vinkelfrekvensen $\omega$ forholder sig til frekvensen $f$ og perioden $T$ som
\[ 
\omega = 2\pi f = \frac{2\pi}{T}
.\]


\subsection{Simpel harmonisk bevægelse (SHM) (14.2)}

\subsubsection{Kraften for SHM (Fjederkraften)} \label{afs:SHMkra}
For ideelle fjedre (dvs. dem der overholder \ref{afs:hooke}: \nameref{afs:hooke}) gælder at 
\[ 
F_x = -kx
.\]
Hvor $F_x$ er den restaurerende kraft udøvet af fjederen, $k$ er fjederkonstanten og $x$ er forskydningen. 

\subsubsection{Acceleration for SHM} \label{afs:SHMacc}
Accelerationen $a_x$ for SHM kan findes som
\[ 
a_x = \frac{\mathrm{d}^2x}{\mathrm{d}t^2} = - \frac{k}{m}x
.\]
Hvor $x$ er forskydningen, $k$ er fjederkonstanten og $m$ er massen af objektet der laver SHM. 


\subsubsection{Vinkelfrekvens for SHM} \label{afs:SHMvinfrek}
Vinkelfrekvensen $\omega$ for et objekt der undergår $SHM$ kan findes som
\[ 
\omega = \sqrt{\frac{k}{m}}
.\]
Hvor $k$ er fjederkonstanten og $m$ er massen af objektet.


\subsubsection{Frekvens for SHM} \label{afs:SHMfrek}
Frekvensen $f$ for et objekt der undergår SHM kan findes som
\[ 
f = \frac{\omega}{2\pi} = \frac{1}{2\pi}\sqrt{\frac{k}{m}}
.\]
Hvor $\omega$ er vinkelfrekvensen, $k$ er fjederkonstanten og $m$ er massen af objektet.


\subsubsection{Perioden for SHM} \label{afs:SHMper}
Perioden $T$ for et objekt der undergår SHM kan findes som
\[ 
T = \frac{1}{f} = \frac{2\pi}{\omega} = 2\pi \sqrt{\frac{m}{k}}
.\]
Hvor $f$ er frekvensen, $\omega$ er vinkelhastigheden, $m$ er massen af objektet og $k$ er fjederkonstanten.


\subsubsection{Position som funktion af tid for SHM} \label{afs:SHMpos}
Givet en amlitude $A$, en vinkelfrekvens $\omega$ og en faseforskydning $\phi$ kan forskydningen af et objekt $x$ der undergår SHM til tiden $t$ findes som
\[ 
x = A \cos (\omega t + \phi)
.\]

\subsection{Energi i SHM (14.3)}

\subsubsection{Mekanisk energi for SHM} \label{afs:emekSHM}
Den totale mekaniske energi $E$ kan for et objekt der undergår SHM findes som
\[ 
E = \frac{1}{2}mv_x^2 + \frac{1}{2}kx^2 = \frac{1}{2}kA^2 = \text{const.}
.\]
Hvor $m$ er massen af objektet, $v_x$ er hastigheden, $k$ er fjederkonstanten, $x$ er forskydningen og $A$ er amplituden. 


\subsection{Det simple pendul (14.5)}

\subsubsection{Vinkelfrekvensen for et simpelt pendul} \label{afs:vinfrekspen}
For et simpelt (matematisk) pendul med tilpas lav amplitude kan vinkelfrekvensen $\omega$ findes som
\[ 
\omega = \sqrt{\frac{k}{m}} = \sqrt{\frac{g}{L}}
.\]
Hvor $k$ er pendulets kraftkonstant ($k = \frac{mg}{L}$), $m$ er pendulets masse og $L$ er pendulets længde.


\subsubsection{Frekvens af et simpelt pendul} \label{afs:frekspen}
For et simpelt pendul med tilpas lav amplitude kan frekvensen $f$ findes som
\[ 
f = \frac{\omega}{2\pi} = \frac{1}{2\pi} \sqrt{\frac{g}{L}}
.\]
Hvor $\omega$ er vinkelfrekvensen og $L$ er pendulets længde.


\subsubsection{Periode for et simpelt pendul} \label{afs:perspen}
For et simpelt pendul med tilpas lav amplitude kan perioden $T$ findes som
\[ 
T = \frac{1}{f} = \frac{2\pi}{\omega} = 2\pi \sqrt{\frac{L}{g}}
.\]
Hvor $f$ er pendulets frekvens, $\omega$ er dets vinkelfrekvens og $L$ er dets længde.



\subsection{Det fysiske pendul (14.6)}

\subsubsection{Bevægelsesligningen for et pendul} \label{afs:bevpen}
Bevægelsesligningen for et pendul er
\[ 
I \frac{\mathrm{d}^2 \theta}{\mathrm{d}t^2} = - F\cdot r \theta 
.\]
Hvor $I$ er pendulets inertimoment, $F$ er kraften og $r$ er armen.


\subsubsection{Vinkelfrekvens for et fysisk pendul} \label{afs:vinfrekfpen}
Vinkelfrekvensen $\omega$ for et fysisk pendul med tilpas lav amplitude kan findes som
\[ 
\omega = \sqrt{\frac{mgd}{I}}
.\]
Hvor $m$ er pendulets masse, $d$ er afstanden fra rotationsaksen tilpendulets massemidtpunkt og $I$ er dets inertimoment.


\subsubsection{Perioden for et fysisk pendul} \label{afs:perfpen}
Perioden $T$ for et fysisk pendul med tilpas lav amplitude kan findes som
\[ 
  T = 2\pi\sqrt{\frac{I}{mgd}}
.\]
Hvor $m$ er pendulets masse, $d$ er afstanden fra rotationsaksen tilpendulets massemidtpunkt og $I$ er dets inertimoment.

\subsection{Dæmpede oscillationer (14.7)}

\subsubsection{Forskydningen af oscillator med dæmpning} \label{afs:foroscdæmp}
Givet en dæmpet oscillator med relativt lav dæmpningsgrad kan forskydningen $x$ som funktion af tiden $t$ findes som
\[ 
x(t) = Ae^{-\left( \frac{b}{2m} \right)t} \cos(\omega' t + \phi)
.\]
Hvor $A$ er den initiale amplitude, $b$ er en dæmpningskonstant, $m$ er massen af oscillatoren, $\omega'$ er vinkelfrekvensen af den dæmpede oscillation og $\phi$ er faseforskydningen


\subsubsection{Vinkelfrekvens af en dæmpet oscillation} \label{afs:vinfrekoscdæmp}
Vinkelfrekvensen af en dæmpet oscillation $\omega'$ kan findes som
\[ 
\omega' = \sqrt{\frac{k}{m}- \frac{b^2}{4m^2}}
.\]
Hvor $k$ er kraftkonstanten for den restaurerende kraft, $m$ er massen og $b$ er en dæmpningskonstant.


\subsubsection{Amplitude af dæmpet oscillation} \label{afs:ampdæmtid}
Givet en initialamplitude $A_1$, en dæmpningskonstant $b$ og en masse $m$ kan amplituden $A_2$ til tiden $t$ findes som
\[ 
A_2 = A_1 e^{-\frac{b}{2m}t}
.\]


\subsection{Tvungne oscillationer og resonans (14.8)}
En dæmpet oscillator vil over tid stoppe med at bevæge sig. Dette kan forhindres ved at tilføje en periodisk kraft (tænk at du skubber din ven på en gynge 1 gang pr. cyklus), denne kraft kaldes \textit{drivkraften}.

\subsubsection{Amplitude af tvungen oscillation} \label{afs:amptvu}
Amplituden af en tvungen oscillation $A$ kan findes som
\[ 
A = \frac{F_{\text{max}}}{\sqrt{\left( k - m\omega_d^2 \right)^2 + b^2\omega_d^2}}
.\]
Hvor $F_{\text{max}}$ er den største størrelse drivkraften antager, $k$ er kraftkonstanten af den restaurerende kraft, $m$ er massen, $\omega_d$ er den drivende vinkelfrekvens og $b$ er en dæmpningskonstant.
