\documentclass[12pt]{article}
\usepackage[danish]{babel}
\usepackage{amsfonts, amssymb, mathtools, amsthm, amsmath}
\usepackage{graphicx, pgfplots}
\usepackage{url}
\usepackage[dvipsnames]{xcolor}
\usepackage{sagetex}

%loaded last
\usepackage[hidelinks]{hyperref}

\usepackage{siunitx}
  \sisetup{exponent-product = \cdot,
    output-decimal-marker = {,}}

%Giles Castelles incfig
\usepackage{import}
\usepackage{xifthen}
\usepackage{pdfpages}
\usepackage{transparent}

\newcommand{\incfig}[2][1]{%
  \def\svgwidth{#1\columnwidth}
  \import{../figures/}{#2.pdf_tex}
}

\setlength{\parindent}{0in}
\setlength{\oddsidemargin}{0in}
\setlength{\textwidth}{6.5in}
\setlength{\textheight}{8.8in}
\setlength{\topmargin}{0in}
\setlength{\headheight}{18pt}

\usepackage{fancyhdr}
\pagestyle{fancy}

\fancyhead{}
\fancyfoot{}
\fancyfoot[R]{\thepage}
\fancyhead[C]{\leftmark}

\pgfplotsset{compat=newest}

\pgfplotsset{every axis/.append style={
  axis x line=middle,    % put the x axis in the middle
  axis y line=middle,    % put the y axis in the middle
  axis line style={<->,color=black}, % arrows on the axis
}}

\usepackage{thmtools}
\usepackage{tcolorbox}
  \tcbuselibrary{skins, breakable}
  \tcbset{
    space to upper=1em,
    space to lower=1em,
  }

\theoremstyle{definition}

\newtcolorbox[auto counter]{definition}[1][]{%
  breakable,
  colframe=ForestGreen,  %frame color
  colback=ForestGreen!5, %background color
  colbacktitle=ForestGreen!25, %background color for title
  coltitle=ForestGreen!70!black,  %title color
  fonttitle=\bfseries\sffamily, %title font
  left=1em,              %space on left side in box,
  enhanced,              %more options
  frame hidden,          %hide frame
  borderline west={2pt}{0pt}{ForestGreen},  %display left line
  title=Definition \thetcbcounter: #1,
}

\newtcolorbox{greenline}{%
  breakable,
  colframe=ForestGreen,  %frame color
  colback=white,          %remove background color
  left=1em,              %space on left side in box
  enhanced,              %more options
  frame hidden,          %hide frame
  borderline west={2pt}{0pt}{ForestGreen},  %display left line
}

\newtcolorbox[auto counter, number within=section]{eks}[1][]{%
  brekable,
  colframe=NavyBlue,  %frame color
  colback=NavyBlue!5, %background color
  colbacktitle=NavyBlue!25,    %background color for title
  coltitle=NavyBlue!70!black,  %title color
  fonttitle=\bfseries\sffamily, %title font
  left=1em,            %space on left side in box,
  enhanced,            %more options
  frame hidden,        %hide frame
  borderline west={2pt}{0pt}{NavyBlue},  %display left line
  title=Eksempel \thetcbcounter: #1
}

\newtcolorbox{blueline}{%
  breakable,
  colframe=NavyBlue,     %frame color
  colback=white,         %remove background
  left=1em,              %space on left side in box,
  enhanced,              %more options
  frame hidden,          %hide frame
  borderline west={2pt}{0pt}{NavyBlue},  %display left line
}

\newtcolorbox{teo}[1][]{%
  breakable,
  colframe=RawSienna,  %frame color
  colback=RawSienna!5, %background color
  colbacktitle=RawSienna!25,    %background color for title
  coltitle=RawSienna!70!black,  %title color
  fonttitle=\bfseries\sffamily, %title font
  left=1em,              %space on left side in box,
  enhanced,              %more options
  frame hidden,          %hide frame
  borderline west={2pt}{0pt}{RawSienna},  %display left line
  title=Teori: #1,
}

\newtcolorbox[auto counter, number within=section]{sæt}[1][]{%
  breakable,
  colframe=RawSienna,  %frame color
  colback=RawSienna!5, %background color
  colbacktitle=RawSienna!25,    %background color for title
  coltitle=RawSienna!70!black,  %title color
  fonttitle=\bfseries\sffamily, %title font
  left=1em,              %space on left side in box,
  enhanced,              %more options
  frame hidden,          %hide frame
  borderline west={2pt}{0pt}{RawSienna},  %display left line
  title=Sætning \thetcbcounter: #1,
  before lower={\textbf{Bevis:}\par\vspace{0.5em}},
  colbacklower=RawSienna!25,
}

\newtcolorbox{redline}{%
  breakable,
  colframe=RawSienna,  %frame color
  colback=white,       %Remove background color
  left=1em,            %space on left side in box,
  enhanced,            %more options
  frame hidden,        %hide frame
  borderline west={2pt}{0pt}{RawSienna},  %display left line
}

\newtcolorbox{for}[1][]{%
  breakable,
  colframe=NavyBlue,  %frame color
  colback=NavyBlue!5, %background color
  colbacktitle=NavyBlue!25,    %background color for title
  coltitle=NavyBlue!70!black,  %title color
  fonttitle=\bfseries\sffamily, %title font
  left=1em,              %space on left side in box,
  enhanced,              %more options
  frame hidden,          %hide frame
  borderline west={2pt}{0pt}{NavyBlue},  %display left line
  title=Forklaring #1,
}

\newtcolorbox{bem}{%
  breakable,
  colframe=NavyBlue,  %frame color
  colback=NavyBlue!5, %background color
  colbacktitle=NavyBlue!25,    %background color for title
  coltitle=NavyBlue!70!black,  %title color
  fonttitle=\bfseries\sffamily, %title font
  left=1em,              %space on left side in box,
  enhanced,              %more options
  frame hidden,          %hide frame
  borderline west={2pt}{0pt}{NavyBlue},  %display left line
  title=Bemærkning:,
}

\makeatother
\def\@lecture{}%
\newcommand{\lecture}[3]{
  \ifthenelse{\isempty{#3}}{%
    \def\@lecture{Lecture #1}%
  }{%
    \def\@lecture{Lecture #1: #3}%
  }%
  \subsection*{\makebox[\textwidth][l]{\@lecture \hfill \normalfont\small\textsf{#2}}}
}

\makeatletter

\newcommand{\opgave}[1]{%
 \def\@opgave{#1}%
 \subsection*{Opgave #1}
}

\makeatother

%Format lim the same way in intext and in display
\let\svlim\lim\def\lim{\svlim\limits}

% horizontal rule
\newcommand\hr{
\noindent\rule[0.5ex]{\linewidth}{0.5pt}
}

\title{Opgaver til forelæsning 18}
\author{Noah Rahbek Bigum Hansen}
\date{19. November 2024}

\begin{document}

\maketitle

\section*{Opg. 9.13}
Show that the angle $\alpha$ between a plumb line and the direction of the earth's center is well approximated by $\tan \alpha = \frac{R_c \Omega^2 \sin 2\theta}{2g}$, where $g$ is the observed free-fall acceleration and we assume the earth is perfectly spherically symmetric. Estimate the maximum and minimum values of the magnitude of $\alpha$.
\bigbreak
Vi har en effektiv tyngdeacceleration på jorden givet ved
\[ 
g_{eff} = g_0 + \Omega^2R \sin \theta \hat{\rho}
.\]
Denne har en radiel komponent, $g_{rad}$, og en tangentiel komponent, $g_{tan}$, der svarer til hhv.
\[ 
g_{rad} = g_0 - \Omega^2R \sin^2 \theta
\]
og
\[ 
g_{tan} = \Omega^2 R \sin \theta \cos \theta
.\]
Dermed bliver
\begin{align*}
  \tan \alpha &= \frac{g_{tan}}{g_{rad}} \\
  &= \frac{\Omega^2 R \sin \theta \cos \theta}{g_0 - \Omega^2 R \sin^2 \theta} \\
  &\approx \frac{\Omega^2 R \sin \theta \cos\theta}{g_0} \\
  &= \frac{\Omega^2 R}{2g_0} \sin 2\theta
.\end{align*}
Hvilket skulle vises. $\alpha$ har et minimum for $\sin 2\theta = 0 \implies\theta = 0$, hvor $\alpha = 0$. $\alpha$'s maximum er ved $\sin 2 \theta = 1 \implies \theta  = \frac{\pi}{4} = \ang{45}$, hvo $\alpha = \tan^{-1} \frac{\Omega^2 R}{2g_0}$


\section*{Opg. 9.25}
A high-speed train is travelling at a constant \qty{150}{m \per s} (about \qty{300}{mph}) on a straight, horizontal track across the South Pole. Find the angle between a plumb line suspended from the ceiling inside the train and another inside a hut on the ground. In what direction is the plumb line on the train deflected.
\bigbreak
Vi har $\dot{y} = v_{y0}$ og $\theta = \pi$, da vi er ved sydpolen. Dermed bliver bevægelsesligningerne i alle retninger (9.53) i notatet
\begin{align*}
  \ddot{x} &= 2\Omega v_{y0} \cos \theta = -2\Omega v_{y0}\\
  \ddot{y} &= 0 \\
  \ddot{z} &= -g
.\end{align*}
Da bliver $\tan(\alpha)$
\[ 
\tan \alpha = \frac{2\Omega v_{y0}}{g}
.\]
Denne vinkel, $\alpha$, er vinklen som loddet på toget laver til et tilsvarende lod i et stillestående system (huset). Loddet i toget bliver peger desuden lidt mod ``højre'' grundet corioliskraften pr. højrehåndreglen.



\section*{Opg. 9.26}
In Section 9.8, we used a method of successive approximations to find the orbit of an object that is dropped from rest, correct to first order in the earth's angular velocity $\Omega$. Show in the same way that if an object is thrown with initial velocity, $v_0$ from a point $O$ on the earth's surface at colatitude $\theta$, that to first order in $\Omega$ its orbit is
\begin{align*}
  x &= v_{x0}t + \Omega(v_{y0} \cos \theta - v_{z0} \sin\theta)t^2 + \frac{1}{3} \Omega gt^3 \sin \theta \\
  y &= v_{y0}t - \Omega(v_{x0} \cos \theta)t^2 \\
  z &= v_{z0}t - \frac{1}{2}gt^2 + \Omega(v_{x0} \sin \theta)t^2
.\end{align*}
[First solve the equations of motion in zeroth order, that is, ignoring $\Omega$ entirely. Substitute your zeroth-order solution for $\dot{x}$, $\dot{y}$, and $\dot{z}$ into the right side of the equations above and integrate to give the next approximation. Assume that $v_0$ is small enough that air resistance is negligible and that $g$ is a constant throughout the flight.]
\bigbreak
Givet $\dot{r} = (\dot{x}, \dot{y}, \dot{z}) = (v_{x0}, v_{y0}, v_{z0})$ har vi
\begin{align*}
  \ddot{x} &= 2\Omega(\dot{y} \cos \theta - \dot{z} \sin \theta) \\
  \ddot{y} &= -2\Omega \dot{x} \cos \theta \\
  \ddot{z} &= -g + 2\Omega \dot{x} \sin \theta
.\end{align*}
0.-ordens approksimationen bliver da
\begin{align*}
  \ddot{x} &= 0, &\quad \ddot{y} &= 0, &\quad \ddot{z} &= -g \\
  && &\Downarrow && \\
  \dot{x} &= v_{x0}, &\quad \ddot{y} &= v_{y0}, &\quad \ddot{z} &= -gt + v_{z0} \\
  && &\Downarrow && \\
  x &= v_{x0}t, &\quad y &= v_{y0}t, &\quad z &= h - gt^2 + v_{z0}t
.\end{align*}
Disse kan indsættes i vores bevægelsesligninger som
\begin{align*}
  \ddot{x} &= 2\Omega(v_{y0} \cos \theta - (v_{z0} - gt) \sin \theta) \\
  \ddot{y} &= -2\Omega v_{x0} \cos \theta \\
  \ddot{z} &= -g + 2\Omega v_{x0} \sin\theta \\
  \\
           &\Downarrow \\
  \\
  \dot{x} &= 2\Omega (v_{y0}t \cos \theta - (v_{z0}t - \frac{1}{2}gt^2)\sin\theta) + v_{x0}\\
  \dot{y} &= -2\Omega v_{x0}t \cos \theta + v_{y0}\\
  \dot{z} &= -gt + 2\Omega v_{x0}t \sin\theta + v_{z0}\\
  \\
          &\Downarrow \\
  \\
  x &= v_{x0}t + 2\Omega \left( \frac{1}{2}t^2 v_{y0} \cos\theta - \left( \frac{1}{2}v_{z0}t^2 - \frac{1}{6}gt^3 \right) \sin \theta \right) \\
    &= v_{x0}t + \Omega \left( t^2 \left( v_{y0}\cos\theta - v_{z0} \sin \theta \right) + \frac{1}{3} gt^3 \sin\theta \right) \\
  y &= v_{y0}t -\Omega v_{x0} t^2 \cos \theta \\
  z &= v_{z0}t -\frac{1}{2}gt^2 + \Omega v_{x0}t^2 \sin \theta
.\end{align*}
Hvilket skulle vises.
\end{document}
