\documentclass[12pt]{article}

\usepackage[utf8]{inputenc}
\usepackage[danish]{babel}
\usepackage[output-decimal-marker={,}]{siunitx}
\usepackage{latexsym, amsfonts, amssymb, amsthm, amsmath, graphicx, pgfplots}

\sisetup{exponent-product = \cdot}

\setlength{\parindent}{0in}
\setlength{\oddsidemargin}{0in}
\setlength{\textwidth}{6.5in}
\setlength{\textheight}{8.8in}
\setlength{\topmargin}{0in}
\setlength{\headheight}{18pt}

\pgfplotsset{compat=newest}

\pgfplotsset{every axis/.append style={
  axis x line=middle,    % put the x axis in the middle
  axis y line=middle,    % put the y axis in the middle
  axis line style={<->,color=black}, % arrows on the axis
}}

\title{Opgaver til forelæsning 7}
\author{Noah Rahbek Bigum Hansen}
\date{3. oktober 2024}

\begin{document}
\section*{Opg. 4.7}
A \qty{68,5}{kg} skater moving initially at \qty{2,40}{m/s} on rough horizontal ice comes to rest uniformly in \qty{3,52}{s} due to friction from the ice.
What force does friction exert on the skater?
\bigbreak
Det antages at skøjteløberen decelerer med konstant acceleration, hvilket ikke er usandsynligt idet friktion ikke afhænger af hastighed ($ \, \forall \, v>0$). Altså har vi at den gennemsnitlige acceleration for skøjteløberen er
\[
  a = \frac{v}{t} = \frac{\qty{2,40}{\frac{m}{s}}}{\qty{3,52}{s}} = \qty{0,6818}{\frac{m}{s^2}}
.\] 
Fra Newtons 2. lov har vi at
\[
  F = ma \implies F = \qty{68,5}{kg}\cdot \qty{0,6818}{\frac{m}{s^2}} = \qty{46,7}{n}
.\] 

\section*{Opg. 4.8}
You walk into an elevator, step onto a scale, and push the “up” button. You recall that your normal weight is \qty{625}{N}. Draw a free-body diagram.

\subsection*{(a)}
When the elevator has an upward acceleration of magnitude \qty{2,50}{m/s^2}, what does the scale read?
\bigbreak
Først omregnes den målte vægt fra sidste opgave til en masse vha. Newtons 2.
\[
  m = \frac{F}{a} = \frac{\qty{625}{N}}{9,81 \frac{m}{s^2}} = \qty{63,71}{kg}
.\] 
En vægt måler normalkraft og ikke den egentligt vægt. Det må desuden gælde at den samlede acceleration tilsvarer summen af tyngdeaccelerationen og elevatorens acceleration
\[
a = g + \qty{2,50}{\frac{m}{s^2}} = \qty{12,31}{\frac{m}{s^2}}
.\] 
Derfor må den nye oplevede kraft være
\[
F = ma \implies F = \qty{63,71}{kg}\cdot \qty{12,31}{\frac{m}{s^2}} = \qty{784}{N}
.\] 
Den viste vægt er derfor:
\[
m_{v} = \frac{\qty{784}{N}}{g} = \qty{79,9}{kg}
.\] 

\subsection*{(b)}
If you hold a \qty{3,85}{kg} package by a light vertical string, what will be the tension in this string when the elevator accelerates as in part \textbf{(a)}?
\bigbreak
Samme formel som ovenfor benyttes dog med en ny masse
\[
F = \qty{3,85}{kg}\cdot \qty{12,31}{\frac{m}{s^2}} = \qty{47,4}{N}
.\] 

\section*{Opg. 4.10}
A dockworker applies a constant horizontal force of \qty{80,0}{N} to a block of ice on a smooth horizontal floor. The frictional force is negligible. The block starts from rest and moves \qty{11,0}{m} in \qty{5,00}{s}.


\subsection*{(a)}
What is the mass of the block of ice?
\bigbreak
Det antages at blokken accelerer med konstant acceleration. Altså kan formlen for strækning ved konstant acceleration benyttes
\[
s = \frac{1}{2}at^2 \implies a = \frac{2s}{t^2} = \frac{2\cdot \qty{11,0}{m}}{(\qty{5,00}{s})^2} = \qty{0,88}{\frac{m}{s^2}}
.\] 
Dernæst benyttes Newtons 2. lov
\[
m = \frac{F}{a} = \frac{\qty{80,0}{N}}{\qty{0,88}{\frac{m}{s^2}}} = \qty{90,9}{kg}
.\] 

\subsection*{(b)}
If the worker stops pushing at the end of \qty{5,00}{s}, how far does the block move in the next \qty{5,00}{s}?
\bigbreak
Den konstante acceleration på \qty{0,88}{m/s^2} vil efter 5 sekunder resultere i en hastighed på
\[
v = a\cdot t = \qty{0,88}{\frac{m}{s^2}}\cdot \qty{5}{s} = \qty{4,40}{\frac{m}{s}}
.\] 
Uden friktion til hastigheden ikke ændre sig over de næste 5 sekunder og derfor vil hastigheden forblive konstant dermed bliver den samlede tilbagelagte strækning i perioden
\[
s_{5-10} = vt \implies s_{5-10} = \qty{4,40}{\frac{m}{s}}\cdot \qty{5,00}{s} = \qty{22,2}{m}
.\] 

\section*{Opg. 4.16}
An astronaut’s pack weighs \qty{17,5}{N} when she is on the earth but only \qty{3,24}{N} when she is at the surface of a moon.


\subsection*{(a)}
What is the acceleration due to gravity on this moon?
\bigbreak
Idet tyngdekraften er lineært afhængig af tyngdeaccelerationen må det gælde at
\[
\frac{g_{\mathrm{moon}}}{g_{\mathrm{moon} }} = \frac{m_{\mathrm{jord} }}{m_{\mathrm{moon}}} \implies g_{\mathrm{moon}} = \frac{m_{\mathrm{moon}}\cdot g_{\mathrm{jord}}}{m_{\mathrm{jord}}} \implies g_{\mathrm{moon}} = \frac{\qty{3,24}{N}\cdot \qty{9,81}{\frac{m}{s^2}}}{\qty{17,5}{N}} = \qty{1,82}{\frac{m}{s^2}}
.\] 


\subsection*{(b)}
What is the mass of the pack on this moon?
\bigbreak
Massen afhænger ikke af tyngdeaccelerationen men vha. Newtons 2. lov og tyngdeaccelerationen på jorden kan ``packens'' masse findes som
\[
m_{\mathrm{pack}} = \frac{\qty{17,5}{N}}{\qty{1,82}{\frac{m}{s^2}}} = \qty{9,62}{kg}
.\] 
Dette er ``packens'' masse både på månen og på jorden og alle andre steder den kunne finde på at opholde sig. 


\section*{Opg. 4.23}
Boxes $A$ and $B$ are in contact on a horizontal, frictionless surface. Box $A$ has mass \qty{20,0}{kg} and box $B$ has mass \qty{5,0}{kg}. A horizontal force of \qty{250}{N} is exerted on box $A$. What is the magnitude of the force that box $A$ exerts on box $B$?
\bigbreak
Først findes accelerationen som boksene kommer til at få. Dette gøres vha. Newtons 2. idet den samlede masse er givet ved $m = m_a + m_b$
\[
  a = \frac{F}{m_a + m_b} \implies a = \frac{\qty{250}{N}}{\qty{20,0}{kg} + \qty{5,00}{kg}} = \qty{10}{\frac{m}{s^2}}
.\] 
Dette må være den acceleration som begge kasser oplever. Denne acceleration kan dernæst omregnes til en kraft
\[
  F = m_b a \implies F = \qty{5,00}{kg} \cdot \qty{10}{\frac{m}{s^2}} = \qty{50,0}{N}
.\] 



\section*{Opg. 4.27}
Crates $A$ and $B$ sit at rest side by side on a frictionless horizontal surface. They have masses $m_A$ and $m_B$, respectively. When a horizontal force $ \Vec{F}$ is applied to crate $A$, the two crates move off to the right.


\subsection*{(a)}
Draw clearly labeled free-body diagrams for crate $A$ and for crate $B$. Indicate which pairs of forces, if any, are third-law action–reaction pairs.


\subsection*{(b)}
If the magnitude of $ \Vec{F}$ is less than the total weight of the two crates, will it cause the crates to move? Explain.
\bigbreak
De to kasser vil bevæge sig for enhver horisontal kraft, hvor $|F| > 0$, dette skyldes at der ikke er nogen friktion


\section*{Opg. 4.36}
An advertisement claims that a particular automobile can ``stop on a dime.'' What net force would be necessary to stop a \qty{850}{kg} automobile traveling initially at \qty{45}{km/h} in a distance equal to the diameter of a dime, \qty{1,8}{cm}?
\bigbreak
Idet vi antager at bilen bremser med konstant acceleration kan denne konstante acceleration findes som
\[
a = \frac{v_1^2-v_0^2}{2\left( x_1 - x_0 \right)}
,\] 
hvor subskriptet 0 angiver en starttilstand og subskriptet 1 angiver en sluttilstand. Altså har vi
\[
a = \frac{0-(\qty{45}{\frac{km}{h}})^2}{2\left( \qty{1,8}{cm}-0 \right) } = \frac{-(\qty{45}{\frac{km}{h}})^2}{\qty{3,6}{cm}} = \qty{-4340}{\frac{m}{s^2}}
.\] 
Denne acceleration kan dernæst omregnes til en tilsvarende kraft vha. Newtons 2. lov
\[
F = ma \implies F = \qty{850}{kg}\cdot \qty{-4340}{\frac{m}{s^2}} = \qty{-3,689e6}{N}
.\] 

\section*{Opg. 4.37}
Two crates, one with mass \qty{4,00}{kg} and the other with mass \qty{6,00}{\kg}, sit on the frictionless surface of a frozen pond, connected by a light rope. A woman wearing golf shoes (for traction) pulls horizontally on the \qty{6,00}{\kg} crate with a force $F$ that gives the crate an acceleration of \qty{2,50}{m/s^2}.


\subsection*{(a)}
What is the acceleration of the \qty{4,00}{kg} crate?
\bigbreak
Idet begge kasser er forbundet med en snor må de begge accelerere med samme hastighed og derfor må den lille kasses acceleration ligeledes være \qty{2,50}{m/s^2}

\subsection*{(b)}
Draw a free-body diagram for the \qty{4,00}{kg} crate. Use that diagram and Newton’s second law to find the tension $T$ in the rope that connects the two crates.
\bigbreak
Spændingen i rebet må tilsvare den kraft som kassen på \qty{4,00}{kg} hiver i rebet med. Altså har vi
\[
F = m_b a = \qty{4,00}{kg}\cdot \qty{2,50}{\frac{m}{s^2}} = \qty{10,0}{N}
.\] 

\subsection*{(c)}
Draw a free-body diagram for the \qty{6,00}{kg} crate. What is the direction of the net force on the \qty{6,00}{kg} crate? Which is larger in magnitude, $T$ or $F$?
\bigbreak
Den samlede kraft som kassen på \qty{6,00}{kg} oplever må svare til differensen mellem den kraft som damen hiver i kassen med og snorkraften som kassen på \qty{4,00}{kg} hiver med. Altså må den samlede kraft oplevet af kassen være $F_{res} = F-T$. Altså har vi at
\[
F-T = m_a a \implies F = T+m_a a \implies F = \qty{10,0}{N} + \qty{6,00}{kg}\cdot \qty{2,50}{\frac{m}{s^2}} = \qty{25}{N}
.\] 
Altså er kraften F størst, dvs. den resulterende krafts retning er i retningen fra kassen mod damen.


\subsection*{(d)}
Use part (c) and Newton’s second law to calculate the magnitude of $F$.
\bigbreak
Løst ovenfor

\section*{Opg. 4.38}
Two blocks connected by a light horizontal rope sit at rest on a horizontal, frictionless surface. Block $A$ has mass \qty{15,0}{kg}, and
block $B$ has mass $m$. A constant horizontal force $F =$ \qty{60,0}{N} is applied to block $A$. In the first \qty{5,00}{s} after the force is applied, block $A$ moves \qty{18,0}{m} to the right.

\subsection*{(a)}
While the blocks are moving, what is the tension $T$ in the rope that connects the two blocks?
\bigbreak
Idet den tilførte kraft er konstant må kassernes acceleration ligeledes være konstant. Altså benyttes formlen for strækning ved konstant acceleration
\[
s = \frac{1}{2}at^2 \implies a = \frac{2s}{t^2} \implies a = \frac{2\cdot \qty{18,0}{m}}{\left( \qty{5,00}{s} \right)^2} = \qty{1,44}{\frac{m}{s^2}}
.\] 
Dermed er kassernes acceleration beregnet. Som vist i opgaven ovenfor kan spændingen i snoren findes som
\[
T = F-ma \implies T = \qty{60,0}{N} - \qty{15,0}{kg} \cdot \qty{1,44}{\frac{m}{s^2}} = \qty{38,4}{N}
.\] 

\subsection*{(b)}
What is the mass of block $B$?
\bigbreak
Massen af den lille blok kan nu beregnes idet vi kender spændingen i rebet (bemærk at fremgangsmåden her er den samme som i opgave 4.37 dog omvendt)
\[
T = m_a a \implies m_a = \frac{T}{a} \implies m_a = \frac{\qty{38,4}{N}}{\qty{1,44}{\frac{m}{s^2}}} = \qty{26,7}{kg} 
.\] 
  

\section*{Opg. 4.45}
Boxes $A$ and $B$ are connected to each end of a light vertical rope. A constant upward force $F =$  \qty{80,0}{N} is applied to box $A$. Starting from rest, box $B$ descends \qty{12,0}{m} in \qty{4,00}{s}. The tension in the rope connecting the two boxes is \qty{36,0}{N}. What are the masses of:


\subsection*{(a)}
Box $A$?
\bigbreak
Først regnes accelerationen af de to kasser med samme formel som opg. 4.37
\[
  a = \frac{2s}{t^2} \implies a = \frac{2\cdot \qty{12,0}{m}}{(\qty{4,00}{s})^2} = \qty{1,50}{\frac{m}{s^2}}
.\] 
Den samlede kraft på boks A må være givet ved $F-T-m_Ag$. Altså har vi at
 \[
   F-T-m_ag = -m_a a \implies m_a = \frac{F-T}{g-a} \implies m_a = \frac{\qty{80,0}{N}-\qty{36,0}{N}}{\qty{9,81}{\frac{m}{s^2}}-\qty{1,50}{\frac{m}{s^2}}} = \qty{5,30}{kg}
.\] 

 \subsection*{(b)}
Box $B$?
\bigbreak
Samme formel som ovenfor kan benyttes her. Altså har vi at
\[
T - m_b g = m_b a \implies m_b = \frac{T}{g-a} \implies m_b = \frac{\qty{36,0}{N}}{\qty{9,81}{\frac{m}{s^2}}-\qty{1,50}{\frac{m}{s^2}}} = \qty{4,34}{kg}
.\] 

\end{document}
