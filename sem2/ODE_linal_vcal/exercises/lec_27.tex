\section*{Lecture 27}

\subsection*{1.} Consider the flow of a fluid with density $\rho \left( \Vec{r} \right) = 1$ and velocity
\[ 
v \left( \Vec{r} \right) = \left( 0,y,0 \right) 
.\]
Let $T$ be the region given by all points $P (x,y,z)$ such that
\[ 
x+y+2z \leq 1, \qquad 0\leq x, \qquad 0 \leq y, \qquad 0 \leq z
.\]
Let $S$ be the boundary surface of $T$.

\paragraph{(a)} Sketch the region $T$.

\paragraph{(b)} Find a parametric representation of $S$ such that the unit surface normal vector is pointing outwards of $T$.

\paragraph{(c)} Calculate the flux of the fluid through $S$ (to the outside of $T$) using surface integrals.

\paragraph{(d)} Calculate the flux of the fluid through $S$ (to the outside of $T$) using a triple integral, i.e. Gauss' divergence theorem.
