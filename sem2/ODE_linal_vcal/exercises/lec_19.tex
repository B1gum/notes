\section*{Lecture 19}

\subsection*{1.} Calculate the Laplace transform of the sine hyperbolic
\[ 
\sinh(at) = \frac{1}{2}\left( e^{at}- e^{-at} \right) 
.\]
Here, $a > 0$ is a constant.
\bigbreak
From the lecture we know that:
\[ 
\mathcal{L}\left( e^{at} \right) = \frac{1}{s-a}
.\]
From linearity of the Laplace transform we have that:
\[ 
\mathcal{L}\left( \sinh \left( at \right)  \right) = \mathcal{L}\left( \frac{1}{2}\left( e^{at} - e^{-at} \right)  \right) = \frac{1}{2} \mathcal{L} \left( e^{at} \right) - \frac{1}{2}\mathcal{L} \left( e^{-at} \right) = \frac{1}{2} \left( \frac{1}{s-a} - \frac{1}{s+a} \right) = \frac{1}{2} \cdot \frac{2a}{s^2 - a^2} = \frac{a}{s^2 - a^2}
.\]
Which holds for $\mathrm{Re}(s) > a$.


\subsection*{2.} Calculate the Laplace transform of $f(t)$ and $f'(t)$, where
\[ 
f(t) = e^{-t}t
.\]
\bigbreak
For $f(t)$ we start by splitting the expression into separate functions as $g(t) = t$ and $f(t) = e^{-t}$. From the lecture we have that:
\[ 
G(s) = \mathcal{L}(g(t)) = \frac{1}{s^2}
.\]
We also know that:
\[ 
\mathrm{If } \mathcal{L}\left( g(t) \right) = G(s), \mathrm{ then } \mathcal{L}\left( e^{at} g(t) \right) = G (s-a)
.\]
This is called s-shifting and in our case $a = -1$. Therefore we get:
\[ 
\mathcal{L}\left( f(t)g(t) \right) = G(s+1) = \frac{1}{\left( s+1 \right)^2}
.\]


\subsection*{3.} Calculate the Laplace transform of
\[ 
g(t) = \int_{0}^{t} \tau e^{-\tau} \, \mathrm{d}\tau
.\]
\bigbreak
The Laplace transform of integrals gives us:
\[ 
\mathcal{L}(g(t)) = \frac{1}{s} \mathcal{L}\left( te^{-t} \right) 
.\]
In the last exercise it was shown that
\[ 
\mathcal{L}\left( te^{-t} \right) = \frac{1}{\left( s+1 \right)^2}
.\]
Therefore we have that:
\[ 
\mathcal{L}\left( g(t) \right) = \frac{1}{s} \mathcal{L}\left( te^{-t} \right) = \frac{1}{s} \cdot \frac{1}{\left( s+1 \right)^2} = \frac{1}{s \left( s+1 \right)^2}
.\]


\subsection*{4.} Find a formula for the Laplace transform of $f'''(t)$ in terms of the Laplace transform of $f(t)$, assuming the Laplace transform exist whenever necessary.
\bigbreak
The Laplace transform of a derivative is:
\[ 
\mathcal{L}\left( f'(t) \right) = s \mathcal{L}\left( f(t) \right) - f_0
.\]
Therefore we can start by examining $f'''(t)$ as:
\begin{align*}
  \mathcal{L}\left( f'''(t) \right) &= s \mathcal{L}\left( f''(t) \right) - f''(0) \\
  &= s^2 \mathcal{L}\left( f'(t) \right) - sf'(0)- f''(0) \\
  &= s^3 \mathcal{L}(f(t)) - s^2 f(0) - s f'(0) - f''(0)
.\end{align*}



\subsection*{5.} Calculate the inverse Laplace transform of
\[ 
F(s) = \frac{s^2 + 2s + 1}{s \left( s^2 + 1 \right) }
.\]
\bigbreak
We start by splitting the fraction up into terms as:
\[ 
  F(s) = \frac{s^2}{s \left( s^2 + 1 \right) } + \frac{2s}{s \left( s^2 + 1 \right) } + \frac{1}{s \left( s^2 + 1 \right) } = \frac{s}{s^2 + 1} + \frac{2}{s^2 + 1} + \frac{1}{s\left( s^2 + 1 \right) }
.\]
We know that:
\begin{align*}
  \mathcal{L}^{-1}\left( \frac{1}{s^2 + 1} \right) &= \sin t \\
  \mathcal{L}^{-1}\left( \frac{s}{s^2 + 1} \right) &= \cos t
.\end{align*}
For the last term we will utilize partial fraction decomposition. We have that:
\[ 
\frac{1}{s \left( s^2 + 1 \right) } = \frac{A}{s} + \frac{B s  + C}{s^2 + 1}
.\]
We multiply both sides by the common denominator to get:
\[ 
1 = A \left( s^2 + 1 \right) + s \left( Bs + C \right) = As^2 + A + Bs^2 + Cs = \left( A+B \right) s^2 + Cs + A
.\]
For the coefficients to match we have:
\begin{align*}
  A + B &= 0 \\
  C &= 0 \\
  A &= 1 \\
  \implies B &= -1
.\end{align*}
Therefore the decomposed fraction becomes:
\[ 
\frac{1}{s \left( s^2 + 1 \right) } = \frac{1}{s} - \frac{s}{s^2 + 1}
.\]
We have that:
\begin{align*}
  \mathcal{L}^{-1} \left( \frac{1}{s} \right) &= 1 \\
  \mathcal{L}^{-1} \left( \frac{s}{s^2 + 1} \right) &= \cos t
.\end{align*}
From the linearity of Laplace transforms we have that:
\[ 
\mathcal{L}^{-1} \left( \frac{1}{s \left( s^2 + 1 \right) } \right) = \mathcal{L}^{-1} \left( \frac{1}{s} \right) - \mathcal{L}^{-1} \left( \frac{s}{s^2 + 1} \right) = 1 - \cos t
.\]
The total inverse Laplace transform therefore is:
\[ 
\mathcal{L}^{-1}\left( F(s) \right) = \mathcal{L}^{-1}\left( \frac{s}{s^2 + 1} \right) + 2\mathcal{L}^{-1} \left( \frac{1}{s^2 + 1} \right) + \mathcal{L}^{-1}\left( \frac{1}{s \left( s^2 + 1 \right) } \right) = \cos t + 2 \sin t + 1 - \cos t = 2 \sin t + 1
.\]


\subsection*{6.} Solve the initial value problem
\[ 
y'(t) = 2y(t), y(0) = 1
\]
using the theory of Laplace transform.
\bigbreak
We start by applying the Laplace transformation to both sides to get:
\begin{align*}
  \mathcal{L}\left( y'(t) \right) &= \mathcal{L}\left( 2y(t) \right)  \\
  s Y(s) - y(0) &= 2Y(s) \\
  s Y(s) - 1 &= 2Y(s) \\
  \left( s-2 \right) Y(s) &= 1 \\
  Y(s) &= \frac{1}{s - 2} \\
  &=  e^{2t}
.\end{align*}



\subsection*{7.} Solve the initial value problem
\[ 
y''(t) - y'(t) = t, y(0) = 1, y'(0) = 0
\]
using the theory of Laplace transform.

\textit{Hint}:
\[ 
\mathcal{L}^{-1} \left( \frac{1}{s^3 \left( s-1 \right) } \right) = e^{t} - \frac{1}{2}t^2 - t - 1
.\]
\bigbreak
We apply the Laplace transform to both sides to get:
\begin{align*}
  \mathcal{L}\left( y''(t) \right) - \mathcal{L}\left( y'(t) \right) &= \mathcal{L}\left( t \right)  \\
  s^2 Y(s) - sy(0) - y'(0) - s Y(s) - y(0) &= \frac{1}{s^2} \\
  \left( s^2 - s \right) Y(s) - s + 1 &= \frac{1}{s^2} \\
  \left( s^2 - s \right) Y(s) &= \frac{1}{s^2} + s - 1 \\
  Y(s) &= \frac{s^3 - s^2 + 1}{s^2} \cdot \frac{1}{s \left( s - 1 \right) } \\
  Y(s) &= \frac{s^3 - s^2 + 1}{s^3 \left( s-1 \right) }  \\
  Y(s) &= \frac{1}{s^3 \left( s-1 \right) } + \frac{s^3 - s^2}{s^3 \left( s-1 \right) } \\
  Y(s) &= \frac{1}{s^3 \left( s-1 \right) } + \frac{s^2 \left( s-1 \right) }{s^3 \left( s-1 \right) } \\
  Y(s) &= \frac{1}{s^3 \left( s-1 \right) } + \frac{1}{s}
.\end{align*}
We have been told that:
\[ 
\mathcal{L}^{-1} \left( \frac{1}{s^3 \left( s-1 \right) } \right) = e^{t} - \frac{1}{2}t^2 - t - 1
.\]
And we also know that
\[ 
\mathcal{L}^{-1} \left( \frac{1}{s} \right) =  1
.\]
By the linearity of inverse Laplace transforms we get:
\[ 
\mathcal{L}^{-1}\left( \frac{1}{s^3 \left( s-1 \right) } \right) + \mathcal{L}^{-1} \left( \frac{1}{s} \right) = e^{t} - \frac{1}{2}t^2 - t - 1 + 1 = e^{t} - \frac{1}{2}t^2 - t
.\]

