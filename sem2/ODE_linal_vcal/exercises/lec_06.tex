\section*{Lecture 6}

\subsection*{1.} Find the eigenvalues and all eigenvectors of
\[ 
A = \begin{pmatrix}
1 & 1\\
1 & 1\\
\end{pmatrix}
.\]
\bigbreak
To find the eigenvalues $\lambda$ we use the relation
\[ 
\mathrm{det}(A-\lambda I) = 0
.\]
This gives
\[ 
\left| \begin{array}{cc}
1 - \lambda & 1\\
1 & 1 -\lambda\\
\end{array} \right| = 0 \implies (1-\lambda)^2 - 1 = 0 \implies \lambda = {2,0}
.\]
We get that the eigenvalues of $A$ is $\lambda_1 = 2$ and $\lambda_2 = 0$. We start with determining the eigenvectors for $\lambda_1 =2$. To find the eigenvectors we start with the relation
\[ 
  A \Vec{x}^{(1)} = \lambda_1 \Vec{x}^{(1)} = 2 \Vec{x}^{(1)} \implies (A - 2I) \Vec{x}^{(1)} = \Vec{0}
.\]
This gives
\begin{align*}
  \begin{pmatrix}
  -1 & 1\\
  1 & -1\\
  \end{pmatrix} \begin{pmatrix}
  x_1^{(1)}\\
  x_2^{(1)}\\
  \end{pmatrix} = \begin{pmatrix}
  0\\
  0\\
  \end{pmatrix}
.\end{align*}
This gives the system
\begin{align*}
  -x_1^{(1)} + x_2^{(1)} &= 0 \\
  x_1^{(1)} - x_2^{(1)} &= 0 \\
.\end{align*}
Both equations imply $x_1^{(1)} = x_2^{(1)}$. Therefore
\[ 
\Vec{x}^{(1)} = c_1 \begin{pmatrix}
1\\
1\\
\end{pmatrix}, \quad c_1 \neq 0
.\]
For $\lambda_2 = 0$ we get
\[ 
A \Vec{x}^{(2)} = \lambda_2 \Vec{x}^{(2)} = \Vec{0} \implies A \Vec{x}^{(2)} = 0
.\]
That is
\[ 
\begin{pmatrix}
1 & 1\\
1 & 1\\
\end{pmatrix} \begin{pmatrix}
x_1^{(2)}\\
x_2^{(2)}\\
\end{pmatrix} = \begin{pmatrix}
0\\
0\\
\end{pmatrix}
.\]
This corresponds to the system
\begin{align*}
  x_1^{(2)} + x_2^{(2)} &= 0 \\
  x_1^{(2)} + x_2^{(2)} &= 0
.\end{align*}
Which implies that $x_1^{(2)} = -x_2^{(2)}$. Therefore
\[ 
\Vec{x}^{(2)} = c_2 \begin{pmatrix}
1\\
-1\\
\end{pmatrix}, \quad c_2 \neq 0
.\]


\subsection*{2.} Find the eigenvalues and all eigenvectors of
\[ 
A = \begin{pmatrix}
1 & 0\\
1 & 0\\
\end{pmatrix}
.\]
\bigbreak
We once again start with the relation
\[ 
\mathrm{det}(A - \lambda I) = 0
.\]
Which gives
\[ 
\begin{pmatrix}
1 - \lambda & 0\\
1 & -\lambda\\
\end{pmatrix} = (1 - \lambda)(-\lambda) + 0 \implies \lambda_1 = 1, \lambda_2 = 0
.\]
For $\lambda_1 = 1$ we get
\[ 
A \Vec{x}^{(1)} = \lambda_1 \Vec{x}^{(1)} = \Vec{x}^{(1)} \implies (A - I) \Vec{x}^{(1)} = \Vec{0}
.\]
This gives
\[ 
\begin{pmatrix}
0 & 0\\
1 & -1\\
\end{pmatrix} \begin{pmatrix}
x_1^{(1)}\\
x_2^{(2)}\\
\end{pmatrix} = \begin{pmatrix}
0\\
0\\
\end{pmatrix}
.\]
Which corresponds to the system
\begin{align*}
  0 &= 0 \\
  x_1^{(1)} - x_2^{(2)} &= 0
.\end{align*}
This implies $x_1^{(1)} = x_2^{(1)}$. Therefore
\[ 
\Vec{x}^{(1)} = c_1 \begin{pmatrix}
1\\
1\\
\end{pmatrix}, \quad c_1 \neq 0
.\]
For $\lambda_2 = 0$ we get
\[ 
A \Vec{x}^{(2)} = \lambda_2 \Vec{x}(2) = 0 \implies A \Vec{x}^{(2)} = 0
.\]
This gives
\[ 
\begin{pmatrix}
1 & 0\\
1 & 0\\
\end{pmatrix} \begin{pmatrix}
x_1^{(2)}\\
x_2^{(2)}\\
\end{pmatrix} = \begin{pmatrix}
0\\
0\\
\end{pmatrix}
.\]
Which corresponds to the system
\begin{align*}
  x_1^{(2)} &= 0 \\
  x_1^{(2)} &= 0
.\end{align*}
Therefore
\[ 
\Vec{x}^{(2)} = c_2 \begin{pmatrix}
0\\
1\\
\end{pmatrix}, \quad c_2 \neq 0
.\]


\subsection*{3.} Find the eigenvalues and all eigenvectors of
\[ 
A = \begin{pmatrix}
0 & -1\\
1 & 0\\
\end{pmatrix}
.\]
\bigbreak
Once again, we start with
\[ 
\mathrm{det}(A-\lambda I) = 0
.\]
Which gives
\[ 
\begin{pmatrix}
-\lambda & -1\\
1 & -\lambda\\
\end{pmatrix} = \lambda^2 + 1 = 0 \implies \lambda_1 = i, \lambda_2 = -i
.\]
For $\lambda_1 = i$ we get
\[ 
A \Vec{x}^{(1)} = i \Vec{x}^{(1)} \implies (A - iI)\Vec{x}^{(1)} = \Vec{0}
.\]
This gives
\[ 
\begin{pmatrix}
-i & -1\\
1 & -i\\
\end{pmatrix} \begin{pmatrix}
x_1^{(1)}\\
x_2^{(1)}\\
\end{pmatrix} = \begin{pmatrix}
0\\
0\\
\end{pmatrix}
.\]
Which corresponds to the system
\begin{align*}
  -i x_1^{(1)} - x_2^{(1)} &= 0 \\
  x_1^{(1)} - ix_2^{(1)} &= 0
.\end{align*}
This can be solved by equating the two equations.
\begin{align*}
  -ix_1^{(1)} - x_2^{(1)} &= x_1^{(1)} - ix_2^{(1)} \\
  x_1^{(1)} + ix_1^{(1)} &= ix_2^{(1)} -x_2^{(1)} \\
  x_1^{(1)}(1 + i) &= x_2^{(1)}(i-1)
.\end{align*}
We can now set $x_2^{(1)} = c_1$, where $c_1 \neq 0$. Doing this makes $x_1^{(1)} = i$. Therefore



\subsection*{4.} Find the eigenvalues and all eigenvectors of
\[ 
A = \begin{pmatrix}
1 & 0\\
0 & 1\\
\end{pmatrix}
.\]
\bigbreak
We start with
\[ 
\mathrm{det}(A - \lambda i) = 0
.\]
Which gives
\[ 
\begin{pmatrix}
1 - \lambda & 0\\
0 & 1 - \lambda\\
\end{pmatrix} = 0 \implies (1-\lambda)^2 = 0 \implies \lambda_1 = 1, \quad \lambda_2 = 1
.\]
For $\lambda = 1$ we get
\[ 
A \Vec{x} = \Vec{x} \implies (A - I)\Vec{x} = \Vec{0}
.\]
This gives
\[ 
\begin{pmatrix}
0 & 0\\
0 & 0\\
\end{pmatrix} \begin{pmatrix}
x_1\\
x_2\\
\end{pmatrix} = \begin{pmatrix}
0\\
0\\
\end{pmatrix}
.\]
Here both $x_1$ and $x_2$ can be chosen to be any value independently of each other and we therefore get
\[ 
\Vec{x} = c_1 \begin{pmatrix}
1\\
0\\
\end{pmatrix} + c_2 \begin{pmatrix}
0\\
1\\
\end{pmatrix}
.\]
Where $c_1 \neq 0$ and $c_2 \neq 0$ are arbitrary constants.


\subsection*{5.} Consider the matrix
\[ 
A = \begin{pmatrix}
1 & 1\\
1 & 1\\
\end{pmatrix}
.\]
Calculate $A^{4}$ using the diagonalization method.
\bigbreak
From exercise 1 we know that $\lambda_1 = 2$ and $\lambda_2 = 0$ whilst we choose the eigenvectors corresponding to $c_1 = c_2 = 1$ as
\[ 
\Vec{x}^{(1)} = \begin{pmatrix}
1\\
1\\
\end{pmatrix} \quad \text{and} \quad \Vec{x}^{(2)} = \begin{pmatrix}
1\\
-1\\
\end{pmatrix}
.\]
From the eigenvalues we can construct the diagonal matrix
\[ 
D = \begin{pmatrix}
\lambda_1 & 0\\
0 & \lambda_2\\
\end{pmatrix} = \begin{pmatrix}
2 & 0\\
0 & 0\\
\end{pmatrix}
.\]
The matrix $X$ can be constructed as
\[ 
X = (\Vec{x}^{(1)} \Vec{x}^{(2)}) = \begin{pmatrix}
1 & 1\\
1 & -1\\
\end{pmatrix}
.\]
We can now find $X^{-1}$ through Gauss-Jordan elimination
\[ 
  (XI) = \left( \begin{array}{cc|cc}
  1 & 1 & 1 & 0\\
  1 & -1 & 0 & 1\\
  \end{array} \right)
.\]
We start by subtracting (1) from (2) to get
\[ 
\left( \begin{array}{cc|cc}
1 & 1 & 1 & 0 \\
0 & -2 & -1 & 1 \\
\end{array} \right)
.\]
We can now do $-\frac{1}{2} \cdot (2)$ to get
\[ 
\left( \begin{array}{cc|cc}
1 & 1 & 1 & 0\\
0 & 1 & \frac{1}{2} & -\frac{1}{2}  \\
\end{array} \right)
.\]
We can now subtract (2) from (1) to get
\[ 
\left( \begin{array}{cc|cc}
1 & 0 & \frac{1}{2} & \frac{1}{2} \\
0 & 1 & \frac{1}{2} & -\frac{1}{2}\\
\end{array} \right)
.\]
Therefore
\[ 
X^{-1} = \begin{pmatrix}
\frac{1}{2} & \frac{1}{2}\\
\frac{1}{2} & -\frac{1}{2}\\
\end{pmatrix}
.\]
We can now find $A^{4}$ as
\begin{align*}
  A^{4} &= XD^{4}X^{-1} \\
  &= \begin{pmatrix}
  1 & 1\\
  1 & -1\\
  \end{pmatrix} \begin{pmatrix}
  2 & 0\\
  0 & 0\\
  \end{pmatrix} \begin{pmatrix}
  \frac{1}{2} & \frac{1}{2}\\
  \frac{1}{2} & -\frac{1}{2}\\
  \end{pmatrix} \\
  &= \begin{pmatrix}
  1 & 1\\
  1 & -1\\
  \end{pmatrix} \begin{pmatrix}
  8 & 8\\
  0 & 0\\
  \end{pmatrix} \\
  &= \begin{pmatrix}
  8 & 8\\
  8 & 8\\
  \end{pmatrix}
.\end{align*}


\subsection*{6.} Consider the matrix
\[ 
A = \begin{pmatrix}
1 & 2 & -2\\
0 & 3 & -1\\
0 & 0 & 2\\
\end{pmatrix}
.\]
Calculate $A^{4}$ using the diagonalization method.
\bigbreak
We start by finding the eigenvalues as
\[ 
\mathrm{det}(A - \lambda I) = \left| \begin{array}{ccc}
1 - \lambda & 2 & -2\\
0 & 3-\lambda & -1\\
0 & 0 & 2 -\lambda\\
\end{array} \right| = 0
.\]
As the above matrix is triangular we can find the determinant as
\[ 
\mathrm{det}(A - \lambda I) = (1-\lambda)(2-\lambda)(3-\lambda) = 0
.\]
Which means the eigenvalues are $\lambda_1 = 1$, $\lambda_2 = 2$ and $\lambda_3 = 3$. This gives the diagonal matrix
\[ 
D = \begin{pmatrix}
1 & 0 & 0\\
0 & 2 & 0\\
0 & 0 & 3\\
\end{pmatrix}
.\]
The eigenvectors for $\lambda_1 = 1$ can be found as
\begin{align*}
  A \Vec{x}^{(1)} &= 1\cdot \Vec{x}^{1} \\
  \implies (A-I) \Vec{x}^{(1)} &= 0 \\
  \implies \begin{pmatrix}
  0 & 2 & -2\\
  0 & 2 & -1\\
  0 & 0 & 1\\
  \end{pmatrix} \begin{pmatrix}
  x_1^{(1)}\\
  x_2^{(1)}\\
  x_3^{(1)}\\
  \end{pmatrix} &= \begin{pmatrix}
  0\\
  0\\
  0\\
  \end{pmatrix}
.\end{align*}
This corresponds to the system of equations
\begin{align*}
  2x_2^{(1)} - 2x_3^{(1)} &= 0 \\
  2x_2^{(1)} - x_3^{(1)} &= 0 \\
  x_3^{(1)} &= 0
.\end{align*}
We quickly see that $x_2^{(1)} = x_3^{(1)} = 0$ and $x_1^{(1)}$ is arbitrary. We choose
\[ 
\Vec{x}^{(1)} = \begin{pmatrix}
1\\
0\\
0\\
\end{pmatrix}
.\]
For $\lambda_2 = 2$ we get
\begin{align*}
  A \Vec{x}^{(2)} &= 2 \Vec{x}^{(2)} \\
  \implies (A - 2I)\Vec{x}^{(2)} &= 0 \\
  \implies \begin{pmatrix}
  -1 & 2 & -2\\
  0 & 1 & -1\\
  0 & 0 & 0\\
  \end{pmatrix} \begin{pmatrix}
  x_1^{(2)}\\
  x_2^{(2)}\\
  x_3^{(2)}\\
  \end{pmatrix} &= \begin{pmatrix}
  0\\
  0\\
  0\\
  \end{pmatrix}
.\end{align*}
This corresponds to the system of equations
\begin{align*}
  -x_1^{(2)} + 2x_2^{(2)} -2x_3^{(2)} &= 0 \\
  x_2^{(2)} - x_3^{(2)} &= 0
.\end{align*}
We quickly see that $x_2^{(2)} = x_3^{(2)}$ and $x_1^{(2)} = 0$. We choose
\[ 
\Vec{x}^{(2)} = \begin{pmatrix}
0\\
1\\
1\\
\end{pmatrix}
.\]
For $\lambda_3 = 3$ we get
\begin{align*}
  A \Vec{x}^{(3)} &= 3 \Vec{x}^{(3)} \\
  \implies (A - 3I) \Vec{x}^{(3)} &= 0 \\
  \implies \begin{pmatrix}
  -2 & 2 & -2\\
  0 & 0 & -1\\
  0 & 0 & -1\\
  \end{pmatrix} \begin{pmatrix}
  x_1^{(3)}\\
  x_2^{(3)}\\
  x_3^{(3)}\\
  \end{pmatrix} &= \begin{pmatrix}
  0\\
  0\\
  0\\
  \end{pmatrix}
.\end{align*}
This corresponds to the system of equations
\begin{align*}
  -2x_1^{(3)} + 2x_2^{(3)} -2x_3^{(3)} &= 0 \\
  -x_3^{(3)} &= 0 \\
.\end{align*}
We see that $x_1^{(3)} = x_2^{(3)}$ whilst $x_3^{(3)} = 0$. We choose
\[ 
\Vec{x}^{(3)} = \begin{pmatrix}
1\\
1\\
0\\
\end{pmatrix}
.\]
We can now construct $X$ as
\[ 
X = (\Vec{x}^{(1)} \Vec{x}^{(2)} \Vec{x}^{(3)}) = \begin{pmatrix}
1 & 0 & 1\\
0 & 1 & 1\\
0 & 1 & 0\\
\end{pmatrix}
.\]
We can now find $X^{-1}$ with Gauss-Jordan elimination as
\[
  (XI) = \left( \begin{array}{ccc|ccc}
  1 & 0 & 1 & 1 & 0 & 0\\
  0 & 1 & 1 & 0 & 1 & 0\\
  0 & 1 & 0 & 0 & 0 & 1\\
  \end{array} \right)
.\]
We switch row (2) and (3) to get
\[ 
\left( \begin{array}{ccc|ccc}
1 & 0 & 1 & 1 & 0 & 0\\
0 & 1 & 0 & 0 & 0 & 1\\
0 & 1 & 1 & 0 & 1 & 0\\
\end{array} \right)
.\]
We can now subtract (2) from (3) to get
\[ 
\left( \begin{array}{ccc|ccc}
1 & 0 & 1 & 1 & 0 & 0\\
0 & 1 & 0 & 0 & 0 & 1\\
0 & 0 & 1 & 0 & 1 & -1\\
\end{array} \right)
.\]
Finally we can subtract (3) from (1) to get
\[ 
\left( \begin{array}{ccc|ccc}
1 & 0 & 0 & 1 & -1 & 1\\
0 & 1 & 0 & 0 & 0 & 1\\
0 & 0 & 1 & 0 & 1 & -1\\
\end{array} \right)
.\]
We therefore get, that
\[ 
X^{-1} = \begin{pmatrix}
1 & -1 & 1\\
0 & 0 & 1\\
0 & 1 & -1\\
\end{pmatrix}
.\]
We can now find $A^{4}$ as
\begin{align*}
  A^{4} &= X D^{4} X^{-1} \\
  &= \begin{pmatrix}
  1 & 0 & 1\\
  0 & 1 & 1\\
  0 & 1 & 0\\
  \end{pmatrix}\begin{pmatrix}
  1 & 0 & 0\\
  0 & 16 & 0\\
  0 & 0 & 81\\
  \end{pmatrix} \begin{pmatrix}
  1 & -1 & 1\\
  0 & 0 & 1\\
  0 & 1 & -1\\
  \end{pmatrix}\\
  &= \begin{pmatrix}
  1 & 0 & 1\\
  0 & 1 & 1\\
  0 & 1 & 0\\
  \end{pmatrix} \begin{pmatrix}
  1 & -1 & 1\\
  0 & 0 & 16\\
  0 & 81 & -81\\
  \end{pmatrix} \\
  &= \begin{pmatrix}
  1 & 80 & -80\\
  0 & 81 & -65\\
  0 & 0 & 16\\
  \end{pmatrix}
.\end{align*}



\subsection*{7.} Find a matrix $A$ that has the eigenvalues $\lambda_1 = 1$, $\lambda_2 = 0$, $\lambda_3 = 3$ and the corresponding eigenvectors
\[ 
\Vec{x}^{(1)} = \begin{pmatrix}
1\\
1\\
1\\
\end{pmatrix}, \Vec{x}^{(2)} = \begin{pmatrix}
0\\
1\\
1\\
\end{pmatrix}, \Vec{x}^{(3)} = \begin{pmatrix}
0\\
0\\
1\\
\end{pmatrix}
.\]
\bigbreak
For the matrix $A$ to have the mentioned Eigenvalues it must have a diagonal matrix of
\[ 
\begin{pmatrix}
1 & 0 & 0\\
0 & 0 & 0\\
0 & 0 & 3\\
\end{pmatrix}
.\]
We can also construct $X$ as
\[ 
X = \left( \Vec{x}^{(1)} \Vec{x}^{(2)} \Vec{x}^{(3)} \right) = \begin{pmatrix}
1 & 0 & 0\\
1 & 1 & 0\\
1 & 1 & 1\\
\end{pmatrix}
.\]
We can now find $X^{-1}$ through Gauss-Jordan elimination as
\[ 
  (XI) = \left( \begin{array}{ccc|ccc}
  1 & 0 & 0 & 1 & 0 & 0\\
  1 & 1 & 0 & 0 & 1 & 0\\
  1 & 1 & 1 & 0 & 0 & 1\\
  \end{array} \right)
.\]
We subtract (2) from (3) to get
\[ 
\left( \begin{array}{ccc|ccc}
1 & 0 & 0 & 1 & 0 & 0\\
1 & 1 & 0 & 0 & 1 & 0\\
0 & 0 & 1 & 0 & -1 & 1\\
\end{array} \right)
.\]
We subtract (1) from (2)
\[ 
\left( \begin{array}{ccc|ccc}
1 & 0 & 0 & 1 & 0 & 0\\
0 & 1 & 0 & -1 & 1 & 0\\
0 & 0 & 1 & 0 & -1 & 1\\
\end{array} \right)
.\]
Therefore
\[ 
X^{-1} = \begin{pmatrix}
1 & 0 & 0\\
-1 & 1 & 0\\
0 & -1 & 1\\
\end{pmatrix}
.\]
We can now find $A$ as
\begin{align*}
  A &= X D X^{-1}  \\
    &= \begin{pmatrix}
1 & 0 & 0\\
1 & 1 & 0\\
1 & 1 & 1\\
\end{pmatrix} \begin{pmatrix}
1 & 0 & 0\\
0 & 0 & 0\\
0 & 0 & 3\\
\end{pmatrix} \begin{pmatrix}
1 & 0 & 0\\
-1 & 1 & 0\\
0 & -1 & 1\\
\end{pmatrix} \\
&= \begin{pmatrix}
1 & 0 & 0\\
1 & 1 & 0\\
1 & 1 & 1\\
\end{pmatrix} \begin{pmatrix}
1 & 0 & 0\\
0 & 0 & 0\\
0 & -3 & 3\\
\end{pmatrix} \\
&= \begin{pmatrix}
1 & 0 & 0\\
1 & 0 & 0\\
1 & -3 & 3\\
\end{pmatrix}
.\end{align*}

