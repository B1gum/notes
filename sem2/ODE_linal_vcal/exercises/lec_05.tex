\section*{Lecture 5}

\subsection*{1.}
Calculate the determinant of
\[ 
\begin{pmatrix}
1 & 2 & 1 & 1\\
0 & 2 & 2 & 0\\
1 & 2 & 2 & 2\\
0 & 2 & 1 & 3\\
\end{pmatrix}
.\]
\bigbreak
We start by choosing the column $k = 1$ as the column about which to apply Cramer's rule. We get
\[ 
D = \mathrm{det}(A) = \sum_{j = 1}^{4} (-1)^{j+1} a_{j1} M_{j1} = (-1)^{2} a_{11} M_{11} + (-1)^{4} a_{31}M_{31} = M_{11} + M_{31}  
.\]
We thus need to find the minors $M_{11}$ and $M_{31}$. These are
\begin{align*}
  M_{11} &= \begin{pmatrix}
2 & 2 & 0\\
2 & 2 & 2\\
2 & 1 & 3\\
\end{pmatrix} &&\text{and} & M_{31} &= \begin{pmatrix}
2 & 1 & 1\\
2 & 2 & 0\\
2 & 1 & 3\\
\end{pmatrix}
.\end{align*}
For $M_{11}$ we will use the first row $j = 1$ to get
\[ 
M_{11} = \left| \begin{array}{ccc}
2 & 2 & 0\\
2 & 2 & 2\\
2 & 1 & 3\\
\end{array} \right| = 2 \left| \begin{array} {cc}
2 & 2\\
1 & 3\\
\end{array} \right| - 2 \left| \begin{array} {cc}
2 & 2\\
2 & 3\\
\end{array} \right| = 2 \cdot 4 - 2 \cdot 2 = 4
.\]
For $M_{31}$ we will use the second row $j = 2$ to get
\[ 
M_{31} = \left| \begin{array}{ccc}
2 & 1 & 1\\
2 & 2 & 0\\
2 & 1 & 3\\
\end{array} \right| = - 2 \left| \begin{array}{cc}
1 & 1\\
1 & 3\\
\end{array} \right| + 2 \left| \begin{array}{cc}
2 & 1\\
2 & 3\\
\end{array} \right| = -2 \cdot 2 + 2 \cdot 4 = 4
.\]
We therefore get
\[ 
D = M_{11} + M_{31} = 4 + 4 = 8
.\]



\subsection*{2.}
Solve
\[ 
A \Vec{x} = \Vec{b}, A = \begin{pmatrix}
1 & 1 & 1\\
0 & 1 & 1\\
0 & 0 & 1\\
\end{pmatrix}, \Vec{b} = \begin{pmatrix}
0\\
1\\
0\\
\end{pmatrix}
.\]
Using Cramer's rule.
\bigbreak
To use Cramer's rule we must first find $D = \mathrm{det}(A)$. As $A$ is a triangular matrix this is just the product of the diagonal so we get
\[ 
D = \mathrm{det}(A) = 1\cdot 1\cdot 1 = 1
.\]
Now we must find the determinant
\[ 
D_1 = \left| \begin{array}{ccc}
0 & 1 & 1\\
1 & 1 & 1\\
0 & 0 & 1\\
\end{array} \right|
.\]
Here we will use the first column to get
\[ 
  D_1 = - 1 \left| \begin{array}{cc}
  1 & 1\\
  0 & 1\\
  \end{array} \right| = -1 \cdot 1 = -1
.\]
Now for the determinant
\[ 
D_2 = \left| \begin{array}{ccc}
1 & 0 & 1\\
0 & 1 & 1\\
0 & 0 & 1\\
\end{array} \right|
.\]
Here we will use the first column to get
\[ 
D_2 = 1 \cdot \left| \begin{array}{cc}
1 & 1\\
0 & 1\\
\end{array} \right| = 1
.\]
Now for the determinant
\[ 
D_3 = \left| \begin{array}{ccc}
1 & 1 & 0\\
0 & 1 & 1\\
0 & 0 & 0\\
\end{array} \right|
.\]
Here we will use the first column to get
\[ 
D_3 = 1 \cdot \left| \begin{array}{cc}
1 & 1\\
0 & 0\\
\end{array} \right| = 0
.\]
Therefore the solution is
\[ 
\Vec{x} = \begin{pmatrix}
D_1 / D\\
D_2 / D\\
D_3 / D\\
\end{pmatrix} = \begin{pmatrix}
-1\\
1\\
0\\
\end{pmatrix}
.\]




\subsection*{3.}
Solve
\[ 
A \Vec{x} = \Vec{b}, A = \begin{pmatrix}
1 & 1 & 1\\
1 & 1 & 2\\
1 & 2 & 1\\
\end{pmatrix}, \Vec{b} = \begin{pmatrix}
0\\
1\\
0\\
\end{pmatrix}
.\]
\bigbreak
We will solve the exercise using Cramer's rule. First we must find $D = \mathrm{det}(A)$. To do this we will use the first row to get
\[ 
D = \left| \begin{array}{ccc}
1 & 1 & 1\\
1 & 1 & 2\\
1 & 2 & 1\\
\end{array} \right| = 1 \left| \begin{array}{cc}
1 & 2\\
2 & 1\\
\end{array} \right| - 1 \left| \begin{array}{cc}
1 & 2\\
1 & 1\\
\end{array} \right| + 1 \left| \begin{array}{cc}
1 & 1\\
1 & 2\\
\end{array} \right| = 1 \cdot -3 - 1 \cdot -1 + 1 \cdot 1 = -1
.\]
Now we will find the determinant $D_1$ as
\[ 
D_1 = \left| \begin{array}{ccc}
0 & 1 & 1\\
1 & 1 & 2\\
0 & 2 & 1\\
\end{array} \right| = -1 \cdot \left| \begin{array}{cc}
1 & 1\\
2 & 1\\
\end{array} \right| = -1 \cdot -1 = 1
.\]
Then we will find $D_2$ as
\[ 
D_2 = \left| \begin{array}{ccc}
1 & 0 & 1\\
1 & 1 & 2\\
1 & 0 & 1\\
\end{array} \right| = 1 \left| \begin{array}{cc}
1 & 1\\
1 & 1\\
\end{array} \right| = 1 \cdot 0 = 0
.\]
Lastly we will find $D_3$ as
\[ 
D_3 = \left| \begin{array}{ccc}
1 & 1 & 0\\
1 & 1 & 1\\
1 & 2 & 0\\
\end{array} \right| = -1 \left| \begin{array}{cc}
1 & 1\\
1 & 2\\
\end{array} \right| = -1 \cdot 1 = -1
.\]
This gives the solution
\[ 
\Vec{x} = \begin{pmatrix}
D_1 / D\\
D_2 / D\\
D_3 / D\\
\end{pmatrix} = \begin{pmatrix}
-1\\
0\\
1\\
\end{pmatrix}
.\]



\subsection*{4.}
Calculate the inverse of
\[ 
A = \begin{pmatrix}
1 & 1 & 2\\
1 & 2 & 1\\
1 & 1 & 1\\
\end{pmatrix}
\]
using Gauss-Jordan elimination. Check your result using $A^{-1} A = I$. Use the inverse to find the folution of $A \Vec{x} = \Vec{b}$ for
\[ 
\Vec{b} = \begin{pmatrix}
1\\
1\\
1\\
\end{pmatrix}
.\]
\bigbreak
We will first verify that the inverse $A^{-1}$ exists by showing $\mathrm{det}(A) \neq 0$
\[ 
D = \mathrm{det}(A) = 1 \left| \begin{array}{cc}
2 & 1\\
1 & 1\\
\end{array} \right| + 1 \cdot (-1) \left| \begin{array}{cc}
1 & 1\\
1 & 1\\
\end{array} \right| + 12\left| \begin{array}{cc}
1 & 2\\
1 & 1\\
\end{array} \right| = 1 \cdot 1 - 1 \cdot 0 + 1 \cdot -2 = -1
.\]
As $D = -1 \neq 0$ an inverse of $A$ $A^{-1}$ exists. We get the combined matrix
\[ 
  (AI) = \begin{pmatrix}
  1 & 1 & 2 & 1 & 0 & 0\\
  1 & 2 & 1 & 0 & 1 & 0\\
  1 & 1 & 1 & 0 & 0 & 1\\
  \end{pmatrix}
.\]
$(2) - (1)$ gives
\[ 
\begin{pmatrix}
1 & 1 & 2 & 1 & 0 & 0\\
0 & 1 & -1 & -1 & 1 & 0\\
1 & 1 & 1 & 0 & 0 & 1\\
\end{pmatrix}
.\]
$(3) - (1)$ gives
\[ 
\begin{pmatrix}
1 & 1 & 2 & 1 & 0 & 0\\
0 & 1 & -1 & -1 & 1 & 0\\
0 & 0 & -1 & -1 & 0 & 1\\
\end{pmatrix}
.\]
$(2) - (3)$ gives
\[ 
  \begin{pmatrix}
  1 & 1 & 2 & 1 & 0 & 0\\
  0 & 1 & 0 & 0 & 1 & -1\\
  0 & 0 & -1 & -1 & 0 & 1\\
  \end{pmatrix}
.\]
$(1) - (2)$ gives
\[ 
\begin{pmatrix}
1 & 0 & 2 & 1 & -1 & 1\\
0 & 1 & 0 & 0 & 1 & -1\\
0 & 0 & -1 & -1 & 0 & 1\\
\end{pmatrix}
.\]
$(1) + 2(3)$ gives
\[ 
\begin{pmatrix}
1 & 0 & 0 & -1 & -1 & 3\\
0 & 1 & 0 & 0 & 1 & -1\\
0 & 0 & -1 & -1 & 0 & 1\\
\end{pmatrix}
.\]
$(-1)(3)$ gives
\[ 
\begin{pmatrix}
1 & 0 & 0 & -1 & -1 & 3\\
0 & 1 & 0 & 0 & 1 & -1\\
0 & 0 & 1 & 1 & 0 & -1\\
\end{pmatrix}
.\]
Therefore the inverse is
\[ 
A^{-1} = \begin{pmatrix}
-1 & -1 & 3\\
0 & 1 & -1\\
1 & 0 & -1\\
\end{pmatrix}
.\]
We can check this as
\[ 
\begin{pmatrix}1 & 1 & 2\\1 & 2 & 1\\1 & 1 & 1\\\end{pmatrix} \cdot \begin{pmatrix}-1 & -1 & 3\\0 & 1 & -1\\1 & 0 & -1\\\end{pmatrix} = \left(\begin{array}{ccc} 1 & 0 & 0 \\ 0 & 1 & 0 \\ 0 & 0 & 1 \\\end{array}\right)
.\]
Which means the inverse has been correctly calculated. We can solve for $\Vec{x}$ as
\[ 
\Vec{x} = A^{-1} \Vec{b} = \begin{pmatrix}-1 & -1 & 3\\0 & 1 & -1\\1 & 0 & -1\\\end{pmatrix} \cdot \begin{pmatrix}1\\1\\1\\\end{pmatrix} = \left(\begin{array}{c} 1 \\ 0 \\ 0 \\\end{array}\right)
.\]




\subsection*{5.}
Calculate the inverse of
\[ 
A = \begin{pmatrix}
1 & 1 & 1\\
0 & 1 & 1\\
0 & 0 & 1\\
\end{pmatrix}
\]
using Gauss-Jordan elimination. Check your result using $A^{-1} A = I$. Use the inverse to find the solution of $A \Vec{x} = \Vec{b}$ for
\[ 
\Vec{b} = \begin{pmatrix}
1\\
1\\
1\\
\end{pmatrix}
.\]
\bigbreak
We can quickly see that the determinant $D = \mathrm{det}(A) = 1 \neq 0$ as $A$ is a triangular matrix. Therefore an inverse exists. We get the combined matrix
\[ 
  (AI) = \begin{pmatrix}
    1 & 1 & 1 & 1 & 0 & 0\\
    0 & 1 & 1 & 0 & 1 & 0\\
    0 & 0 & 1 & 0 & 0 & 1\\
  \end{pmatrix}
.\]
$(1) - (2)$ gives
\[ 
\begin{pmatrix}
1 & 0 & 0 & 1 & -1 & 0\\
0 & 1 & 1 & 0 & 1 & 0\\
0 & 0 & 1 & 0 & 0 & 1\\
\end{pmatrix}
.\]
$(2) - (3)$ gives
\[ 
\begin{pmatrix}
1 & 0 & 0 & 1 & -1 & 0\\
0 & 1 & 0 & 0 & 1 & -1\\
0 & 0 & 1 & 0 & 0 & 1\\
\end{pmatrix}
.\]
Therefore the inverse $A^{-1}$ of $A$ is
\[ 
A^{-1} = \begin{pmatrix}
1 & -1 & 0\\
0 & 1 & -1\\
0 & 0 & 1\\
\end{pmatrix}
.\]
We can check this as
\[ 
 \begin{pmatrix}1 & 1 & 1\\0 & 1 & 1\\0 & 0 & 1\\\end{pmatrix} \cdot \begin{pmatrix}1 & -1 & 0\\0 & 1 & -1\\0 & 0 & 1\\\end{pmatrix} = \left(\begin{array}{ccc} 1 & 0 & 0 \\ 0 & 1 & 0 \\ 0 & 0 & 1 \\\end{array}\right)
.\]
Which confirms our result. We can now solve for $\Vec{x}$ as
\[ 
\Vec{x} = A^{-1} \Vec{b} = \begin{pmatrix}1 & -1 & 0\\0 & 1 & -1\\0 & 0 & 1\\\end{pmatrix} \cdot \begin{pmatrix}1\\1\\1\\\end{pmatrix} = \left(\begin{array}{c} 0 \\ 0 \\ 1 \\\end{array}\right)
.\]


\subsection*{6.}
Calculate the inverse of
\[ 
A = \begin{pmatrix}
1 & 1 & 2\\
1 & 2 & 1\\
1 & 1 & 1\\
\end{pmatrix}
\]
using the method in Section 6.2. Check your result using $A^{-1} A = I$.
\bigbreak
Firstly we determine the determinant using the first row
\[ 
D = \mathrm{det}(A) = \left| \begin{array}{ccc}
1 & 1 & 2\\
1 & 2 & 1\\
1 & 1 & 1\\
\end{array} \right| = 1 \cdot \left| \begin{array}{cc}
2 & 1\\
1 & 1\\
\end{array} \right| + 1\cdot (-1) \cdot \left| \begin{array}{cc}
1 & 1\\
1 & 1\\
\end{array} \right| + 2 \cdot \left| \begin{array}{cc}
1 & 2\\
1 & 1\\
\end{array} \right| = 1 \cdot 1 -1 \cdot 0 + 2 \cdot -1 = -1
.\]
All the minors are
\begin{align*}
  M_{11} &= \left| \begin{array}{cc}
  2 & 1\\
  1 & 1\\
  \end{array} \right| = 1, & M_{12} &= \left| \begin{array}{cc}
  1 & 1\\
  1 & 1\\
\end{array} \right| = 0, & M_{13} &= \left| \begin{array}{cc}
1 & 2\\
1 & 1\\
\end{array} \right| = -1, \\
  M_{21} &= \left| \begin{array}{cc}
  1 & 2\\
  1 & 1\\
\end{array} \right| = -1, & M_{22} &= \left| \begin{array}{cc}
1 & 2\\
1 & 1\\
\end{array} \right| = -1, & M_{23} &= \left| \begin{array}{cc}
1 & 1\\
1 & 1\\
\end{array} \right| = 0, \\
    M_{31} &= \left| \begin{array}{cc}
    1 & 2\\
    2 & 1\\
    \end{array} \right| = -3, & M_{32} &= \left| \begin{array}{cc}
  1 & 2\\
  1 & 1\\
  \end{array} \right| = -1, & M_{33} &= \left| \begin{array}{cc}
1 & 1\\
1 & 2\\
\end{array} \right| = 1
.\end{align*}
The matrix $C = \left| c_{jk} \right| = (-1)^{j+k}M_{jk}$ is thus
\[ 
C = \begin{pmatrix}
1 & 0 & -1\\
1 & -1 & 0\\
-3 & 1 & 1\\
\end{pmatrix}
.\]
And therefore $C^{T}$ is
\[ 
C^{T} = \begin{pmatrix}
1 & 1 & -3\\
0 & -1 & 1\\
-1 & 0 & 1\\
\end{pmatrix}
.\]
We then have
\[ 
A^{-1} = \frac{C^{T}}{D} = \begin{pmatrix}
-1 & -1 & 3\\
0 & 1 & -1\\
1 & 0 & -1\\
\end{pmatrix}
.\]
We can check this as
\[ 
A A^{-1} = \begin{pmatrix}1 & 1 & 2\\1 & 2 & 1\\1 & 1 & 1\\\end{pmatrix} \cdot \begin{pmatrix}-1 & -1 & 3\\0 & 1 & -1\\1 & 0 & -1\\\end{pmatrix} = \left(\begin{array}{ccc} 1 & 0 & 0 \\ 0 & 1 & 0 \\ 0 & 0 & 1 \\\end{array}\right)
\]
