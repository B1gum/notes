\documentclass[a4paper]{article}
\usepackage[danish]{babel}
\usepackage{amsfonts, amssymb, mathtools, amsthm, amsmath}
\usepackage{graphicx, pgfplots}
\usepackage{url}
\usepackage[dvipsnames]{xcolor}
\usepackage{lastpage}

%loaded last
\usepackage[hidelinks]{hyperref}

\usepackage{siunitx}
  \sisetup{exponent-product = \cdot,
    output-decimal-marker = {,}}

%Giles Castelles incfig
\usepackage{import}
\usepackage{xifthen}
\usepackage{pdfpages}
\usepackage{transparent}

\newcommand{\incfig}[2][1]{%
  \def\svgwidth{#1\columnwidth}
  \import{./figures/}{#2.pdf_tex}
}

\setlength{\parindent}{0in}
\setlength{\parskip}{12pt}
\setlength{\oddsidemargin}{0in}
\setlength{\textwidth}{6.5in}
\setlength{\textheight}{8.8in}
\setlength{\topmargin}{0in}
\setlength{\headheight}{18pt}

\usepackage{fancyhdr}
\pagestyle{fancy}

\fancyhead{}
\fancyfoot{}
\fancyfoot[R]{Side \thepage{} af \pageref{LastPage}}
\fancyhead[L]{\footnotesize{Noah Rahbek Bigum Hansen}}

% Redefine the plain page style to be consistent
\fancypagestyle{plain}{
  \fancyhead{} % Clears all header content
  \fancyfoot{} % Clears all footer content
  \renewcommand{\headrulewidth}{0pt} % Removes the horizontal line
  \fancyfoot[R]{Side \thepage{} af \pageref{LastPage}} % Page number in the footer
}
\pgfplotsset{compat=newest}

\pgfplotsset{every axis/.append style={
  axis x line=middle,    % put the x axis in the middle
  axis y line=middle,    % put the y axis in the middle
  axis line style={<->,color=black}, % arrows on the axis
}}

\makeatletter

\newcommand{\opgave}[1]{%
 \def\@exercise{#1}%
 \paragraph{Opgave #1}
}

\makeatother

%Format lim the same way in intext and in display
\let\svlim\lim\def\lim{\svlim\limits}

% horizontal rule
\newcommand\hr{
\noindent\rule[0.5ex]{\linewidth}{0.5pt}
}

\author{Noah Rahbek Bigum Hansen -- 202405538}


\title{Ordinary Differential Equations, Linear Algebra and Vector Calculus – Exam}
\begin{document}

\maketitle

\exercise{1}
Let $a$ be a real number and
\[ 
A = \begin{pmatrix}
1 & a\\
1 & 1\\
\end{pmatrix}
.\]
Calculate all values of $a$ such that the inverse $A^{-1}$ doesn't exist. 
\bigbreak
For the inverse to exist, we must have $\mathrm{det}(A) \neq 0$. The determinant of $A$ can simply be found as:
\[ 
\mathrm{det}(A) = \left| \begin{array}{cc}
1 & a\\
1 & 1\\
\end{array} \right| = 1 - a \neq 0
.\]
This equality does not hold for $a = 1$, hence no inverse $A^{-1}$ exists for $a = 1$. For $a \neq 1$ we have $\mathrm{det}(A)\neq 0$ and therefore an inverse exists.


\exercise{2}
Let $a$ be a real number and
\[ 
A = \begin{pmatrix}
1 & a\\
1 & 1\\
\end{pmatrix}
.\]
Calculate all values of $a$ such that $A$ has the eigenvalue $\lambda = 1$.
\bigbreak
We know that $\lambda$ is an eigenvalue to a matrix, if and only if:
\[ 
\mathrm{det}(A - \lambda I) = 0
.\]
Here we are looking for solutions with $\lambda = 1$ and therefore we get:
\[ 
\mathrm{det}(A - I) = \left| \begin{array}{cc}
0 & a\\
1 & 0\\
\end{array} \right| = 0 \implies 0 - a = 0 \implies a = 0
.\]
Therefore $a = 0$ is the solution we are looking for. 


\exercise{3}
Consider the linear system of $m$ equations with $n$ unknowns
\[ 
A \Vec{x} = \Vec{0}
.\]
Suppose all solutions $\Vec{x}$ are of the form
\[ 
\Vec{x} = c_1 \begin{pmatrix}
1\\
1\\
1\\
0\\
\end{pmatrix} + c_2 \begin{pmatrix}
0\\
1\\
1\\
1\\
\end{pmatrix}
.\]
Find $n$ and $\mathrm{rank}(A)$.
\bigbreak
As our solution is a linear combination of two vectors with length 4 we have exactly 4 unknowns as well. We can write our solution as
\[ 
\Vec{x} = \begin{pmatrix}
c_1 \\
c_1 + c_2 \\
c_1 + c_2\\
c_2 \\
\end{pmatrix}
.\]
And here we can quickly see that $\mathrm{rank}(\Vec{x}) = 2$ as we have two linearly independent vectors (entry 1 and entry 4) and the rest (entry 2 and 3) are simply the sum of these.


\exercise{4}
Consider the nonlinear system of first order ODEs
\begin{align*}
  y_1'(t) &= e^{y_1(t)} + y_2(t) \\
  y_2'(t) &= e^{y_1(t)} -2
.\end{align*}
Calculate the location of all critical points.
\bigbreak
We define the system as
\begin{align*}
  y_1'(t) &= e^{y_1(t)} + y_2(t) = f_1(y_1(t), y_2(t)) \\
  y_2'(t) &= e^{y_1(t)} - 2 = f_2(y_1(t), y_2(t))
.\end{align*}
For a critical point we have that:
\[ 
\frac{\mathrm{d}y_2}{\mathrm{d}y_1} = \frac{y_2'(t) \, \mathrm{d}t}{y_1'(t) \, \mathrm{d}t } = \frac{y_2'(t)}{y_1'(t)} = \frac{f_2(y_1(t), y_2(t))}{f_1(y_1(t), y_2(t))} = \frac{0}{0}
.\]
For $f_2 = 0$ we have:
\begin{align*}
  e^{y_1(t)} - 2 &= 0 \\
  e^{y_1(t)} &= 2 \\
  y_1(t) &= \ln 2
.\end{align*}
And for $f_1 = 0$ we have:
\begin{align*}
  e^{\ln 2} + y_2(t) &= 0 \\
  2 + y_2(t) &= 0 \\
  y_2(t) &= -2
.\end{align*}
Therefore for the condition
\[ 
\frac{\mathrm{d}y_2}{\mathrm{d}y_1} = \frac{0}{0}
\]
to hold we must have $y_1(t) = \ln 2$ and $y_2(t) = -2$.

\exercise{5}
Calculate the general solution of
\[ 
y''(x) - 4y(x) = x + e^{x}
.\]
\bigbreak
We compare this to the standard form for a nonhomogeneous second order linear ODE:
\[ 
y''(x) + p(x) y'(x) + q(x) y(x) = r(x)
.\]
Here we see that $p(x) = 0$, $q(x) = -4$ and $r(x) = x + e^{x}$. To find the solution of this we start by finding the solution to the corresponding homogeneous ODE:
\[ 
y_h''(x) - 4y_h(x) = 0
.\]
This is a homogeneous linear second order ODE with constant coefficients. The characteristic equation is:
\[ 
  \lambda^2 - 4 = 0 \implies \lambda = \{-2, 2\}
.\]
As these are both real and different we have the two solutions:
\[ 
y_{h_1}(x) = c_1 e^{2x} \quad \text{and} \quad y_{h_2} = c_2 e^{-2x}
.\]
These are linearly independent and constitute a basis. The general solution to the homogeneous ODE is therefore:
\[ 
y_h(x) = c_1 e^{2x} + c_2 e^{-2x}
.\]
To find a general solution for the nonhomogeneous ODE we employ the method of undetermined coefficients. Using the basic and sum rules we get a solution of the form:
\[ 
y_r(x) = C e^{x} + K_1 x + K_0
.\]
This has the derivatives:
\begin{align*}
  y_r'(x) &= C e^x + K_1 \\
  y_r''(x) &= C e^{x}
.\end{align*}
We can now insert this into the nonhomogeneous ODE as:
\begin{align*}
  y_r''(x) - 4y_r(x) &= x + e^{x} \\
  C e^{x} - 4C e^{x} - 4K_1 x + 4K_0 &= x + e^{x} \\
  -3 C e^{x}  - 4K_1 x + 4K_0 &= x + e^{x}
.\end{align*}
\begin{align*}
  -3 C e^{x} &= e^{x} & -4 K_1 x &= x \\
  -3 C &= 1 & -4K_1 &= 1 \\
  C &= - \frac{1}{3} & K_1 &= -\frac{1}{4}
.\end{align*}
Therefore
\[ 
y_r(x) = -\frac{1}{3} e^{x} - \frac{1}{4}x
.\]
And the general solution therefore is:
\[ 
y(x) = y_h(x) + y_r(x) = c_1 e^{2x} + c_2 e^{-2x} - \frac{1}{3} e^{x} - \frac{1}{4} x = c_1 e^{2x} + c_2 e^{-2x} - \frac{4e^{x} + 3x}{12}
.\]

\exercise{6}
Solve the initial value problem for $x > 0$
\[ 
2xy(x)^2 + 2x^2y(x)y'(x) = 0, y(1) = 1
.\]
using the theory of exact ODEs.
\bigbreak
We start by comparing the given ODE to the standard form for an exact ODE:
\[ 
M(x,y) + N(x,y)y' = 0
.\]
Here we see that:
\[ 
M(x,y) = 2xy^2 \quad \text{and} \quad N(x,y) = 2x^2 y
.\]
We can check if it actually is exact as:
\[ 
\frac{\partial M}{\partial y} = 4xy = \frac{\partial N}{\partial x}
.\]
As this holds the ODE is exact. We then get:
\[ 
u(x,y) = \int M(x,y) \, \mathrm{d}x + k(y) = \int 2 \cdot x \cdot y^2 \, \mathrm{d}x + k(y) = x^2 y^2 + k(y)
.\]
We also have that:
\[ 
N(x,y) = \frac{\partial u(x,y)}{\partial y} \implies 2x^2 y(x) = 2x^2 y + k(y) \implies k(y) = 0
.\]
Our function $u(x,y)$ is therefore:
\[ 
u(x,y) = x^2 y^2
.\]
We now choose a constant $c$
\begin{align*}
  u(x,y) &= c \\
  x^2 y^2 &= c \\
  y^2 &= \frac{c}{x^2} \\
  y &= \sqrt{\frac{c}{x}} \\
  y &= \frac{k}{x}, k = \sqrt{c}
.\end{align*}
And therefore our general solution  is:
\[ 
y(x) = \frac{k}{x}
.\]
We can now insert our initial value as:
\[ 
1 = \frac{k}{1} \implies k = 1
.\]
And the particular solution is therefore:
\[ 
y(x) = \frac{1}{x}
.\]


\exercise{7}
Solve the initial value problem for $x > 0$
\[ 
xy'(x) + y(x) = 0, y(1) = 1
.\]
Using the theory of separable ODEs.
\bigbreak
We start by rewriting the ODE as:
\[ 
y'(x) =  \frac{y(x)}{x}
.\]
This can be compared to the standard form for a separable ODE:
\[ 
y'(x) = \frac{h(x)}{g(y(x))}
.\]
We see that $h(x) = \frac{1}{x}$ and $g(y(x)) = - \frac{1}{y(x)}$. We can now integrate as:
\begin{align*}
  \int_{y(x_0)}^{y(x)} g(y) \, \mathrm{d}y &= \int_{x_0}^{x} h(\hat{x}) \, \mathrm{d}\hat{x} \\
  \left[ -\ln (y) \right]_{y(x_0)}^{y(x)} &= \left[ \ln x \right]_{x_0}^{x} \\
  - \ln y(x) + \ln 1 &= \ln x - \ln 1 \\
  \ln y(x) &= - \ln x - \ln 2 \\
  y(x) &= e^{-\frac{x}{2}}
.\end{align*}
It seems like something has gone wrong here as this does not fulfill our initial condition ($y(1) = \frac{1}{\sqrt{e}} \neq 1$). The actual solution to the ODE is expected to have the form:
\[ 
y(x) = \frac{c_1}{x}
  .\]
In which case the initial condition gives the solution:
\[ 
y(x) = \frac{1}{x}
.\]


\exercise{8}
Consider the curve $C$ given by all points $P(x,y,z)$ such that
\[ 
x - y^2 = 0, z = x, 0 \leq x \leq 1, y \leq 0
.\]
Find a parametric representation.
\bigbreak
We start by rewriting the surface as:
\[ 
y^2 = x, z = x, 0 \leq x \leq 1, y \leq 0
.\]
We can now quickly see that we can parameterize it as: $(t, \pm \sqrt{t}, t)$. As we are only looking for the solutions with $y \leq 0$ we can reduce it to $(t, - \sqrt{t}, t)$.
And finally we can add the boundaries for $x$:
\[ 
\Vec{r}(t) (t , - \sqrt{t}, t), 0 \leq t \leq 1
.\]



\exercise{9}
Consider the surface $S$ given by all points $P(x,y,z)$ such that
\[ 
x - y^2 = 0, z \leq x, 0 \leq x \leq 1, y \geq 0
.\]
Find a parametric representation.
\bigbreak
This case has a lot of similarities with the above. However, this time the $z$-value is free (as we are working with a surface and not a curve anymore). We therefore get this again:
\[ 
\Vec{r}(u, v) = (u , \pm \sqrt{u} , v)
.\]
This time we are only looking for the positive values of $y$, hence:
\[ 
\Vec{r}(u,v) = (u, \sqrt{u}, v)
.\]
And now the boundaries can be added as:
\[ 
\Vec{r}(u,v) = ( u, \sqrt{u}, v),  0 \leq u \leq 1, v \leq u
.\]


\exercise{10}
Consider the vector field $\Vec{F} \left( \Vec{r} \right) = \left( ye^{xy}+ ze^{x}, x e^{xy} + 1, e^{x} \right)$. Calculate a scalar field $f \left( \Vec{r} \right)$ such that $\nabla f \left( \Vec{r} \right) = \Vec{F} \left( \Vec{r} \right)$
\bigbreak
For the condition to hold, we have that:
\begin{align*}
  F_1(\Vec{r}) &= \frac{\partial f (\Vec{r})}{\partial x} \\
  \implies y e^{xy} + z e^{x} &= \frac{\partial f (\Vec{r})}{\partial x} \\
  \implies f \left( \Vec{r} \right) &= e^{x y}+e^x z + g(y,z)\\
  F_2 \left( \Vec{r} \right) &= \frac{\partial f (\Vec{r})}{\partial y} \\
  \implies x e^{xy} + 1 &= x e^{x y} + g(z)\\
  \implies g(z) = 1
.\end{align*}
Therefore the surface becomes:
\[ 
f (\Vec{r}) = e^{xy} + e^{x}z + 1
.\]


\end{document}
