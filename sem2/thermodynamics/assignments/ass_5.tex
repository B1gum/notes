\documentclass[a4paper]{article}

\usepackage[english]{babel}
\usepackage{amsfonts, amssymb, mathtools, amsthm, amsmath}
\usepackage{graphicx, pgfplots}
\usepackage{bm} 
\usepackage{url}
\usepackage[dvipsnames]{xcolor}
\usepackage{lastpage}
\usepackage{chngcntr}
  \counterwithin{figure}{section}
  \renewcommand{\thefigure}{\thesection.\arabic{figure}}

%loaded last
\usepackage[hidelinks]{hyperref}
\usepackage[nameinlink]{cleveref} 

\usepackage{siunitx}
  \sisetup{exponent-product = \cdot,
    output-decimal-marker = {,}}

%Giles Castelles incfig
\usepackage{import}
\usepackage{xifthen}
\usepackage{pdfpages}
\usepackage{transparent}

\newcommand{\incfig}[2][1]{%
  \def\svgwidth{#1\columnwidth}
  \import{./figures/}{#2.pdf_tex}
}

\setlength{\parindent}{0in}
\setlength{\parskip}{12pt}
\setlength{\oddsidemargin}{0in}
\setlength{\textwidth}{6.5in}
\setlength{\textheight}{8.8in}
\setlength{\topmargin}{0in}
\setlength{\headheight}{18pt}

\usepackage{fancyhdr}
\pagestyle{fancy}

\fancyhead{}
\fancyfoot{}
\fancyfoot[R]{\thepage}
\fancyhead[C]{\leftmark}

\pgfplotsset{compat=newest}

\pgfplotsset{every axis/.append style={
  axis x line=middle,    % put the x axis in the middle
  axis y line=middle,    % put the y axis in the middle
  axis line style={<->,color=black}, % arrows on the axis
}}

\usepackage{thmtools}
\usepackage{tcolorbox}
  \tcbuselibrary{skins, breakable}
  \tcbset{
    space to upper=1em,
    space to lower=1em,
  }

\theoremstyle{definition}

\newtcolorbox[auto counter]{definition}[1][]{%
  breakable,
  colframe=ForestGreen,  %frame color
  colback=ForestGreen!5, %background color
  colbacktitle=ForestGreen!25, %background color for title
  coltitle=ForestGreen!70!black,  %title color
  fonttitle=\bfseries\sffamily, %title font
  left=1em,              %space on left side in box,
  enhanced,              %more options
  frame hidden,          %hide frame
  borderline west={2pt}{0pt}{ForestGreen},  %display left line
  title=Definition \thetcbcounter: #1,
}

\newtcolorbox{greenline}{%
  breakable,
  colframe=ForestGreen,  %frame color
  colback=white,          %remove background color
  left=1em,              %space on left side in box
  enhanced,              %more options
  frame hidden,          %hide frame
  borderline west={2pt}{0pt}{ForestGreen},  %display left line
}

\newtcolorbox[auto counter, number within=section]{dis}[1][]{%
  breakable,
  colframe=NavyBlue,  %frame color
  colback=NavyBlue!5, %background color
  colbacktitle=NavyBlue!25,    %background color for title
  coltitle=NavyBlue!70!black,  %title color
  fonttitle=\bfseries\sffamily, %title font
  left=1em,            %space on left side in box,
  enhanced,            %more options
  frame hidden,        %hide frame
  borderline west={2pt}{0pt}{NavyBlue},  %display left line
  title=Discussion \thetcbcounter: #1
}

\newtcolorbox{blueline}{%
  breakable,
  colframe=NavyBlue,     %frame color
  colback=white,         %remove background
  left=1em,              %space on left side in box,
  enhanced,              %more options
  frame hidden,          %hide frame
  borderline west={2pt}{0pt}{NavyBlue},  %display left line
}

\newtcolorbox{exa}[1][]{%
  breakable,
  colframe=RawSienna,  %frame color
  colback=RawSienna!5, %background color
  colbacktitle=RawSienna!25,    %background color for title
  coltitle=RawSienna!70!black,  %title color
  fonttitle=\bfseries\sffamily, %title font
  left=1em,              %space on left side in box,
  enhanced,              %more options
  frame hidden,          %hide frame
  borderline west={2pt}{0pt}{RawSienna},  %display left line
  title=Example: #1,
}

\newtcolorbox[auto counter, number within=section]{sæt}[1][]{%
  breakable,
  colframe=RawSienna,  %frame color
  colback=RawSienna!5, %background color
  colbacktitle=RawSienna!25,    %background color for title
  coltitle=RawSienna!70!black,  %title color
  fonttitle=\bfseries\sffamily, %title font
  left=1em,              %space on left side in box,
  enhanced,              %more options
  frame hidden,          %hide frame
  borderline west={2pt}{0pt}{RawSienna},  %display left line
  title=Sætning \thetcbcounter: #1,
  before lower={\textbf{Bevis:}\par\vspace{0.5em}},
  colbacklower=RawSienna!25,
}

\newtcolorbox{redline}{%
  breakable,
  colframe=RawSienna,  %frame color
  colback=white,       %Remove background color
  left=1em,            %space on left side in box,
  enhanced,            %more options
  frame hidden,        %hide frame
  borderline west={2pt}{0pt}{RawSienna},  %display left line
}

\newtcolorbox{des}[1][]{%
  breakable,
  colframe=NavyBlue,  %frame color
  colback=NavyBlue!5, %background color
  colbacktitle=NavyBlue!25,    %background color for title
  coltitle=NavyBlue!70!black,  %title color
  fonttitle=\bfseries\sffamily, %title font
  left=1em,              %space on left side in box,
  enhanced,              %more options
  frame hidden,          %hide frame
  borderline west={2pt}{0pt}{NavyBlue},  %display left line
  title=Description of #1,
}

\makeatother
\def\@lecture{}%
\newcommand{\lecture}[3]{
  \ifthenelse{\isempty{#3}}{%
    \def\@lecture{Lecture #1}%
  }{%
    \def\@lecture{Lecture #1: #3}%
  }%
  \subsection*{\makebox[\textwidth][l]{\@lecture \hfill \normalfont\small\textsf{#2}}}
}

\makeatletter

\newcommand{\exercise}[1]{%
 \def\@exercise{#1}%
 \subsection*{Exercise #1}
}

\makeatother

%Format lim the same way in intext and in display
\let\svlim\lim\def\lim{\svlim\limits}

% horizontal rule
\newcommand\hr{
\noindent\rule[0.5ex]{\linewidth}{0.5pt}
}

\author{Noah Rahbek Bigum Hansen}



\title{Afleveringsopgave 5 – Termodynamik}
\date{6. Marts 2025}

\begin{document}

\maketitle

\opgave{6.3}
Et trykluftsanlæg med en stempelkompresser komprimerer \qty{0,1}{kg/s} tør atmosfærisk luft fra \qty{20}{\celsius} ved atmosfæretrykket (\qty{1,013}{bar(a)}) til arbejdsluft ved \qty{10}{bar(o)}. Gassen kan betragtes som en idealgas. Følgende data er givet:
\begin{itemize}
  \item Kompressor, $\eta_s = \num{0,70}$
  \item Kompressor, $\eta_{\mathrm{mek}} = \num{0,98}$
  \item Motor, $\eta_m = \num{0,95}$
  \item Netspænding, $\qty{400}{V}$
  \item $\cos\phi = \num{0,95}$
\end{itemize}

\paragraph{1.} Bestem temperaturen i $[\unit{\celsius}]$ for kompressionen, hvis denne foregår isentropisk.
\begin{figure}[ht]
    \centering
    \incfig{a5_1}
    \caption{Systemskitse af kompressoren}
    \label{fig:a5_1}
\end{figure}
\bigbreak
I EES er indskrevet de relevante størrelser (trykket før- og efter kompressionen samt temperaturen før). Derudover er entropien til både start- og sluttilstanden fundet inden betingelsen om at disse er ens er skrevet ind. Dette er gjort som:
\begin{verbatim}
  P[1]=1,013 [bar]
  T[1]=20 [C]
  P[2]=10 [bar] + 1,013  [bar] 
  
  s[1] = entropy(Air_ha; T=T[1]; P=P[1])
  s[2] = entropy(Air_ha; T=T[2]; P=P[2])
 
  s[2] = s[1]
\end{verbatim}
På baggrund af det ovenstående har EES bestemt $T_2 = \qty{301,8}{\celsius}$.

\paragraph{2.} Bestem det tekniske arbejde i $[\unit{kW}]$ for kompressoren.
\begin{figure}[ht]
    \centering
    \incfig{a5_1}
    \caption{Systemskitse af kompressoren}
\end{figure}
\bigbreak
Idet processen er isentropisk må det ligeledes gælde at den er adiabatisk. Idet vi også ser bort fra kinetisk og potentiel energi i luften reduceres formlen for det tekniske arbejde til (Formel 5.1.12 i bogen)
\[ 
w_{t, \mathrm{isen}} = h_2 - h_1
.\]
Dette er gjort i EES som:
\begin{verbatim}
  P[1]=1,013 [bar]
  T[1]=20 [C]
  P[2]=10 [bar] + 1,013  [bar] 
  
  s[1] = entropy(Air_ha; T=T[1]; P=P[1])
  s[2] = entropy(Air_ha; T=T[2]; P=P[2])
 
  s[2] = s[1]

  h[1] = enthalpy(Air_ha;T=T[1];P=P[1])
  h[2] = enthalpy(Air_ha;T=T[2];P=P[2])
 
  w_tisen = h[2] - h[1]
\end{verbatim}
Her giver EES et specifikt indre arbejde (som er lig det tekniske arbejde, da processen er adiabatisk) på $w_{t,\mathrm{isen}} = \qty{287,5}{\frac{kJ}{kg}}$. For at omregne fra et specifikt arbejde til en effekt multipliceres dette blot med massestrømmen som
\[ 
  \dot{W}_{t, isen} = w_{t, isen} \cdot \dot{m} = \qty{287,5}{\frac{kJ}{kg}} \cdot \qty{0,1}{\frac{kg}{s}} = \qty{28,75}{kW}
.\]
altså er det tekniske arbejde for kompressoren $\dot{W}_{t, isen} = \qty{28,75}{kW}$.


\paragraph{3.} Bestem akseleffekten i $[\unit{kW}]$.
\begin{figure}[ht]
    \centering
    \incfig[0.75]{a5_2}
    \caption{Systemskitse af kompressoren og motoren}
\end{figure}
\bigbreak
Der gælder, at den isentropiske virkningsgrad $\eta_{is}$ er forholdet mellem det isentropiske (dvs. reversible og adiabatiske) arbejde $\dot{W_{t, isen}}$ og det egentlige arbejde for maskinen $\dot{W_i}$. I vores tilfælde er den reversible og adiabatiske del af arbejdet fundet som det tekniske arbejde ovenfor og det egentlige arbejde $\dot{W}_i$ er akseleffekten vi ønsker at finde i denne delopgave. Vi har
\[ 
\eta_{is} = \frac{\dot{W}_{t, isen}}{\dot{W}_i} \implies \dot{W}_i = \frac{\dot{W}_{t,isen}}{\eta_{is}}
.\]
I opgaven får vi oplyst at $\eta_{is}= \num{0,7}$ for kompressoren. Derfor får vi
\[ 
\dot{W}_i = \frac{\qty{28,75}{kW}}{\num{0,7}} = \qty{41,0}{kW} 
.\]
Altså er akseleffekten for kompressoren $\dot{W}_i = \qty{41,0}{kW}$.

\newpage
\paragraph{4.} Bestem eleffekt i $[\unit{kW}]$ til drivmotoren samt strømstyrken i $[\unit{A}]$ i hver leder til motoren.
\begin{figure}[ht]
    \centering
    \incfig[0.75]{a5_3}
    \caption{Systemskitse af kompressoren og elmotoren}
\end{figure}
\bigbreak
Fra bogen har vi formlen
\[ 
  P_{el} = \frac{\dot{W}_i}{\eta_{mek} \eta_m}
.\]
Indsættes kendte størrelser kan el-effekten findes som
\[ 
P_{el} = \frac{\qty{41,0}{kW}}{\num{0,98} \cdot \num{0,95}} = \qty{44}{kW} 
.\]
Fra bogen ved vi også, at effekten $P_{el}$ kan kan findes som
\[ 
P_{el} = \sqrt{3}\cdot  I\cdot V\cdot \cos\phi
.\]
Heri kan strømstyrken isoleres som
\[ 
I = \frac{P_{el}}{\sqrt{3}\cdot V\cdot \cos\phi} = \frac{\qty{44}{kW}}{\sqrt{3}\cdot \qty{400}{V} \cdot \num{0,95}} = \qty{66,85}{A}
.\]


\paragraph{5.} Bestem den reelle temperatur i $[\unit{\celsius}]$ efter kompressoren.
\begin{figure}[ht]
    \centering
    \incfig{a5_4}
    \caption{Systemskitse af kompressoren}
\end{figure}
\bigbreak
Den isentropiske virkningsgrad kan defineres som
\[ 
\eta_{is} = \frac{h_{2, isen} - h_1}{h_{2, reel} - h_1} \implies h_{2, reel} = h_1 + \frac{h_{2, isen} - h_1}{\eta_{is}}
.\]
Efter at den reelle entalpi til sluttilstanden er fundet kan den reelle temperatur til sluttilstanden findes i EES vha. et stofdatakald. Dette er gjort som:
\begin{verbatim}
  P[1]=1,013 [bar]
  T[1]=20 [C]
  P[2]=10 [bar] + 1,013  [bar] 
 
  s[1] = entropy(Air_ha; T=T[1]; P=P[1])

  s[2] = entropy(Air_ha; T=T[2]; P=P[2])

  s[2] = s[1]

  h[1] = enthalpy(Air_ha;T=T[1];P=P[1])
  h[2] = enthalpy(Air_ha;T=T[2];P=P[2])

  eta_s = 0,7

  h_real[2] = h[1] + (h[2] - h[1])/eta_s

  T_reel[2]=temperature(Air_ha;P=P[2];h=h_real[2])
\end{verbatim}
Til dette giver EES $T_{reel,2} = \qty{417,8}{\celsius}$. 

\paragraph{6.} Hvor stor slagvolumen har stempelkompressoren, hvis den kører med \qty{3000}{omdrejninger/minut}. (Stofværdier $(C_p, \kappa)$ skal antages ikke konstante dvs. afhængige af temperatur of tryk -- brug entalpier eller middelværdier for $C_p, \kappa$)
\begin{figure}[ht]
    \centering
    \incfig[0.7]{a5_5}
    \caption{Systemskitse af kompressoren}
\end{figure}
\bigbreak
For at finde slagvolumenet af stempelkompressoren vil vi først bestemme volumenstrømmen der flyder ind i kompressoren. Her gælder generelt formlen:
\[ 
\dot{V} = \frac{\dot{m}}{\rho}
.\]
Idet det antages at kompressoren kun har et stempel må det gælde at denne volumenstrøm skal deles ud på alle kompressorens omdrejninger. Vi har altså
\[ 
V_{slag} = \frac{\dot{V}}{\mathrm{RPM}}
.\]
Dette er regnet i EES vha. følgende kode:
\begin{verbatim}
  P[1]=1,013 [bar]
  T[1]=20 [C]
  m_dot[1] = 0,1 [kg/s]
   
  rho[1] = density(Air_ha;T=T[1];P=P[1])
  V_dot[1] = m_dot[1]/rho[1]
   
  RPM[1] = 3000 [RPM]
  RPS[1] = 50 [RPS]
   
  V_slag[1] = V_dot[1]/RPS[1]
\end{verbatim}
Her giver EES outputtet $V_{slag, 1} = \qty{0,00166}{m^3} = \qty{1,66}{L}$.

\opgave{6.7}
Definér og bestem den højest opnåelige termiske virkningsgrad for et kraftværk som har en damptemperatur på \qty{550}{\celsius}. Kondenseringsvarmen køles med udeluft. Vejrudsigten siger blæst og dagtemperatur på \qty{10}{\celsius}.
\begin{figure}[ht]
    \centering
    \incfig[0.6]{a5_6}
    \caption{Systemskitse for kraftværket}
    \label{fig:a5_6}
\end{figure}
\bigbreak
Den højest opnåelige termiske virkningsgrad for et kraftværk kan findes vha. formlen for den termiske virkningsgrad for en Carnot-arbejdsproducerende maskine der virker i samme temperaturinterval. Denne er generelt givet som
\[ 
\eta_{th,C} = 1 - \frac{T_L}{T_H}
.\]
Hvis vi sætter kendte størrelser ind fås
\[ 
\eta_{th, C} = 1 - \frac{\qty{283}{K}}{\qty{823}{K}} = \num{0,656} 
.\]
Altså er den højest opnåelige termiske virkningsgrad for et kraftværk, der arbejdet i det nævnte temperaturinterval $\eta_{th,C} = \num{0,656} = 65,6\%$.
\end{document}
