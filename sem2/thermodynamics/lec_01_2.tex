\lecture{1-2}{30. Januar 2025}{Introduktion of termodynamikkens love}

\subsection{Termodynamiske processer}
En termodynamisk process er enhver proces, hvor der flyttes varme. Der findes mange forskellige termodynamiske processer, hvoraf nogle af dem er
\begin{itemize}
  \item Isotermisk proces -- Proces med konstant temperatur
  \item Isobar proces -- Proces med konstant tryk
  \item Isokor proces -- Proces med konstant volumen
  \item Adiabatisk proces -- Proces uden varmeudveksling med omgivelserne
  \item Isentalp proces -- Proces med konstant entalpi
  \item Isentrop proces -- Proces med konstant entropi
\end{itemize}

\subsection{Idealgasligningen}
Idealgasligningen er en tilstandsligning, der angiver en relation mellem tryk, temperatur og specifikt volumen for et specifikt materiale. Denne findes på et par forskellige former hvoraf 3 af dem er
\begin{align*}
  pV &= nR_u T \\
  pv &= R_i T \\
  pV &= m R_i T
.\end{align*}
Hvor $p$ er trykket, $V$ er volumenet, $v$ er det specifikke volumen, $n$ er stofmængden, $R_u$ er den universelle gaskonstant ($R_u = \qty{8,314}{\frac{J}{K \cdot mol}}$), $R_i$ er den individuelle gaskonstant (stofspecifik), $m$ er massen og $T$ er temperaturen.

Det kan bemærkes, at $R_i$ kan findes som
\[ 
R_i = \frac{R_u}{M}
\]
hvor $M$ er molarmassen af gassen.
