\lecture{12}{5. Marts 2025}{Composites, Pt. 1: Materials and general rules}

Consider a composite of carbon fibers ($\sigma_{f}^{*} = \qty{3,5}{GPa}$ and $d = \qty{0,005}{mm}$) in an epoxy matrix ($\tau_c = \qty{100}{MPa}$). 

\paragraph{1)} A fiber length of \qty{0,05}{mm} is \_\_\_\_\_\_ the critical fiber length.
\begin{itemize}
  \item Greater than
  \item Equal to
  \item Less than
\end{itemize}
\bigbreak
We can calculate the critical fiber length as
\[ 
l_c = \frac{\sigma_f d}{2 \cdot \tau_c} = \frac{\qty{3,5}{GPa} \cdot \qty{0,005}{mm}}{2 \cdot \qty{100}{MPa} } = \qty{0,088}{mm} 
.\]
So the answer is ``c) less than''.


\paragraph{2)} The fiber length must be greater than \_\_\_\_\_\_ to be considered continuous.
\begin{itemize}
  \item \num{0,0875} 
  \item \num{1,0} 
  \item \num{1,3125}
\end{itemize}
\bigbreak
As shown above the critical fiber length is $l_c = \qty{0,088}{mm}$ meaning the correct answer is ``a) \num{0,0875}''.


\paragraph{3)} To make the critical fiber length equal to \qty{0,05}{mm}, the diameter should be \_\_\_\_\_\_.
\begin{itemize}
  \item left unchanged
  \item decreased
  \item increased
\end{itemize}
\bigbreak
We reuse the formula
\[ 
l_c = \frac{\sigma_f d}{2 \tau_c} \implies d = \frac{2 l_c \tau_c}{\sigma_f}
.\]
Therefore we get
\[ 
d = \frac{2 \cdot \qty{0,05}{mm} \cdot \qty{100}{MPa}}{\qty{3,5}{GPa}} = \qty{0,00286}{mm}
.\]



\exercise{16.7} In an aligned and continuous carbon fiber-reinforced nylon \num{6,6} composite, the fibers are to carry 97\% of a load applied in the longtitudal direction.
\begin{table}[ht]
\centering
\caption{}
\begin{tabular}{c|c|c}
 & \textbf{Modulus of Elasticity} [\unit{GPa}] & \textbf{Tensile strength [\unit{MPa}]} \\ \hline
  Carbon fiber & 260 & 4000 \\
  Nylon \num{6,6} & \num{2,8} & \num{76} 
\end{tabular}
\end{table}

\paragraph{a.} Using the data provided, determine the volume fraction of fibers required.
\bigbreak
In an aligned continuous fiber reinforced composite the matrix and the fibers are under isostrain. We therefore have the conditions
\begin{align*}
  \sigma_c &= \sigma_m V_m + \sigma_f V_f \\
  \epsilon_c &= \epsilon_m = \epsilon_f
.\end{align*}
These two conditions can with the use of Hookes law be rewritten as
\begin{align*}
  \sigma_m &= E_m \epsilon_c \\
  \sigma_f &= E_f \epsilon_c
.\end{align*}
Thus
\[ 
\sigma_c = \epsilon_c \left( E_m V_m + E_f V_f \right)
.\]
The fraction of the total load $F_c$ carried by the fibers is
\[ 
\frac{F_f}{F_c} = \frac{\sigma_f V_f}{\sigma_c} = \frac{E_f \epsilon_c V_f}{\epsilon_c \left( E_m V_m + E_f V_f \right)} = \frac{E_f V_f}{E_m \left( 1- V_f \right) + E_f V_f}
.\]
We want this to be equal to 97\%, so
\[ 
\num{0,97} = \frac{E_f V_f}{E_m \left( 1- V_f \right) + E_f V_f}
.\]
We can now solve for $V_f$ as:
\begin{align*}
  E_f V_f &= \num{0,97} E_m \left( 1 - V_f \right) + \num{0,97} E_f V_f \\
  \num{0,03}  E_f V_f &= \num{0,97} E_m - \num{0,97} E_m V_f \\
  \num{0,03} E_f V_f + \num{0,97} E_m V_f &= \num{0,97} E_m \\
  V_f \left( \num{0,03} E_f + \num{0,97} E_m \right) &= \num{0,97} E_m \\
  V_f &= \frac{\num{0,97} E_m}{\num{0,03} E_f + \num{0,97} E_m} \\
  &= \frac{\num{0,97} \cdot \qty{2,8}{GPa}}{\num{0,03} \cdot \qty{260}{GPa} + \num{0,97} \cdot \qty{2,8}{GPa} } \\
  &= \num{0,2583} = \num{25,83} \% 
.\end{align*}


\paragraph{b.} What is the tensile strength of this composite? Assume that the matrix stress at fiber failure is \qty{50}{MPa}.
\bigbreak
The tensile strength of the composite is (as previously mentioned) given by
\[ 
\sigma_c = \sigma_m V_m + \sigma_f V_f = \qty{76}{MPa} \cdot \num{0,7417} + \qty{4000}{MPa} \cdot \num{0,2583} = \qty{1089,57}{MPa} 
.\]


\exercise{16.9} Compute the longtitudinal strength of an aligned carbon fiber-epoxy matrix copmosite having a \num{0,20} volume fraction of fibers, assuming the following:
\begin{itemize}
  \item An average fiber diameter of \qty{6e-3}{mm}
  \item An average fiber length of \qty{8,0}{mm}
  \item A fiber fracture strength of \qty{4,5}{GPa} 
  \item A fiber-matrix bond strength of \qty{75}{MPa}
  \item A matrix stress at composite failure of \qty{6,0}{MPa}
  \item A matrix tensile strength of \qty{60}{MPa} 
\end{itemize}
\bigbreak
We will first check if the fiber length exceeds the critical fiber length as
\[ 
l_c = \frac{\sigma_f^{*} d}{2 \tau_c} = \frac{\qty{4500}{MPa} \cdot \qty{6e-3}{mm}}{2 \cdot \qty{75}{MPa}} = \qty{0,18}{mm} 
.\]
Sinde $l = \qty{8,0}{mm} \gg l_c = \qty{0,18}{mm}$ the fibers are easily continuous. We can now use the strength formula for long fibers given by
\[ 
\sigma_{cd}^{*} = \sigma_f^{*}V_f \left( 1 - \frac{l_c}{2l} \right) + \sigma_m' \left( 1 - V_f \right)
.\]

We can substitute in the given values (we remember that $\sigma_m'$ is the matrix stress at the moment the composite fails which is often lower than the matrix’s standalone tensile strength) as
\[ 
\sigma_{cd}^{*} = \num{0,20} \cdot \qty{4500}{MPa} \cdot \left( 1 - \frac{\qty{0,18}{mm}}{2 \cdot \qty{8,0}{mm}} \right) + \num{0,80} \cdot  \qty{6,0}{Mpa} = \qty{894,7}{MPa} 
.\]

