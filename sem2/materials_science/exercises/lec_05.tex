\lecture{5}{10. Februar 2025}{Diffusion in solids}

\exercise{5.1}
Briefly explain the difference between self-diffusion and interdiffusion.
\bigbreak
Self-diffusion is the process of a material ``diffusing'' into itself. It can be thought of as molecules moving around in the crystal lattice in the sense that if you label a series of atoms sequentially in a lattice then after some time the sequential ordering will have broken down simply because the atom you labelled 1 or 2 will not be in the position corresponding to 1 or 2 and so on.

Interdiffusion is the process of a material diffusing into another. In the case of solids it could for example be interstituent diffusion of carbon into iron which can be used to make steel.

All in all the main difference between the 2 is that self-diffusion is diffusion of a solid into itself and interdiffusion is between solids/materials.

\exercise{5.3}
Carbon diffuses in iron via an interstitial mechanism -- for BCC iron from one tetrahedral site to an adjacent one. In Section 4.3 (Figure 4.3b) we note that a general set of point coordinates for this site are $1 \frac{1}{2} \frac{1}{4}$. Specify the family of crystallographic directions in which this diffusion of carbon in BCC iron takes place.
\bigbreak
Another tetrahedral site in the BCC lattice is at lattice position $1 \frac{1}{4} \frac{1}{2}$. To find the crystallographic direction between these two points we first find the difference in all three directions between the points as
\begin{align*}
\Delta x &= 0 \\
\Delta y &= - \frac{1}{2} \\
\Delta z &= \frac{1}{2}
.\end{align*}
We can reduce this to integer values to get $[0 \overline{1} 1]$. Another example would be $[0 1 \overline{1}]$ and so on.

\exercise{5.D2}
The wear resistance of a steel shaft is to be improved by hardening its surface by increasing the nitrogen content within an outer surface layer as a result of nitrogen diffusion into the steel; the nitrogen is to be supplied from an external nitrogen-rich gas at an elevated and constant temperature. The initial nitrogen content of the steel is \num{0,0025} wt\%, whereas the surface concentration is to be maintained at \num{0,45} wt\%. For this treatment to be effective, a nitrogen content of \num{0,12} wt\% must be established at a position \qty{0,45}{mm} below the surface. Specify an appropriate heat treatment in terms of temperature and time for a temperature between \qty{475}{\celsius} and \qty{625}{\celsius}. The preexponential and activation energy for the diffusion of nitrogen in iron are \qty{5e-7}{m^2 / s} and \qty{77000}{J/mol} respectively, over this temperature range.
\bigbreak
We start off with the solution to Fick's 2nd law
\[ 
\frac{C(x,t) - C_0}{C_s - C_0} = 1 - \mathrm{erf} \left( \frac{x}{2 \sqrt{Dt}} \right)
.\]
We have that
\begin{align*}
  C_0 &= \num{0,0025} \%  \\
  C_s &= \num{0,45} \% \\
  C(x,t) &= \num{0,12} \% \\
  x &= \qty{0,45}{mm}  
.\end{align*}
Thus we have that
\[ 
\frac{\num{0,12} - \num{0,0025}}{\num{0,45} - \num{0,0025}} = 1 - \mathrm{erf}(z)
\]
with $z = \frac{x}{2 \sqrt{Dt}}$. We get that
\[ 
\mathrm{erf}(z) = \num{0,737} \implies z = \num{0,791} 
.\]
Therefore we get
\[ 
\num{0,791} = \frac{x}{2 \sqrt{Dt}}
.\]
We can solve for $Dt$ as
\begin{align*}
  \num{0,791}  &= \frac{x}{2 \sqrt{Dt}} \\
  \frac{1}{\sqrt{Dt}} &= \frac{2 \cdot \num{0,791}}{x} \\
  Dt &= \left( \frac{\qty{0,45}{mm}}{2 \cdot \num{0,791}} \right)^2 \\
  Dt &= \qty{0,081}{mm^2} 
.\end{align*}
We also know that $D$ can be expressed as
\[ 
D = D_0 e^{- \frac{Q_d}{RT}}
.\]
We know the values of $D_0 = \qty{5e-7}{\frac{m^2}{s}}$ and $Q_d = \qty{77000}{\frac{J}{mol}}$. The entire expression becomes
\[ 
t \cdot D_0 e^{- \frac{Q_d}{RT}} = \qty{0,081}{mm^2} 
.\]
By choosing $T = \qty{600}{\celsius} = \qty{873}{K}$ we can calculate the exponential
\[ 
  e^{- \frac{\qty{77000}{\frac{J}{mol}}}{R \cdot \qty{873}{K}}} = \num{2,47e-5} 
.\]
Therefore we have
\begin{align*}
  t &= \frac{\qty{0,081}{mm^2}}{\num{2,47e-5} \cdot D_0} \\
  &= \frac{\qty{0,081}{mm^2}}{\num{2,47e-5} \cdot \qty{5e-7}{\frac{m^2}{s}}} \\
  &= \qty{6559}{s} = \qty{109}{min} = \qty{1,82}{h}
.\end{align*}
Therefore it should be appropriate to lat the heat treatment run for about \qty{109}{min} at \qty{600}{\celsius}. 
